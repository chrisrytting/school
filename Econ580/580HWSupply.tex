\documentclass[letterpaper,12pt]{article}

\usepackage{threeparttable}
\usepackage{geometry}
\geometry{letterpaper,tmargin=1in,bmargin=1in,lmargin=1.25in,rmargin=1.25in}
\usepackage[format=hang,font=normalsize,labelfont=bf]{caption}
\usepackage{amsmath}
\usepackage{mathrsfs}
\usepackage{multirow}
\usepackage{array}
\usepackage{delarray}
\usepackage{listings}
\usepackage{amssymb}
\usepackage{amsthm}
\usepackage{lscape}
\usepackage{natbib}
\usepackage{setspace}
\usepackage{float,color}
\usepackage[pdftex]{graphicx}
\usepackage{pdfsync}
\usepackage{verbatim}
\usepackage{placeins}
\usepackage{geometry}
\usepackage{pdflscape}
\synctex=1
\usepackage{hyperref}
\hypersetup{colorlinks,linkcolor=red,urlcolor=blue,citecolor=red}
\usepackage{bm}


\theoremstyle{definition}
\newtheorem{theorem}{Theorem}
\newtheorem{acknowledgement}[theorem]{Acknowledgement}
\newtheorem{algorithm}[theorem]{Algorithm}
\newtheorem{axiom}[theorem]{Axiom}
\newtheorem{case}[theorem]{Case}
\newtheorem{claim}[theorem]{Claim}
\newtheorem{conclusion}[theorem]{Conclusion}
\newtheorem{condition}[theorem]{Condition}
\newtheorem{conjecture}[theorem]{Conjecture}
\newtheorem{corollary}[theorem]{Corollary}
\newtheorem{criterion}[theorem]{Criterion}
\newtheorem{definition}{Definition} % Number definitions on their own
\newtheorem{derivation}{Derivation} % Number derivations on their own
\newtheorem{example}[theorem]{Example}
\newtheorem{exercise}[theorem]{Exercise}
\newtheorem{lemma}[theorem]{Lemma}
\newtheorem{notation}[theorem]{Notation}
\newtheorem{problem}[theorem]{Problem}
\newtheorem{proposition}{Proposition} % Number propositions on their own
\newtheorem{remark}[theorem]{Remark}
\newtheorem{solution}[theorem]{Solution}
\newtheorem{summary}[theorem]{Summary}
\bibliographystyle{aer}
\newcommand\ve{\varepsilon}
\renewcommand\theenumi{\roman{enumi}}
\newcommand\norm[1]{\left\lVert#1\right\rVert}

\begin{document}

\title{580 Homework Supply}
\author{Chris Rytting}
\maketitle
\subsection*{1}
Let $y \in Y \text{ and } y'\leq y$. Also, $y' \not \in -\mathbb{R}_+^L$. Assume by way of contradiction that $y' \not \in Y$. Since $Y$ is closed and convex, by theorem M.G.2, we have that there is a $p \in \mathbb{R}^L$ with $p \neq 0$ and $c \in \mathbb{R}$ such that $p \cdot y' > C \text{ and } p \cdot y < C \quad \forall y \in Y$. We know that $P \geq 0$ since $-\mathbb{R}_+^L$, implying $p \cdot y' > p \cdot y$, which is a contradiction since $y' \leq y$, and we have the desired result.

\subsection*{2}

This is a close analogue to the UMP, so we can use Roy's identity to yield $z_1$ and we have
\[z_1 = - \frac{\frac{\partial R }{\partial w_1}}{\frac{\partial R} {\partial C}}= -\frac{\frac{-p \alpha}{w_1} }{\frac{p}{C}} = \frac{\alpha C}{w_1} \]

\subsection*{3 (a)}
We will choose whichever good is cheaper since they are perfect substitutes. Let $i$ be said good. Then we have that
\[z_i = q\]
\[z_\neq i = 0\]
and we have that
\[C = w_iz_i\]


\subsection*{3 (b)}

We have that $z_1 = q, z_2 = q$, and this yields 
\[C = q(w_1 + w_2) \]

\subsection*{3 (c)}

We have
\[ \mathscr{L} = z_iw_i + z_2w_2 + \lambda (f-(z_1^{\rho} + z_2^{\rho})^{\rho}) \]
FOC's
\begin{align*}
    \frac{\partial \mathscr{L} }{\partial z_1} &= w_1 - \lambda \frac{\rho}{\rho} z_1^{\rho -1}(z_1^{\rho} + z_2^{\rho})^{\frac{1}{\rho}-1} \\
    \frac{\partial \mathscr{L} }{\partial z_2} &= w_2 - \lambda \frac{\rho}{\rho} z_2^{\rho -1}(z_1^{\rho} + z_2^{\rho})^{\frac{1}{\rho}-1} \\
    \frac{\partial \mathscr{L} }{\partial \lambda} &= (z_1^{\rho} + z_2^{\rho})^{\frac{1}{\rho}}
\end{align*}
Thus, we have that
\[\frac{w_1}{w_2} = \frac{z_1^{\rho-1}}{z_1^{\rho-1}}\]
giving us that:
\begin{align*}
    z_1 &= z_2 \left( \frac{w_1}{w_2} \right)^{\frac{1}{\rho-1}} \\
    z_2 &= z_1 \left( \frac{w_2}{w_1} \right)^{\frac{1}{\rho-1}} \\
    C &= w_1^{\rho / \rho - 1}z_2 \left( \frac{1}{w_2} \right)^{\frac{1}{\rho-1}} + w_2^{\rho / \rho - 1}z_1 \left( \frac{1}{w_1} \right)^{\frac{1}{\rho-1}} 
\end{align*}

\subsection*{4}

Let the cost function before the increase in $p$ be given by
\[C(q) = z_1 \bar w_1 + z_2 \bar w_2 + \cdots + z_{L-1} \bar w_{L-1}\] 
and the cost function after the increase in $p$ (in the short run) be given by
\[C_{SR}(q) = z_1 \bar w_1 + z'_2 \bar w_2 + \cdots + z'_{L-1} \bar w_{L-1}\] 
and the cost function after the increase in $p$ (in the long run) be given by
\[C_{LR}(q) = z'_1 \bar w_1 + z'_2 \bar w_2 + \cdots + z'_{L-1} \bar w_{L-1}\] 
Now, since the $z$'s are functions of $w$ and $p$ alone, we know that an increase in $p$ will result in an increase in each $z$ (or in no increase in the case that $z$'s are complements of one another in the production of good $L$, in which case this proof is trivial as no factor increases in the short run and they all increase in the long run).\\\\
Proceeding with the general case, though, we know that the overall use of inputs $Z$ is given by
\[Z = z_1 + z_2 + \cdots + z_{L-1} \]
and the change in inputs in the short run is given by 
\[\Delta_{SR} = Z_{SR} - Z = \sum^{L-1}_{i=2} z'_i - z_i \]


and the change in inputs in the long run is given by 
\[\Delta_{LR} = Z_{LR} - Z = \sum^{L-1}_{i=1} z'_i - z_i \]

and the difference of these is given by 
\[ \Delta_{LR} - \Delta_{SR} = z'_1 - z_1 > 0 \]
since the $z'_l$'s are greater than the $z'_l$'s, and we have the desired result.

\subsection*{5 (i)}
Yes. As $z$ increases, $q$ also increases. The same holds for marginal product.


\subsection*{5 (ii)}
Essentially, the requirement is that marginal benefit equals marginal cost, which needs be the case for a maximization problem.\\\\
If marginal cost were higher, we would want to engage in less trade, and if marginal cost were lower, we would want to engage in more trade.

\subsection*{5 (iii)}
The requirement is similar as in (ii). The solution of a maximization problem is its optimal point. Therefore, if marginal benefit equals marginal cost, we have our optimal solution.

\subsection*{6 (a)}

Have the firm with the lowest $\beta_j$ produce all goods, as this will increase marginal cost the least.


\subsection*{6 (b)}

Have the firm with the lowest $\beta_j$ produce all goods, as this will decrease marginal cost the least.

\subsection*{6 (c)}
Where $\beta_j > 0$:\\\\

Have the firm with the lowest $\beta_j$ produce all goods, as this will increase marginal cost the least.\\\\




Where $\beta_j < 0$:\\\\

Have the firm with the lowest $\beta_j$ produce all goods, as this will decrease marginal cost the least.









\end{document}
