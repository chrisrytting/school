
\documentclass[letterpaper,12pt]{article}

\usepackage{threeparttable}
\usepackage{geometry}
\geometry{letterpaper,tmargin=1in,bmargin=1in,lmargin=1.25in,rmargin=1.25in}
\usepackage[format=hang,font=normalsize,labelfont=bf]{caption}
\usepackage{amsmath}
\usepackage{mathrsfs}
\usepackage{multirow}
\usepackage{array}
\usepackage{delarray}
\usepackage{listings}
\usepackage{amssymb}
\usepackage{amsthm}
\usepackage{lscape}
\usepackage{natbib}
\usepackage{setspace}
\usepackage{float,color}
\usepackage[pdftex]{graphicx}
\usepackage{pdfsync}
\usepackage{verbatim}
\usepackage{placeins}
\usepackage{geometry}
\usepackage{pdflscape}
\synctex=1
\usepackage{hyperref}
\hypersetup{colorlinks,linkcolor=red,urlcolor=blue,citecolor=red}
\usepackage{bm}


\theoremstyle{definition}
\newtheorem{theorem}{Theorem}
\newtheorem{acknowledgement}[theorem]{Acknowledgement}
\newtheorem{algorithm}[theorem]{Algorithm}
\newtheorem{axiom}[theorem]{Axiom}
\newtheorem{case}[theorem]{Case}
\newtheorem{claim}[theorem]{Claim}
\newtheorem{conclusion}[theorem]{Conclusion}
\newtheorem{condition}[theorem]{Condition}
\newtheorem{conjecture}[theorem]{Conjecture}
\newtheorem{corollary}[theorem]{Corollary}
\newtheorem{criterion}[theorem]{Criterion}
\newtheorem{definition}{Definition} % Number definitions on their own
\newtheorem{derivation}{Derivation} % Number derivations on their own
\newtheorem{example}[theorem]{Example}
\newtheorem{exercise}[theorem]{Exercise}
\newtheorem{lemma}[theorem]{Lemma}
\newtheorem{notation}[theorem]{Notation}
\newtheorem{problem}[theorem]{Problem}
\newtheorem{proposition}{Proposition} % Number propositions on their own
\newtheorem{remark}[theorem]{Remark}
\newtheorem{solution}[theorem]{Solution}
\newtheorem{summary}[theorem]{Summary}
\bibliographystyle{aer}
\newcommand\ve{\varepsilon}
\renewcommand\theenumi{\roman{enumi}}
\newcommand\norm[1]{\left\lVert#1\right\rVert}

\begin{document}

\title{Math 320 Homework 5.1}
\author{Chris Rytting}
\maketitle

\subsection*{5.1}


We know that 
\[B^{k}_{n} = \binom nk (1-x)^{n-k} x^k \]
Differentiating, we get
\[\binom nk (kx^{k-1}(1-x)^{n-k}-(n-k)x^k(1-x)^{n-k-1}) = 0\]
\[\implies k(1-x)-(n-k)x = 0\]
\[\implies k-kx-nx+kx = 0\]
and we have that $x = n/k$.

\subsection*{5.2}


\[B_n[f](0) = \sum_{k=0}^n f(k/n)B^n_k(0)\]
\[=\sum_{k=1}^{n}f(k/n)*0 + f(0) = f(0)\]
\[B_n[f](1) = \sum_{k=0}^{n-1} f(k/n)B_k^n(1)\]
\[=\sum_{k=0}^{n-1}f(k/n)*0+f(1)*1\]
and by lemma 5.1.2
\[= f(1)\]
as desired.

\subsection*{5.3 (i)}
\begin{align*}
B_n[1] &= \sum ^n _{k=0} 1 \binom nk x^k (1-x)^{n-k} \\
&= \sum^n_{k=0}\binom nk x^k (1-x)^{n-k}
\end{align*}
We know, though, that the Bernstein polynomials sum to 1 and we have the desired result.

\subsection*{5.3 (ii)}


\begin{align*}
    B_n[x] &= \sum_{k=0}^n \frac{k}{n} \frac{n!}{k!(n-k)!} x^k (1-x)^{n-k} \\
    &= \sum_{k=0}^n \frac{(n-1)!}{n(k-1)!(n-k)!} x^k (1-x)^{n-k} \\
    &= \sum_{k=0}^n x \binom{n-1}{k-1} x^k (1-x)^{n-k} \\
    &= x \sum_{k=0}^n \text{B}_{k-1}^{n-1}(x)\\
    &= x
\end{align*}
since the Bernstein polynomials sum to one.

\subsection*{5.3 (iii)}

\begin{align*}
    B_n[x^2] &= \sum^{n}_{k = 0}  f(\frac{k}{n}) B_k^n(x) \\
    &WTS\\
    \sum^{n}_{k = 0}  \frac{k^2}{n^2} \frac{n!}{(k!(n-k)!}x^k(1-x)^{n-k} &= x^2 + \frac{x-x^2}{n} \\
    \sum^{n}_{k = 0}  \frac{k}{n} \frac{(n-1)!}{( k-1)!(n-k)!}x^k(1-x)^{n-k} &= x^2 + \frac{x-x^2}{n} \\
    \sum^{n}_{k = 0}  k \frac{(n-1)!}{( k-1)!(n-k)!}x^k(1-x)^{n-k} &= nx^2 + x-x^2 \\
    \sum^{n}_{k = 0}  k \frac{(n-1)!}{( k-1)!(n-k)!}x^k(1-x)^{n-k} &= (n-1)x^2 + x \\
    x\sum^{n}_{k = 1}  k \binom {n-1}{k-1}x^{k-1}(1-x)^{n-k} &= (n-1)x^2 + x \\
\end{align*}
Now, letting $j = k-1$, we have
\begin{align*}
    x\sum^{n-1}_{j = 0}  (j+1) \binom {n-1}{j}x^{j}(1-x)^{n-1-j} &= (n-1)x^2 + x \\
\end{align*}
which yields, by the binomial theorem and distributing through,
\begin{align*}
    &=x\sum^{n-1}_{j = 0}  (j) \binom {n-1}{j}x^{j}(1-x)^{n-1-j} + x \\
    &=(n-1)x^2 + x \\
\end{align*}
which is the desired result.

\subsection*{5.4}
We have the system of equations:
\[
\begin{bmatrix}
    1 & -1 & 1 & -1 \\
    1 & 0 & 0 & 0 \\
    1 & 1 & 1 & 1 \\
    1 & 2 & 4 & 8
\end{bmatrix}
\begin{bmatrix}
    a_0 \\ a_1 \\ a_2 \\ a_3
\end{bmatrix}
=
\begin{bmatrix}
    2 \\ -4 \\ -6 \\ 16
\end{bmatrix}
\]
And by applying the inverse to both sides we get
\[
\begin{bmatrix}
    a_0 \\ a_1 \\ a_2 \\ a_3
\end{bmatrix}
=
\begin{bmatrix}
    0 & 1 & 0 & 0 \\
    -1/3 & -1/2 & 1 & -1/6 \\
    1/2 & -1 & 1/2 & 0 \\
    -1/6 & 1/2 & -1/2 & 1/6
\end{bmatrix}
\begin{bmatrix}
    2 \\ -4 \\ -6 \\ 16
\end{bmatrix}
=
\begin{bmatrix}
    -4 \\ -2 \\ 2 \\ -2
\end{bmatrix}
\]
Yielding the polynomial 
\[p_3(x) = -2x^3 + 2x^2 - 2x -4\]

\subsection*{5.5}
Given that $p_3(x) = \sum_{j=0}^{3} f(x_j) L_{3,j}$.\\
\\
We have $x=(-1, 0,1,2)$ and $f(x)=(2,-4,-6,-16)$ we find $L_{3,j}$:
\begin{align*}
    L_{3,0} &= \left( \frac{x}{-1} \right)\left( \frac{x-1}{-2} \right)\left( \frac{x-2}{-3} \right) \\
    L_{3,1} &= \left( \frac{x+1}{1} \right)\left( \frac{x-1}{-1} \right)\left( \frac{x-2}{-2} \right) \\
    L_{3,2} &= \left( \frac{x+1}{2} \right)\left( \frac{x}{1} \right)\left( \frac{x-2}{-1} \right) \\
    L_{3,3} &= \left( \frac{x+1}{3} \right)\left( \frac{x}{2} \right)\left( \frac{x-1}{1} \right) 
\end{align*}
Yielding
\[ p_3(x) = \frac{-2}{6}\big(x(x-1)(x-2)\big) - \frac{4}{2}\big( (x+1)(x-1)(x-2)\big) + \frac{6}{2}\big( (x+1)(x)(x-2)\big) - \frac{16}{6}\big( (x+1)(x)(x-1)\big)  \]
Simplifying, we get
\[p_3(x) = -2x^3 + 2x^2 - 2x -4\]
\end{document}
