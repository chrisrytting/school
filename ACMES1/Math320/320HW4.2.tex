\documentclass[letterpaper,12pt]{article}

\usepackage{threeparttable}
\usepackage{geometry}
\geometry{letterpaper,tmargin=1in,bmargin=1in,lmargin=1.25in,rmargin=1.25in}
\usepackage[format=hang,font=normalsize,labelfont=bf]{caption}
\usepackage{amsmath}
\usepackage{mathrsfs}
\usepackage{multirow}
\usepackage{array}
\usepackage{delarray}
\usepackage{listings}
\usepackage{amssymb}
\usepackage{amsthm}
\usepackage{lscape}
\usepackage{natbib}
\usepackage{setspace}
\usepackage{float,color}
\usepackage[pdftex]{graphicx}
\usepackage{pdfsync}
\usepackage{verbatim}
\usepackage{placeins}
\usepackage{geometry}
\usepackage{pdflscape}
\synctex=1
\usepackage{hyperref}
\hypersetup{colorlinks,linkcolor=red,urlcolor=blue,citecolor=red}
\usepackage{bm}


\theoremstyle{definition}
\newtheorem{theorem}{Theorem}
\newtheorem{acknowledgement}[theorem]{Acknowledgement}
\newtheorem{algorithm}[theorem]{Algorithm}
\newtheorem{axiom}[theorem]{Axiom}
\newtheorem{case}[theorem]{Case}
\newtheorem{claim}[theorem]{Claim}
\newtheorem{conclusion}[theorem]{Conclusion}
\newtheorem{condition}[theorem]{Condition}
\newtheorem{conjecture}[theorem]{Conjecture}
\newtheorem{corollary}[theorem]{Corollary}
\newtheorem{criterion}[theorem]{Criterion}
\newtheorem{definition}{Definition} % Number definitions on their own
\newtheorem{derivation}{Derivation} % Number derivations on their own
\newtheorem{example}[theorem]{Example}
\newtheorem{exercise}[theorem]{Exercise}
\newtheorem{lemma}[theorem]{Lemma}
\newtheorem{notation}[theorem]{Notation}
\newtheorem{problem}[theorem]{Problem}
\newtheorem{proposition}{Proposition} % Number propositions on their own
\newtheorem{remark}[theorem]{Remark}
\newtheorem{solution}[theorem]{Solution}
\newtheorem{summary}[theorem]{Summary}
\bibliographystyle{aer}
\newcommand\ve{\varepsilon}
\renewcommand\theenumi{\roman{enumi}}
\newcommand\norm[1]{\left\lVert#1\right\rVert}

\begin{document}

\title{Math 320 Homework 4.2}
\author{Chris Rytting}
\maketitle


\subsection*{4.8}


\begin{align*}
    \mathscr{F}(f(t)) & = \int^b_a e^{i \xi t} dt - \int _b ^c e^{i \xi t}dt \\
    & = \frac{e^{i \xi t}}{i \xi} \big| ^b _a -  \frac{e^{i \xi t}}{i \xi}\big| ^c _b  \\
    & = \frac{2e^{i\xi b} - e^{i\xi a} - e^{i\xi c}}{i \xi}
\end{align*}
        which is the desired result.

\subsection*{4.9}


If we let $u = at$, then we have
\begin{align*}
    f(at) & = \int^\infty _{-\infty} e^{i \xi t} f(at) dt \\
    & = \frac{1}{a}\int^\infty _{-\infty} e^{i \xi \frac{u}{a}} f(u) du \\
    & = \frac{1}{a} \hat f(\frac{\xi}{a})
\end{align*}
        which is the desired result.

\subsection*{4.10}

It suffices to show equality in the other direction. Note then, that

\begin{align*}
    \frac{1}{2\pi} \hat f(-\xi) & = \frac{1}{2 \pi} \int ^\infty_{-\infty} e^{-i(-\xi )t}f(t)dt \\
    & =  \frac{1}{2 \pi} \int ^\infty_{-\infty} e^{i\xi t}f(t)dt \\
    & = \mathscr{F}^{-1}(f)\\
\end{align*}
        which is the desired result.

\subsection*{4.11}

Again, it suffices to show equality in the other direction. Note then, that

\begin{align*}
    \hat f(\xi -a) & = \int ^\infty_{-\infty} e^{-i(\xi-a) t}f(t)dt \\
    & = e^{iat} \int ^\infty_{-\infty} e^{-i\xi t}f(t)dt \\
    & = e^{iat} \hat f(\xi) \\
    & = \mathscr{F}(e^{iat}f(t))
\end{align*}
        which is the desired result.

\subsection*{4.12}


We have that
\[\hat f(\xi) = \mathscr{F}(f(t)) = \int^\infty_{-\infty} e^{-i\xi t} f(t)dt\]
Therefore, we also have that
\begin{align*}
    \mathscr{F}(\overline{f(t)})&  = \int^\infty_{-\infty} e^{-i\xi t} \overline{f(t)}dt \\
    & = \overline{\int^\infty_{-\infty} e^{-i(-\xi) t} f(t)dt} \\
    & = \overline{\hat f(-\xi)}
\end{align*}
        which is the desired result.



\subsection*{4.13 (i)}
By letting \[u = t - \tau\]
we have
        \begin{align*}
            &\int^\infty_{-\infty} f(\tau)g(t-\tau)d\tau \\
            & = -\int^{-\infty}_\infty f(t-u)g(u) du \\
            & = \int^\infty_{-\infty} f(t-u)g(u) du
        \end{align*}
        which is the desired result.
\subsection*{4.13 (ii)}
Note that
        \begin{align*}
            f*(g*h) & = \int_{-\infty}^\infty f(t-\tau_1) \int_{-\infty}^\infty g(\tau-\nu)h(\nu)d\nu d\tau \\
            & = \int_{-\infty}^\infty \int_{-\infty}^\infty f(t-\tau)g(\tau - \nu)h(\nu)d\nu d\tau \\
            \\
             \text{Now, if we let }~\tau - \nu &= \tau_0, \text{ then we have} \\
            \\
            & = \int_{-\infty}^\infty \int_{-\infty}^\infty f(t-\tau_0 - \nu)g(\tau_0)h(\nu) d\tau d\nu \\
            & = \int_{-\infty}^\infty h(\nu) \int_{-\infty}^\infty f(t-\nu - \tau_0)g(\tau_0) d\tau d\nu \\
            & = h * (f*g) \\
            & = (f * g) * h
        \end{align*}
        which is the desired result.
\subsection*{4.13 (iii)}
    \begin{align*}
    f*(\alpha g + \beta h) &= \int^{\infty}_{-\infty} f(t - \tau) (\alpha g(\tau) + \beta h(\tau)) d\tau \\
    &= \int^{\infty}_{-\infty} \alpha f(t - \tau) g(\tau) d\tau + \int^{\infty}_{-\infty} \beta f(t - \tau) h(\tau) d\tau \\
    &= (\alpha f * g) + (\beta f * h)    
    \end{align*}
        which is the desired result.



\subsection*{4.14 (i)}
Note that
        \begin{align*}
            \mathscr{F}(f*g) &= \int_{-\infty}^\infty  e^{-i \xi t} (f*g)(t) dt \\
            & =   \int_{-\infty}^\infty e^{-i \xi t} \int_{-\infty}^\infty f(\tau) g(t-\tau)d\tau dt \\
            & = \int_{-\infty}^\infty \int_{-\infty}^\infty e^{-i \xi t} f(\tau) g(t-\tau) d\tau dt \\\\
            & \text{ Now, letting }s = t - \tau\quad t = s+\tau \quad dt = ds, \text{ we have that } \\\\
            & = \int_{-\infty}^\infty \int_{-\infty}^\infty e^{-i \xi (s+\tau)} f(\tau) g(s) d\tau ds \\
            & = \int_{-\infty}^\infty \int_{-\infty}^\infty e^{-i \xi \tau}f(\tau) \cdot e^{-i \xi s} g(s) d\tau ds \\
            & = \int_{-\infty}^\infty e^{-i \xi \tau} f(\tau) d\tau \cdot \int_{-\infty}^\infty  e^{-i\xi s } g(s)ds \\
            & = \mathscr{F}(f) \cdot \mathscr{F}(g)
        \end{align*}
        which is the desired result.

\subsection*{4.14 (ii)}
        \begin{align*}
            \mathscr{F}(g)* \mathscr{F}(f) & = \int_{-\infty}^\infty \int_{-\infty}^\infty f(x)e^{-i\tau x} dx \int_{-\infty}^\infty g(y) e^{-i(t-\xi)y}dyd\tau \\
            & = \int_{-\infty}^\infty \int_{-\infty}^\infty \int_{-\infty}^\infty f(x)g(y)e^{-i \tau x}e^{-i t y}e^{i \tau y} dx dy d\tau \\
            & =2\pi  \int_{-\infty}^\infty g(y) \frac{1}{2\pi} \int_{-\infty}^\infty   \int_{-\infty}^\infty
            f(x)e^{-i \tau x} e^{i\tau y} d\tau dx e^{-i t y }dy \\
            & \text{ Now, as this is the inverse of the inverse, we have that  } \\
            & = 2 \pi \int_{-\infty}^\infty g(y)f(y)e^{-it y} dy \\
            & = \mathscr{F}(fg)
        \end{align*}
        which is the desired result.



\end{document}
