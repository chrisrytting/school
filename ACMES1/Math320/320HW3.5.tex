\documentclass[letterpaper,12pt]{article}

\usepackage{threeparttable}
\usepackage{geometry}
\geometry{letterpaper,tmargin=1in,bmargin=1in,lmargin=1.25in,rmargin=1.25in}
\usepackage[format=hang,font=normalsize,labelfont=bf]{caption}
\usepackage{amsmath}
\usepackage{mathrsfs}
\usepackage{multirow}
\usepackage{array}
\usepackage{delarray}
\usepackage{listings}
\usepackage{amssymb}
\usepackage{amsthm}
\usepackage{lscape}
\usepackage{natbib}
\usepackage{setspace}
\usepackage{float,color}
\usepackage[pdftex]{graphicx}
\usepackage{pdfsync}
\usepackage{verbatim}
\usepackage{placeins}
\usepackage{geometry}
\usepackage{pdflscape}
\synctex=1
\usepackage{hyperref}
\hypersetup{colorlinks,linkcolor=red,urlcolor=blue,citecolor=red}
\usepackage{bm}


\theoremstyle{definition}
\newtheorem{theorem}{Theorem}
\newtheorem{acknowledgement}[theorem]{Acknowledgement}
\newtheorem{algorithm}[theorem]{Algorithm}
\newtheorem{axiom}[theorem]{Axiom}
\newtheorem{case}[theorem]{Case}
\newtheorem{claim}[theorem]{Claim}
\newtheorem{conclusion}[theorem]{Conclusion}
\newtheorem{condition}[theorem]{Condition}
\newtheorem{conjecture}[theorem]{Conjecture}
\newtheorem{corollary}[theorem]{Corollary}
\newtheorem{criterion}[theorem]{Criterion}
\newtheorem{definition}{Definition} % Number definitions on their own
\newtheorem{derivation}{Derivation} % Number derivations on their own
\newtheorem{example}[theorem]{Example}
\newtheorem{exercise}[theorem]{Exercise}
\newtheorem{lemma}[theorem]{Lemma}
\newtheorem{notation}[theorem]{Notation}
\newtheorem{problem}[theorem]{Problem}
\newtheorem{proposition}{Proposition} % Number propositions on their own
\newtheorem{remark}[theorem]{Remark}
\newtheorem{solution}[theorem]{Solution}
\newtheorem{summary}[theorem]{Summary}
\bibliographystyle{aer}
\newcommand\ve{\varepsilon}
\renewcommand\theenumi{\roman{enumi}}
\newcommand\norm[1]{\left\lVert#1\right\rVert}

\begin{document}

\title{Math 320 Homework 3.5}
\author{Chris Rytting}
\maketitle

\subsection*{3.27}
\begin{align*}
E[X] &= \int^{b}_{a} x \frac{1}{b-a}dx \\
&= \frac{x^2}{2(b-a)}\Big|^b_a \\
&= \frac{b^2 -a^2}{2(b-a)} \\
&= \frac{(a+b)(b-a)}{2(b-a)} \\
&= \frac{a+b}{2}
\end{align*}

\begin{align*}
    V(X) &= E[X^2] - E[X]^2 \\
    &= \int^{b}_{a} \frac{x^2}{b-a} - \left( \frac{b+a}{2} \right)^2\\
    &= \frac{x^3}{3(b-a)}\Big|^b_a - \left( \frac{b+a}{2} \right)^2\\
    &= \frac{b^3 - a^3}{3(b-a)} - \left( \frac{b+a}{2} \right)^2\\
    &= \frac{(b - a)(a^2 + ab + b_2)}{3(b-a)} - \left( \frac{b+a}{2} \right)^2\\
    &= \frac{(a^2 + ab + b_2)}{3} -  \frac{b^2 + 2ab +a^2}{4} \\
    &= \frac{(4a^2 + 4ab + 4b_2)}{3} -  \frac{3b^2 + 6ab +3a^2}{4} \\
    &= \frac{(4a^2 + 4ab + 4b_2) -  3b^2 - 6ab - 3a^2}{12} \\
    &= \frac{(a^2 -2ab + b_2)  }{12} \\
    &= \frac{(b-a)^2 }{12} \\
\end{align*}

\subsection*{3.28 (i)}

\[ \int^{5}_{0} \frac{2}{15}e^{\frac{-2x}{15}} dx = .48658\]

\subsection*{3.28 (ii)}

\[ 1 - \int^{15}_{0} \frac{2}{15} e^{\frac{-2x}{15}} dx = .135335\]

\subsection*{3.28 (iii)}

\[ \int^{15}_{0} \frac{(\frac{2}{15})^3x^2e^{\frac{-2x}{15}}}{\Gamma(3)} = .32332 \]

\subsection*{3.29 (i)}

We will maximize the p.d.f. to do so:
\[\frac{\partial}{\partial x} \frac{1}{\sqrt{2 \pi \sigma^2}} \text{exp} \left( - \frac{(x-\mu)^2}{2 \sigma^2} \right) = \frac{1}{\sqrt{2 \pi \sigma^2}} \left(- \frac{2(x-\mu)}{2 \sigma^2} \right) 
\text{exp} \left( - \frac{(x-\mu)^2}{2 \sigma^2} \right) \]
Which equals 0 at $x=\mu$, meaning that the p.d.f. is maximized at this point.

\subsection*{3.29 (ii)}

Maximizing p.d.f., we have
\[\frac{\partial f_X(x)}{\partial x} = \frac{b^a(a-1)x^{a-2}e^{-xb} - b^ax^{a-1}be^{-xb}}{\Gamma(a)} = 0 \]
yields
\[ b^a(a-1)x^{a-2}e^{-xb} = b^ax^{a-1}be^{-xb} \]
yielding
\[x = \frac{a-1}{b} < \frac{a}{b} = \mu\]
So the mode is less than the mean, as desired.

\subsection*{3.29 (iii)}
\[\frac{\partial f_X(x)}{\partial x} = \frac{\Gamma(a+b)}{\Gamma(a)\Gamma(b)} \left( (a-1)x^{a-2}(1-x)^{b-1} - x^{a-1}(b-1)(1-x)^{b-2} \right)\]
yields
\[ (a-1)x^{a-2}(1-x)^{b-1} = x^{a-1}(b-1)(1-x)^{b-2} \]
yielding
\[x = \frac{a-1}{a+b-2}\]
which is less than $\mu$ when $a=1, b=2$ and
greater than $\mu$ when $a=500, b=1$.

\subsection*{3.30 (i)}

We have that the normal distribution is given by 
\[\int^{\mu + k \sigma}_{\mu - k \sigma} \frac{1}{\sqrt{2 \pi \sigma^2}} e^{-\frac{(x - \mu)^2}{2\sigma^2}}\]
Now, making the substitution $u = \frac{x-\mu}{\sigma}$, and finding our new bounds given by
\[ \frac{\mu + k \sigma - \mu}{\sigma} = k \text{ and } \frac{\mu - k \sigma - \mu}{\sigma} = -k\]
And our new integral is 
\[ \frac{1}{\sigma\sqrt{2 \pi }}\int^{k}_{-k} \sigma e^{-\frac{u^2}{2}} = \frac{1}{\sqrt{2 \pi }}\int^{k}_{-k}  e^{-\frac{u^2}{2}}\]
Which is the standard normal distribution's p.d.f., the desired result.

\subsection*{3.30 (ii)}

Probabilities that $X$ lies in between $-k$ and $k$ for 

\[ k = 1: 0.68268949213708585\]
\[ k = 2: 0.95449973610364158\]
\[ k = 3: 0.99730020393673979\]
\[ k = 4: 0.99993665751633376\]
\[ k = 5: 0.99999942669685615\]
\[ k = 6: 0.9999999980268246\]

\subsection*{3.31}
Using the beta distribution, we have that this probability is given by 
\[\frac{\Gamma(6)}{\Gamma(3)\Gamma(3)} \int^{1/3}_{0} x^2 - 2x^3 + x^4 dx \approx .2099\]


\subsection*{3.32}
Letting $y = xb$, we have that
\[ \int^{}_{} \frac{b^ax^{a-1}e^{-xb}dx}{\Gamma(a)} = 
\int^{}_{} \frac{y^{a-1}e^{-y}}{\Gamma(a)}dy = 
\frac{1}{\Gamma(a)}\int^{}_{} y^{a-1}e^{-y}dy= 
\frac{\Gamma(a)}{\Gamma(a)} = 1 \]
which is the desired result.





\end{document}
