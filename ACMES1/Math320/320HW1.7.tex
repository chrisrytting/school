\documentclass[letterpaper,12pt]{article}

\usepackage{threeparttable}
\usepackage{geometry}
\geometry{letterpaper,tmargin=1in,bmargin=1in,lmargin=1.25in,rmargin=1.25in}
\usepackage[format=hang,font=normalsize,labelfont=bf]{caption}
\usepackage{amsmath}
\usepackage{mathrsfs}
\usepackage{multirow}
\usepackage{array}
\usepackage{delarray}
\usepackage{listings}
\usepackage{amssymb}
\usepackage{amsthm}
\usepackage{lscape}
\usepackage{natbib}
\usepackage{setspace}
\usepackage{float,color}
\usepackage[pdftex]{graphicx}
\usepackage{pdfsync}
\usepackage{verbatim}
\usepackage{placeins}
\usepackage{geometry}
\usepackage{pdflscape}
\synctex=1
\usepackage{hyperref}
\hypersetup{colorlinks,linkcolor=red,urlcolor=blue,citecolor=red}
\usepackage{bm}


\theoremstyle{definition}
\newtheorem{theorem}{Theorem}
\newtheorem{acknowledgement}[theorem]{Acknowledgement}
\newtheorem{algorithm}[theorem]{Algorithm}
\newtheorem{axiom}[theorem]{Axiom}
\newtheorem{case}[theorem]{Case}
\newtheorem{claim}[theorem]{Claim}
\newtheorem{conclusion}[theorem]{Conclusion}
\newtheorem{condition}[theorem]{Condition}
\newtheorem{conjecture}[theorem]{Conjecture}
\newtheorem{corollary}[theorem]{Corollary}
\newtheorem{criterion}[theorem]{Criterion}
\newtheorem{definition}{Definition} % Number definitions on their own
\newtheorem{derivation}{Derivation} % Number derivations on their own
\newtheorem{example}[theorem]{Example}
\newtheorem{exercise}[theorem]{Exercise}
\newtheorem{lemma}[theorem]{Lemma}
\newtheorem{notation}[theorem]{Notation}
\newtheorem{problem}[theorem]{Problem}
\newtheorem{proposition}{Proposition} % Number propositions on their own
\newtheorem{remark}[theorem]{Remark}
\newtheorem{solution}[theorem]{Solution}
\newtheorem{summary}[theorem]{Summary}
\bibliographystyle{aer}
\newcommand\ve{\varepsilon}
\renewcommand\theenumi{\roman{enumi}}
\newcommand\norm[1]{\left\lVert#1\right\rVert}

\begin{document}

\title{Homework 1.7 Math 320}
\author{Chris Rytting}
\maketitle

\subsection*{1.37}
For $n=1$, it is obvious that there are $1! = 1$ permutations.
Assume that it is true that for $n-1$ there are $(n-1)!$ permutations such that for 
\[(k_1,k_2,k_3,\cdots,k_{n-1}) \text{ there are $(n-1)!$ permutations}.\]
Now, let us add an $n$th element. We know that for 
\[(k_n, k_1,k_2,k_3,\cdots,k_{n-1}) \text{ there are $(n-1)!$ permutations, and that for}\]
\[(k_1,k_n, k_2,k_3,\cdots,k_{n-1}) \text{ there are $(n-1)!$ permutations}.\]
Proceeding inductively, there are $n$ spots into which we could insert $k_n$, each resulting in a set with $(n-1)!$, permutations, leading to the conclusion that there are $n(n-1)!=n!$ permutations in a set with $n$ elements.

\subsection*{1.38 (i)}
\[ 6! \]

\subsection*{1.38 (ii)}
\[ 5!2! \]

\subsection*{1.38 (iii)}
\[ 4!3! \]

\subsection*{1.38 (iv)}
\[ 3!3!2! \]

\subsection*{1.39}
\[C(13,2)C(4,2)C(4,2)C(11,1)4 = 78\cdot6\cdot6\cdot11\cdot4 = 123,552\]

\subsection*{1.40}
\[\text{Total number of ways to win \$100: }C(1,1)C(5,3)C(54,2) = 1 \cdot 10 \cdot 1431 = 14,310\]
\[\text{Total possible combinations: }C(35,1)C(59,5) = 35\cdot5,006,386 = 175,223,510\]
\[\text{Probability of winning \$100: }14,310/175,223,510 = .000081667\]










\subsection*{1.41 (i)}
Note that
\begin{align*}
    (1+x)^n = \sum^{n}_{k=0} \binom nk x^k
\end{align*}
Differentiating both sides with respect to $x$, we have
\begin{align*}
    n(1+x)^{n-1} = \sum^{n}_{k=0} \binom nk kx^{k-1}
\end{align*}
Letting $x=1$, we have
\begin{align*}
    n2^{n-1} = \sum^{n}_{k=0} \binom nk k
    n2^{n-1} = \sum^{n}_{k=1} \binom nk k + 0
\end{align*}

\subsection*{1.41 (ii)}
Note that
\begin{align*}
    (1+x)^n = \sum^{n}_{k=0} \binom nk x^k
\end{align*}
Differentiating both sides with respect to $x$, we have
\begin{align*}
    n(1+x)^{n-1} = \sum^{n}_{k=0} \binom nk kx^{k-1}\\
    xn(1+x)^{n-1} = \sum^{n}_{k=0} \binom nk kx^{k}
\end{align*}
Differentiating both sides again (with respect to $x$), we have
\begin{align*}
    n(1+x)^{n-1} + xn(n-1)(1+x)^{n-2}  = \sum^{n}_{k=0} \binom nk k^2x^{k-1}
\end{align*}
Letting $x=1$, we have
\begin{align*}
    n(2)^{n-1} + n(n-1)(2)^{n-2}  = \sum^{n}_{k=0} \binom nk k^2\\
    n2^{n-2}(1 + n)  = \sum^{n}_{k=0} \binom nk k^2\\
    n2^{n-2}(1 + n)  = \sum^{n}_{k=1} \binom nk k^2 + 0
\end{align*}

\subsection*{1.42}

\begin{align*}
    (1+x)^n(1+x)^m = \sum^{n}_{k=0} \binom nk x^k \cdot \sum^{m}_{j=0} \binom mj x^j\\
    (1+x)^{n+m} = \sum^{n}_{k=0} \sum^{m}_{j=0} \binom nk \binom mj x^j x^k
\end{align*}
Now, let $r = k + j$, implying $j = r - k$. We also know that the monomials $x_1, x_2,\dots,x_n$ are linearly independent, implying that
\begin{align*}
    \implies \sum^{m+n}_{r=0} \sum^{r}_{k=0} \binom nk \binom m{r-k} x^r = \sum^{n+m}_{r=0} x^r \\
    \implies \sum^{r}_{k=0} \binom nk \binom m{r-k} = \binom {m+n}r
\end{align*}










\end{document}
