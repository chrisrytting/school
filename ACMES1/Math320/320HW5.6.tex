\documentclass[letterpaper,12pt]{article}

\usepackage{threeparttable}
\usepackage{geometry}
\geometry{letterpaper,tmargin=1in,bmargin=1in,lmargin=1.25in,rmargin=1.25in}
\usepackage[format=hang,font=normalsize,labelfont=bf]{caption}
\usepackage{amsmath}
\usepackage{mathrsfs}
\usepackage{multirow}
\usepackage{array}
\usepackage{delarray}
\usepackage{listings}
\usepackage{amssymb}
\usepackage{amsthm}
\usepackage{lscape}
\usepackage{natbib}
\usepackage{setspace}
\usepackage{float,color}
\usepackage[pdftex]{graphicx}
\usepackage{pdfsync}
\usepackage{verbatim}
\usepackage{placeins}
\usepackage{geometry}
\usepackage{pdflscape}
\synctex=1
\usepackage{hyperref}
\hypersetup{colorlinks,linkcolor=red,urlcolor=blue,citecolor=red}
\usepackage{bm}


\theoremstyle{definition}
\newtheorem{theorem}{Theorem}
\newtheorem{acknowledgement}[theorem]{Acknowledgement}
\newtheorem{algorithm}[theorem]{Algorithm}
\newtheorem{axiom}[theorem]{Axiom}
\newtheorem{case}[theorem]{Case}
\newtheorem{claim}[theorem]{Claim}
\newtheorem{conclusion}[theorem]{Conclusion}
\newtheorem{condition}[theorem]{Condition}
\newtheorem{conjecture}[theorem]{Conjecture}
\newtheorem{corollary}[theorem]{Corollary}
\newtheorem{criterion}[theorem]{Criterion}
\newtheorem{definition}{Definition} % Number definitions on their own
\newtheorem{derivation}{Derivation} % Number derivations on their own
\newtheorem{example}[theorem]{Example}
\newtheorem{exercise}[theorem]{Exercise}
\newtheorem{lemma}[theorem]{Lemma}
\newtheorem{notation}[theorem]{Notation}
\newtheorem{problem}[theorem]{Problem}
\newtheorem{proposition}{Proposition} % Number propositions on their own
\newtheorem{remark}[theorem]{Remark}
\newtheorem{solution}[theorem]{Solution}
\newtheorem{summary}[theorem]{Summary}
\bibliographystyle{aer}
\newcommand\ve{\varepsilon}
\renewcommand\theenumi{\roman{enumi}}
\newcommand\norm[1]{\left\lVert#1\right\rVert}

\begin{document}

\title{Math 320 Homework 5.6}
\author{Chris Rytting}
\maketitle

\subsection*{5.24}
\begin{align*}
    w_0 &= \int^{x_2}_{x_0} \frac{(x-x_1)(x-x_2)}{(x_0 - x_1)(x_0 - x_2)}dx 
    \\&= \frac{1}{(x_0 - x_1)(x_0 - x_2)}\int^{x_2}_{x_0}(x-x_1)(x-x_2) dx 
    \\&= \frac{1}{(x_0 - x_1)(x_0 - x_2)}\int^{x_2}_{x_0}x^2 - x_1 x - x_2 x + x_1 x_2 dx 
    \\&= \frac{1}{(x_0 - x_1)(x_0 - x_2)}\left[x^2 - x_1 x - x_2 x + x_1 x_2 \right]^{x_2}_{x_0}
    \\&= \frac{1}{(x_0 - x_1)(x_0 - x_2)}\left[\frac{x_2}{3}^3-  \frac{x_1x_2^2}{2} - \frac{x_2^3}{2} + x_1 x_2^2 - \frac{x_0^3}{3} + \frac{x_1 x_0^2}{2} + \frac{x_2 x_0^2}{2} - x_1x_2x_0\right]
    \\&= \frac{1}{h^2}\left[\frac{2x_2}{6}^3 - \frac{3x_1x_2^2}{6} - \frac{3x_2^3}{6} + \frac{6x_1 x_2^2}{6} - \frac{2x_0^3}{6} + \frac{3x_1 x_0^2}{6} + \frac{3x_2 x_0^2}{6} - \frac{6x_1x_2x_0}{6}\right]
    \\&= \frac{1}{h^2}\left[-\frac{x_2}{6}^3 - \frac{3x_1x_2^2}{6}  + \frac{6x_1 x_2^2}{6} - \frac{2x_0^3}{6} + \frac{3x_1 x_0^2}{6} + \frac{3x_2 x_0^2}{6} - \frac{6x_1x_2x_0}{6}\right]
    \\&= \frac{1}{h^2}\left[\frac{-x_2^3 + 3x_1x_2^2 - 2x_0^3 + 3x_1 x_0^2 + 3x_2 x_0^2 - 6x_1x_2x_0}{6}\right]
    \\&= \frac{1}{h^2}\left[\frac{-x_2^2(x_2 - x_1 - 2x_1) - x_0^2 (2x_0 - 3x_1) + x_2x_0(4x_0 - 6x_1)}{6}\right]
    \\&= \frac{1}{h^2}\left[\frac{-x_2^2(h - 2x_1) - x_0^2 (2x_0 - 2x_1 -x_1) + x_2x_0(3x_0 - 3x_1 - 3x_1)}{6}\right]
    \\&= \frac{1}{h^2}\left[\frac{-x_2^2(h - 2x_1) - x_0^2 (-2h -x_1) + x_2x_0(-3h - 3x_1)}{6}\right]
    \\&= \frac{1}{6}\frac{1}{h^2}\left[-x_2^2(h - 2x_1) - x_0^2 (-2h -x_1) + x_2x_0(-3h - 3x_1)\right]
    \\&= \frac{1}{6}\left[\frac{-x_2^2 + x_0^2 2 - x_2x_0 }{h} + \frac{2x_2^2x_1 + x_0^2x_1 - 3x_1x_2x_0}{h^2}\right]
    \\&= \frac{1}{6}\left[\frac{(x_0 - x_2)(2x_0 + x_2) }{h} + \frac{2x_2^2x_1 + x_0x_1(x_0 - x_2 - 2x_2}{h^2} + \right]
    \end{align*}
    \begin{align*}
    \\&= \frac{1}{6}\left[-2(2x_0 + x_2) + \frac{2x_2^2x_1  + x_0x_1(x_0 - x_2 - 2x_2)}{h^2}  \right]
    \\&= \frac{x_2 - x_0}{6}
    \\&= \frac{h}{3}
\end{align*}



\subsection*{5.26}
For the case where k = 0, we have
\[ \int_a^b 1 ~ dx = b - a  \]
Which implies
\begin{align*}
    \frac{b-a}{3n} \big( 1 + 4 + 2 + \dots + 2 + 4 + 1 \big)
    &= \frac{b-a}{3n} \big( (1 + 4 + 1) + (1 + 4 + 1) + \dots + (1 + 4 + 1) \big)\\
    &= \frac{b-a}{3n}\cdot (3n)\\
    &= b-a
\end{align*}
There are $\frac{n}{2}$ instances of $4$ and summation is equal to $\frac{n}{2} \cdot (6) = 3n$.\\
\\
For the case where k = 1, we have:
\[ \int_a^b x ~ dx = \frac{1}{2} (b^2 - a^2) = \frac{1}{2} (b-a)(b+a) \]
which yields
\begin{align*}
    \frac{b-a}{3n} &\big( x_0 + 4x_1 + 2x_2 + \dots + 2x_{n-2} + 4x_{n-1} + x_n \big)\\
    &= \frac{b-a}{3n} \big( x_0 + 4x_1 + 2x_2 + \dots + 2x_{n-2} + 4x_{n-1} + x_n \big)\\ 
    &= \frac{b-a}{3n} \cdot (b+a)\big( 1 + 4 + 2 + \dots + 2 + 4 + 1 \big)\\ 
    &= \frac{b-a}{3n}\cdot (b+a)\big( (1 + 4 + 1) + (1 + 4 + 1) + \dots + (1 + 4 + 1) \big)\\
    &= \frac{b-a}{3n}\cdot (b+a)(3n)\\
    &= b^2 + a^2
\end{align*}
implying  \[x_0 + x_n = x_1 + x_{n-1} = x_2 + x_{n-2} = \dots = b + a\]
For the case where k = 2, we have:
\[ \int_a^b x^2 ~ dx = \frac{1}{3} (b^3 - a^3) \]
which yields
\begin{align*}
    \frac{b-a}{3n} &\big( x_0^2 + 4x_1^2 + 2x_2^2 + \dots + 2x_{n-2}^2 + 4x_{n-1}^2 + x_n^2 \big)\\
    &= \frac{b-a}{3n} \big( x_0^2 + x_n^2 + 4x_1^2 + 4x_{n-1}^2 + 2x_2^2 + 2x_{n-2}^2 + \dots\big)\\ 
    &= \frac{b-a}{3n} \big( x_0^2 + x_n^2 + 4( x_1^2 + x_{n-1}^2) + 2 (x_2^2 + x_{n-2}^2) + \dots\big)\\ 
    &= \frac{b-a}{3n} \cdot \frac{1}{3(b-a)}(b^3 - a^3)\big( 1 + 4 + 2 + \dots + 2 + 4 + 1 \big)\\  
    &= \frac{b-a}{3n} \cdot \frac{1}{3(b-a)}(b^3 - a^3)\big( (1 + 4 + 1) + (1 + 4 + 1) + \dots + (1 + 4 + 1) \big)\\
    &= \frac{b-a}{3n} \cdot \frac{1}{3(b-a)}(b^3 - a^3)(3n)\\
    &= \frac{1}{3}(b^3 - a^3)\\
\end{align*}
By the integral MVT, \[\frac{1}{b-a}\int_a^b f(x)  dx\]
For the case where k = 3, we have
\[ \int_a^b x^3 ~ dx = \frac{1}{4} (b^4 - a^4) \]
yielding 
\begin{align*}
    \frac{b-a}{3n} &\big( x_0^2 + 4x_1^3 + 2x_2^3 + \dots + 2x_{n-2}^3 + 4x_{n-1}^3 + x_n^3 \big)\\
    &= \frac{b-a}{3n} \big( x_0^3 + x_n^3 + 4x_1^3 + 4x_{n-1}^3 + 2x_2^3 + 2x_{n-2}^3 + \dots\big)\\ 
    &= \frac{b-a}{3n} \big( x_0^3 + x_n^3 + 4( x_1^3 + x_{n-1}^3) + 2 (x_2^3 + x_{n-2}^3) + \dots\big)\\ 
    &= \frac{b-a}{3n} \cdot \frac{1}{4(b-a)}(b^4 - a^4)\big( 1 + 4 + 2 + \dots + 2 + 4 + 1 \big)\\  
    &= \frac{b-a}{3n} \cdot \frac{1}{4(b-a)}(b^4 - a^4)\big( (1 + 4 + 1) + (1 + 4 + 1) + \dots + (1 + 4 + 1) \big)\\
    &= \frac{b-a}{3n} \cdot \frac{1}{4(b-a)}(b^4 - a^4)(3n)\\
    &= \frac{1}{4} (b^4 - a^4)\\
\end{align*}
By the integral MVT, \[\frac{1}{b-a}\int_a^b f(x) ~ dx\]

\subsection*{5.27}
We know that:
\begin{align*}
    t &= kh^a 
\end{align*}
Logarithimically scaling, and evaluating the equation at two different values of $t_1,t_2$, we have
\begin{align*}
    \log(t_1)& = \log(k) + a \log(h_1)\\
    \log(t_2)& = \log(k) + a \log(h_2)
\end{align*}
yielding
\[\frac{\log(t_1) -\log(t_2)}{\log(h_1) - log(h_2)} = a \]
By values in table $5.2$, yields 
\[\frac{\log(2.7*10^{-5}) -\log(1.7*10^{-6})}{\log(1024) - \log(4096)} \approx 2.00003 \]
as for Simpson's method
\[\frac{\log(1.3162e-9) -\log(5.1585e-12)}{\log(1024) - \log(4096)} \approx 3.99997 \]
yielding the desired result.





\end{document}
