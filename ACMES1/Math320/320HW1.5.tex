\documentclass[letterpaper,12pt]{article}

\usepackage{threeparttable}
\usepackage{geometry}
\geometry{letterpaper,tmargin=1in,bmargin=1in,lmargin=1.25in,rmargin=1.25in}
\usepackage[format=hang,font=normalsize,labelfont=bf]{caption}
\usepackage{amsmath}
\usepackage{multirow}
\usepackage{array}
\usepackage{delarray}
\usepackage{listings}
\usepackage{amssymb}
\usepackage{amsthm}
\usepackage{lscape}
\usepackage{natbib}
\usepackage{setspace}
\usepackage{float,color}
\usepackage[pdftex]{graphicx}
\usepackage{pdfsync}
\usepackage{verbatim}
\usepackage{placeins}
\usepackage{geometry}
\usepackage{pdflscape}
\synctex=1
\usepackage{hyperref}
\hypersetup{colorlinks,linkcolor=red,urlcolor=blue,citecolor=red}
\usepackage{bm}


\theoremstyle{definition}
\newtheorem{theorem}{Theorem}
\newtheorem{acknowledgement}[theorem]{Acknowledgement}
\newtheorem{algorithm}[theorem]{Algorithm}
\newtheorem{axiom}[theorem]{Axiom}
\newtheorem{case}[theorem]{Case}
\newtheorem{claim}[theorem]{Claim}
\newtheorem{conclusion}[theorem]{Conclusion}
\newtheorem{condition}[theorem]{Condition}
\newtheorem{conjecture}[theorem]{Conjecture}
\newtheorem{corollary}[theorem]{Corollary}
\newtheorem{criterion}[theorem]{Criterion}
\newtheorem{definition}{Definition} % Number definitions on their own
\newtheorem{derivation}{Derivation} % Number derivations on their own
\newtheorem{example}[theorem]{Example}
\newtheorem{exercise}[theorem]{Exercise}
\newtheorem{lemma}[theorem]{Lemma}
\newtheorem{notation}[theorem]{Notation}
\newtheorem{problem}[theorem]{Problem}
\newtheorem{proposition}{Proposition} % Number propositions on their own
\newtheorem{remark}[theorem]{Remark}
\newtheorem{solution}[theorem]{Solution}
\newtheorem{summary}[theorem]{Summary}
\bibliographystyle{aer}
\newcommand\ve{\varepsilon}
\renewcommand\theenumi{\roman{enumi}}
\newcommand\norm[1]{\left\lVert#1\right\rVert}

\begin{document}

\title{Homework 1.5}
\author{Chris Rytting}
\maketitle

\subsection*{1.26}

\begin{align*}
\sum^{n}_{k=5} (k-5)^2 &= \sum^{n-5}_{j = 0} j^2
\\&= \sum^{n-5}_{j = 1} j^2
\\&= \frac{(2n-9)(n-4)(n-5)}{6}
\end{align*}

\subsection*{1.27}

\begin{align*}
    \sum^{n}_{k = 4}& \sum^{k-8}_{j = -3} (k-4) \quad \text{Now if we let $i = k-4$, we have}
\\&=\sum^{n}_{k=0} \sum^{i-4}_{j = -3} i
\\&=\sum^{n}_{k=0} i^2
\\&= \frac{(2n+1)(n+1)n}{6}
\end{align*}


\subsection*{1.28}

\begin{align*}
    &\sum^{n-3}_{j = -3} \sum^{n+3}_{k = j+3} (k-3) \quad \text{Now if we let $l = k-3$, we have}
    \\&=\sum^{n-3}_{j = -3} \sum^{n}_{l = j} l
   \\&= \sum^{n-3}_{j = -3} (\sum^{n}_{l = 0} l - \sum^{j-1}_{l = 0} l)
    \\&=\sum^{n-3}_{j = -3} \frac{(n)(n+1)}{2}- \frac{(j-1)(j)}{2})
    \\&=\sum^{n-3}_{j = -3} \frac{(n)(n+1)}{2}- \sum^{n-3}_{j = -3}  \frac{(j-1)(j)}{2})
    \\&=\frac{(n+1)^2n}{2}- \frac{1}{2}(\sum^{n-3}_{j = -3}j^2 - \sum^{n-3}_{j = -3}j )
    \\&=\frac{(n+1)^2n}{2}- \frac{1}{2}(\sum^{n-3}_{j = 0}j^2- \sum^{0}_{j = -3}j^2 - (\sum^{n-3}_{j = 0}j - \sum^{0}_{j = -3}j))
    \\&=\frac{(n+1)^2n}{2}- \frac{1}{2}( \frac{(n-3)(n-2)(2n-5)}{6} - 14 - \frac{(n-3)(n-2)}{2} - 6))
    \\&=\frac{1}{2}((n+1)^2n- ( \frac{(n-3)(n-2)(2n-5)}{6} - \frac{(n-3)(n-2)}{2} - 20))
\end{align*}


\subsection*{1.29}
\[ \sum^{n}_{k=1} k^3 = (\frac{n(n+1)}{2})^2\]
If $f(k) = k^4$, then we know by the FTFC that 
\[( \Delta f)(k) = (k+1)^4 - k^4 = k^3 + 6k^2 + 4k +1\]
\begin{align*}
\sum^{b-1}_{k =a} (4k^3 + 6k^2 + 4k + 1) &= \sum^{b-1}_{k = a} (\Delta f) 
\\&=  f(b) -f(a) 
\\&= b^4- a^4
\\&=  \sum^{b-1}_{k=a} (4k^3 + 6k^2 + 4k + 1)
\\&= 4 \sum^{b-1}_{k = a} k^3 + 6 \sum^{b-1}_{k =a} k^2 + 4 \sum^{b-1}_{k=a} k + \sum^{b-1}_{k=a} 1
\\&= 4 \sum^{b-1}_{k = a} k^3 + 6 \sum^{b-1}_{k =a} k^2 + 4 \sum^{b-1}_{k=a} k + (b-a)
\\\implies b^4 - a^4 - (b-a) &= b^4 - b - (a^4 -a) 
\\&= b(b^3-1) - a(a^3 - 1)
\\&= 4 \sum^{b-1}_{k = a} k^3+6 \sum^{b-1}_{k = a} k^2+4 \sum^{b-1}_{k = a} k
\end{align*}

Now let $a=1, b = n+1$, then we have that
\begin{align*}
(n+1)( ( n+1)^3 -1) &= 4 \sum^{n}_{k=1} k^3 + (2n+1)(n+1)n + 2n(n+1)
\\\implies \sum^{n}_{k=1} k^3&= \frac{(n(n+1))}{2}^2
\end{align*}


\subsection*{1.30}
Note that if $f(i) = \frac{-1}{i}$, then 
\[ (\Delta f)(i) = \frac{-1}{i + 1} + \frac{1}{i}\]
\[ \implies \frac{-i}{i(i+1)} + \frac{i+ 1}{i(i+1)} = \frac{1}{i(i+1)}\]
so we have that
\[ \sum^{n}_{i=1} (\Delta f) (i) = \frac{-1}{n+1} - \frac{-1}{1} = 1 - \frac{1}{n+1}\]

\subsection*{1.31 (i)}
\begin{align*}
\sum^{n}_{k=0} \sum^{n}_{j=k} &= \sum^{n}_{k=0} ( \sum^{n}_{j = 0} j - \sum^{k-1}_{j = 0} j)
\\&= \sum^{n}_{k = 0} (\frac{n(n+1)}{2} - \frac{(k-1)k}{2})
\\&= (n+1)^2n - \sum^{n}_{k=0} k^2 + \sum^{n}_{k=0} k
\\&= \frac{1}{2}[(n+1)^2n - \frac{n(n+1)(2n+1}{6} + \frac{n(n+1)}{2}]
\\&= \frac{1}{3}n^3 + \frac{3}{2}n^2 + \frac{2}{3}n
\\
\text{Going the other way we have}
\sum^{n}_{j=0 } \sum^{j}_{k = 0} j &= \sum^{n}_{j=1} j^2 + \sum^{n}_{j=1}  
\\&= \frac{n(n+1)(2n+1}{6} + \frac{n(n+1)}{2}
\\&= \frac{1}{3}n^3 + \frac{3}{2}n^2 + \frac{2}{3}n
\end{align*}
Which is equal to the first part so we have the desired result.

\subsection*{1.31 (ii)}
\begin{align*}
\sum^{n}_{k=0}  \sum^{k}_{j = 0} j &= \sum^{n}_{k = 0} (\frac{(k+1)k}{2}
\\&= \sum^{n}_{k=1} k^2 + \sum^{n}_{k=1} k
\\&= \frac{1}{2}[\frac{(2n+1)(n+1)n}{6} + \frac{n(n+1)}{2}]
\\&= \frac{n^3 + 3n^2 + 2n}{6} 
\\
\text{Going the other way, we have}\\
\sum^{k}_{j=0} \sum^{n}_{k=0}j &= \sum^{n}_{j = 0} \sum^{n}_{k = j} j
\\&= \sum^{n}_{j=0} (n-j+1)j
\\&= \sum^{n}_{j = 0} jn - j^2 + j
\\&= \sum^{n}_{j = 0} jn - \sum^{n}_{j=0} + \sum^{n}_{j = 0} 
\\&= \frac{n^2(n+1)}{2} - \frac{(2n+1)(n+1)n}{6} + \frac{n(n+1)}{2}
\\&= \frac{n^3 + 3n^2 + 2n}{6}
\end{align*}
Which is equal to the first part so we have the desired result.

\subsection*{1.32}
\begin{align*}
\sum^{N}_{t= 0 } \beta^t &= \frac{\beta^{N+1}-1}{\beta-1}
\\\implies \sum^{N}_{t=0} t\beta^{t-1} &= \frac{(\beta-1)[(N+1)\beta^N] - (\beta^{N+1} -1}{(\beta-1)^2}\\
\\
&\text{Note that $ \vert \beta \vert < 1$, so taking the limit as $N \rightarrow \infty$ yields}
\\
\\ &\text{lim}_N\rightarrow\infty \frac{(\beta-1)[(N+1)\beta^N] - (\beta^{N+1} -1}{(\beta-1)^2}\\
\\&= \frac{1}{1-\beta^2}
\text{Now, note that}
\\&= \sum^{\infty}_{t=0} t\beta^{t-1} 
\\&= \sum^{\infty}_{t=0} 
\\&= \frac{1}{(1-\beta^2)}
\\
\implies \frac{1}{\beta} \sum^{\infty}_{t = 0} t \beta^t &= \frac{1}{(1-\beta^2)}\\
\implies \sum^{\infty}_{t = 0} t \beta^t &= \frac{\beta}{(1-\beta^2)}
\end{align*}







\end{document}
