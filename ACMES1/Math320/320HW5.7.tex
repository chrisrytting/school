\documentclass[letterpaper,12pt]{article}

\usepackage{threeparttable}
\usepackage{geometry}
\geometry{letterpaper,tmargin=1in,bmargin=1in,lmargin=1.25in,rmargin=1.25in}
\usepackage[format=hang,font=normalsize,labelfont=bf]{caption}
\usepackage{amsmath}
\usepackage{mathrsfs}
\usepackage{multirow}
\usepackage{array}
\usepackage{delarray}
\usepackage{listings}
\usepackage{amssymb}
\usepackage{amsthm}
\usepackage{lscape}
\usepackage{natbib}
\usepackage{setspace}
\usepackage{float,color}
\usepackage[pdftex]{graphicx}
\usepackage{pdfsync}
\usepackage{verbatim}
\usepackage{placeins}
\usepackage{geometry}
\usepackage{pdflscape}
\synctex=1
\usepackage{hyperref}
\hypersetup{colorlinks,linkcolor=red,urlcolor=blue,citecolor=red}
\usepackage{bm}


\theoremstyle{definition}
\newtheorem{theorem}{Theorem}
\newtheorem{acknowledgement}[theorem]{Acknowledgement}
\newtheorem{algorithm}[theorem]{Algorithm}
\newtheorem{axiom}[theorem]{Axiom}
\newtheorem{case}[theorem]{Case}
\newtheorem{claim}[theorem]{Claim}
\newtheorem{conclusion}[theorem]{Conclusion}
\newtheorem{condition}[theorem]{Condition}
\newtheorem{conjecture}[theorem]{Conjecture}
\newtheorem{corollary}[theorem]{Corollary}
\newtheorem{criterion}[theorem]{Criterion}
\newtheorem{definition}{Definition} % Number definitions on their own
\newtheorem{derivation}{Derivation} % Number derivations on their own
\newtheorem{example}[theorem]{Example}
\newtheorem{exercise}[theorem]{Exercise}
\newtheorem{lemma}[theorem]{Lemma}
\newtheorem{notation}[theorem]{Notation}
\newtheorem{problem}[theorem]{Problem}
\newtheorem{proposition}{Proposition} % Number propositions on their own
\newtheorem{remark}[theorem]{Remark}
\newtheorem{solution}[theorem]{Solution}
\newtheorem{summary}[theorem]{Summary}
\bibliographystyle{aer}
\newcommand\ve{\varepsilon}
\renewcommand\theenumi{\roman{enumi}}
\newcommand\norm[1]{\left\lVert#1\right\rVert}

\begin{document}

\title{Math 320 Homework 5.7}
\author{Chris Rytting}
\maketitle

\subsection*{5.31}

Suppose to the contrary that we have $n$ nodes

\[\int_{-1}^1 p_{2n}(x) = \sum_{i=0}^{n-1} f(x_i)w_i\]


Yielding the sytstem of equations

\begin{align*}
   2 &= \int_{-1}^1 1 = \sum_i^{n-1}w_i\\
   0 &= \int_{-1}^1 x = \sum_i^{n-1} x_i^1w_i\\
   \frac{2}{3} &= \int_{-1}^1 x^2 = \sum_i^{n-1} x_i^2w_i\\
   0 &= \int_{-1}^1 x^3= \sum_i^{n-1} x_i^3w_i\\
   &\vdots\\
   \frac{2}{2n+1} &= \int_{-1}^1 x^2 = \sum_i^{n-1} x_i^{2n}w_i\\
\end{align*}

If the system has a solution, it will yield a gaussian quadrature for $n$ nodes on $\mathbb{R}[x]_{2n}$\\



However, there is one more unknown than there are equations. Therefore, we cannot find a solution.

\subsection*{5.32}
The third-degree Taylor series approximation around 0 is given by
\begin{align*}
p(x) &\approx f(a) + \frac{f'(a)}{1!}(x-a)^1 + \frac{f''(a)}{2!}(x-a)^2 +\frac{f'''(a)}{3!}(x-a)^3\\
&=\sin(3) + \cos(3)x - \frac{sin(3)}{2}x^2 - \frac{\cos(3)}{6}x^3
\end{align*}
Now, by Example 5.7.2, we compute the integral and using a Taylor series approximation
\begin{align*}
\int_{-1}^1 & \frac{-\cos(3)}{6}x^3 - \frac{\sin(3)}{2}x^2 + \cos(3)x +sin(3)dx\\
 &= \frac{-cos(3)}{6}(\frac{-1}{\sqrt{3}})^3 - \frac{sin(3)}{2}(\frac{-1}{\sqrt{3}})^2 + \cos(3)(\frac{-1}{\sqrt{3}}) +\sin(3)\\
       &-\frac{cos(3)}{6}(\frac{1}{\sqrt{3}})^3 - \frac{sin(3)}{2}(\frac{1}{\sqrt{3}})^2 + \cos(3)(\frac{1}{\sqrt{3}}) +\sin(3)\\
       &= 0.2352
\end{align*}
Now, computing the integral of $\sin(x+3)$, we can compare it to the previous computation
\[\int_{-1}^1 sin(x+3)dx = -cos(x+3)|_{-1}^1 = -cos(4) - (-cos(3)) = 0.2374\] 
Which are nearly the same.

\subsection*{5.33}
Let $y_i = g(x_i) =  a(1-x_i) + b(x_i)$. Then we have that if $\{y_i\}_{i = 0}^n \subset [a,b]$ are the roots of the $n+1$st Legendre polynomial, then for all $q(x) = \mathbb{R}[x]_{2n+1}$ we have 
\[ \int^{b}_{a} q(x) dx = \sum^{n}_{i=0} q(y_i)w_i\]
where
\[w_i = \int^{b}_{a} L_{i,n}(x) dx, \quad i = 0,1,2,\cdots,n \]
are the integrals of the Lagrange basis polynomials. Since we have a map $g:[-1,1] \to [a,b]$ and a map $g^{-1}: [a,b]\to [-1,1] $, we can apply the proof of Theorem 5.7.4 without loss of generality by mapping back and forth between these intervals. 

\subsection*{5.34}

\begin{lstlisting}

import numpy as np

def f(x):
    return np.abs(x)

def z(x):
    return np.cos(x)


def quadrature(f,n):
    a = np.linspace(-1,1,n+1)
    roots,weights = np.polynomial.legendre.leggauss(n+1)
    function_vals = f(roots)
    return np.sum(function_vals*weights)
print quadrature(z,4)
print np.sin(1)*2.

1.68294197041
1.68294196962
(Very close to one another)



print abs(x) yields the following values with for n = 10,20,30,...,100:
for i in xrange(10,110,10):
    print quadrature(f,i)

0.987523109474
0.996438310884
0.99834153543
0.999044665942
0.999379701294
0.999565044201
0.999678211086
0.999752337668
0.999803515872
0.999840326218

\end{lstlisting}
One's approximation is better since it is smooth and therefore more conducive to using polynomials.


\end{document}
