\documentclass[letterpaper,12pt]{article}

\usepackage{threeparttable}
\usepackage{geometry}
\geometry{letterpaper,tmargin=1in,bmargin=1in,lmargin=1.25in,rmargin=1.25in}
\usepackage[format=hang,font=normalsize,labelfont=bf]{caption}
\usepackage{amsmath}
\usepackage{centernot}
\usepackage{multirow}
\usepackage{array}
\usepackage{delarray}
\usepackage{listings}
\usepackage{amssymb}
\usepackage{amsthm}
\usepackage{lscape}
\usepackage{natbib}
\usepackage{setspace}
\usepackage{float,color}
\usepackage[pdftex]{graphicx}
\usepackage{pdfsync}
\usepackage{verbatim}
\usepackage{placeins}
\usepackage{geometry}
\usepackage{pdflscape}
\synctex=1
\usepackage{hyperref}
\hypersetup{colorlinks,linkcolor=red,urlcolor=blue,citecolor=red}
\usepackage{bm}


\theoremstyle{definition}
\newtheorem{theorem}{Theorem}
\newtheorem{acknowledgement}[theorem]{Acknowledgement}
\newtheorem{algorithm}[theorem]{Algorithm}
\newtheorem{axiom}[theorem]{Axiom}
\newtheorem{case}[theorem]{Case}
\newtheorem{claim}[theorem]{Claim}
\newtheorem{conclusion}[theorem]{Conclusion}
\newtheorem{condition}[theorem]{Condition}
\newtheorem{conjecture}[theorem]{Conjecture}
\newtheorem{corollary}[theorem]{Corollary}
\newtheorem{criterion}[theorem]{Criterion}
\newtheorem{definition}{Definition} % Number definitions on their own
\newtheorem{derivation}{Derivation} % Number derivations on their own
\newtheorem{example}[theorem]{Example}
\newtheorem{exercise}[theorem]{Exercise}
\newtheorem{lemma}[theorem]{Lemma}
\newtheorem{notation}[theorem]{Notation}
\newtheorem{problem}[theorem]{Problem}
\newtheorem{proposition}{Proposition} % Number propositions on their own
\newtheorem{remark}[theorem]{Remark}
\newtheorem{solution}[theorem]{Solution}
\newtheorem{summary}[theorem]{Summary}
\bibliographystyle{aer}
\newcommand\ve{\varepsilon}
\renewcommand\theenumi{\roman{enumi}}
\newcommand\norm[1]{\left\lVert#1\right\rVert}

\begin{document}
\title{Homework 1.4}
\author{Chris Rytting}
\maketitle

\subsection*{1.19 (i)}
Given an integer $a = \sum^{n-1}_{k = 0} a_k 10^k = a_0 10^0 +a_1 10^1 +\cdots +a_{n-1} 10^{n-1}$
Note that
\begin{align*}
    a &= \sum^{n-1}_{k = 0} a_k 10^k \\&= a_0 10^0 +a_1 10^1 +\cdots +a_{n-1}10^{n-1}
    \\&= (9a_1 +99a_2 + \cdots +(10^{n-1}-1)a_{n-1}) + (a_0 + a_1 +\cdots + a_{n-1})
    \\&= 3(3a_1 +33a_2 + \cdots +((10^{n-1}-1)/3)a_{n-1}) + (a_0 + a_1 +\cdots + a_{n-1})
\end{align*}
Note that $a$ is divisible by 3 if and only if the second term $(a_0 + a_1 +\cdots + a_{n-1})$ is divisible by 3, since it should be apparent that every term in the first expression is divisible by 3.

\subsection*{1.19(ii)}
Given an integer $a = \sum^{n-1}_{k = 0} a_k 10^k = a_0 10^0 +a_1 10^1 +\cdots +a_{n-1} 10^{n-1}$
Note that
\begin{align*}
    a &= \sum^{n-1}_{k = 0} a_k 10^k \\&= a_0 10^0 +a_1 10^1 +\cdots +a_{n-1}10^{n-1}
    \\&= (9a_1 +99a_2 + \cdots +(10^{n-1}-1)a_{n-1}) + (a_0 + a_1 +\cdots + a_{n-1})
    \\&= 9(a_1 +11a_2 + \cdots +((10^{n-1}-1)/9)a_{n-1}) + (a_0 + a_1 +\cdots + a_{n-1})
\end{align*}
Note that $a$ is divisible by 9 if and only if the second term $(a_0 + a_1 +\cdots + a_{n-1})$ is divisible by 9, since it should be apparent that every term in the first expression is divisible by 9 

\subsection*{1.19(iii)}
Given an integer $a = \sum^{n-1}_{k = 0} a_k 10^k = a_0 10^0 +a_1 10^1 +\cdots +a_{n-1} 10^{n-1}$
Note that
\begin{align*}
    a &= \sum^{n-1}_{k = 0} a_k 10^k \\&= a_0 10^0 +a_1 10^1 +\cdots +a_{n-1}10^{n-1}
    \\&= (11a_1 +99a_2 + 1001a_3 \cdots +(10^{n-1}-1)a_{n-1}) + (a_0 - a_1 + a_2 - a_3\cdots + (-1)^{n-1}a_{n-1})\\
    &(\text{With the coefficients of $a_i$ being given by $10^{i} +(- 1)^{i-1}$})
    \\&= 11(a_1 +9a_2 + 91a_3 \cdots +((10^{n-1}-1)/11)a_{n-1}) + (a_0 - a_1 + a_2 - a_3\cdots + (-1)^{n-1}a_{n-1})\\
\end{align*}
Note that $a$ is divisible by 11 if and only if the second expression $(a_0 + a_1 +\cdots + a_{n-1})$ is divisible by 11, since every term in the first expression is divisible by 11.

\subsection*{1.20}
Since $a \equiv b (\text{mod }  c)$, we know that $a-b = cn$ for some $n \in \mathbb{Z} $. Furthermore, since $d \vert c$, we know that $c = dm$ for some $m \in \mathbb{Z} $. Note that 
\[ a-b = cn = dmn \implies a-b = dk \implies d \vert (a-b) \implies a \equiv b(\text{mod } d)\]
where $k = mn \implies k \in \mathbb{Z}$.

\subsection*{1.21}
\begin{align*}
    34^{34}&=-2^{34}
    \\&=2^{34}
    \\&=(2^6)^52^4
    \\&=(4^5)2^4
    \\&=4^6
    \\&=4^24^24^2
    \\&=4 \cdot 4 \cdot 4
    \\&=4 \cdot 4
    \\&=4
\end{align*}
\subsection*{1.22 (i)}
\begin{align*}
((((((14^2)^2)^2)^2)^2)^2)^2 &= (((((69^2)^2)^2)^2)^2)^2
\\&= ((((62^2)^2)^2)^2)^2
\\&= (((34^2)^2)^2)^2
\\&= ((13^2)^2)^2
\\&= (42^2)^2
\\&= 113^2
\\&= 69
\end{align*}
\subsection*{1.22 (ii)}
\begin{align*}
    18^{254} &= (18^{15})^{16}18^{14}
    \\&=(76^{16})103
    \\&=(76^8)^2103
    \\&=47^2103
    \\&=50 \cdot 103
    \\&=70
\end{align*}
\subsection*{1.22 (iii)}
\begin{align*}
    25^{640} &= (25^{10})^{64}
    \\&=(76^8)^8
    \\&=47^8
    \\&=76
\end{align*}
\subsection*{1.23}
We need $\text{gcd}(x,c) = 1$.
\\
For the sufficient condition, we have Proposition 1.4.9 which states that for any integers $a,b,c$ if gcd$(a,b) = 1$, then $a\vert bc$ implies that $a \vert c$. 
\\
For the necessary condition, note that for $m,n \in \mathbb{Z}$ we need
\[ ax \equiv bx (\text{mod }c) \implies ax - bx = cn \implies x(a-b) = cn \]
\[\implies a \equiv b (\text{mod }c) \implies a - b = cm \]
Now let $a-b = \alpha \quad x = \xi d \text{ where } d= \text{gcd}(x,c)$. If we let $d > 1$, contrary to the condition we've required, then
\[ \xi d \alpha = cn \centernot\implies \alpha = cm\]
For the only way to guarantee that $c\vert \alpha $ (since we have no control over $m$ or $n$), is to require $\alpha$ to be the only term such that $c \vert \alpha$. If $d \vert cn \text{ where }d \neq 1$, then $\alpha$ can take on any number of values to fulfill the first condition that will not imply the second.


\subsection*{1.25}
We know $a \geq b \quad a,b \in \mathbb{Z}$. 
In the case that $a=b$, the program will terminate in one step, which is obviously $O(af(a)) \quad a > b$, since $f(a)$ is nondecreasing. 
In the case that $a>b$, we know that, since $a,b \in \mathbb{Z}$, $a \geq b + 1$. In this case, the EA will terminate in, at most, $b + 1$ steps (since $b > 0$ and at each stage we have $r_k > r_{k+1}, \geq 0$, so $|b| > r_0 > r_1 > \cdots \leq 0$). So even in the worst case scenario, the algorithm will be $O(b+1f(a)) \leq O(af(a))$ since $f(a)$ is non-decreasing and therefore at least a constant function.






\end{document}
