
\documentclass[letterpaper,12pt]{article}

\usepackage{threeparttable}
\usepackage{geometry}
\geometry{letterpaper,tmargin=1in,bmargin=1in,lmargin=1.25in,rmargin=1.25in}
\usepackage[format=hang,font=normalsize,labelfont=bf]{caption}
\usepackage{amsmath}
\usepackage{mathrsfs}
\usepackage{multirow}
\usepackage{array}
\usepackage{delarray}
\usepackage{listings}
\usepackage{amssymb}
\usepackage{amsthm}
\usepackage{lscape}
\usepackage{natbib}
\usepackage{setspace}
\usepackage{float,color}
\usepackage[pdftex]{graphicx}
\usepackage{pdfsync}
\usepackage{verbatim}
\usepackage{placeins}
\usepackage{geometry}
\usepackage{pdflscape}
\synctex=1
\usepackage{hyperref}
\hypersetup{colorlinks,linkcolor=red,urlcolor=blue,citecolor=red}
\usepackage{bm}


\theoremstyle{definition}
\newtheorem{theorem}{Theorem}
\newtheorem{acknowledgement}[theorem]{Acknowledgement}
\newtheorem{algorithm}[theorem]{Algorithm}
\newtheorem{axiom}[theorem]{Axiom}
\newtheorem{case}[theorem]{Case}
\newtheorem{claim}[theorem]{Claim}
\newtheorem{conclusion}[theorem]{Conclusion}
\newtheorem{condition}[theorem]{Condition}
\newtheorem{conjecture}[theorem]{Conjecture}
\newtheorem{corollary}[theorem]{Corollary}
\newtheorem{criterion}[theorem]{Criterion}
\newtheorem{definition}{Definition} % Number definitions on their own
\newtheorem{derivation}{Derivation} % Number derivations on their own
\newtheorem{example}[theorem]{Example}
\newtheorem{exercise}[theorem]{Exercise}
\newtheorem{lemma}[theorem]{Lemma}
\newtheorem{notation}[theorem]{Notation}
\newtheorem{problem}[theorem]{Problem}
\newtheorem{proposition}{Proposition} % Number propositions on their own
\newtheorem{remark}[theorem]{Remark}
\newtheorem{solution}[theorem]{Solution}
\newtheorem{summary}[theorem]{Summary}
\bibliographystyle{aer}
\newcommand\ve{\varepsilon}
\renewcommand\theenumi{\roman{enumi}}
\newcommand\norm[1]{\left\lVert#1\right\rVert}

\begin{document}

\title{Math 320 Homework 4.8}
\author{Chris Rytting}
\maketitle

\subsection*{4.37}
We know that 
\[ \psi ( x) = \varphi(2x) - \varphi(2x-1)\]
and that
\[ \psi_{jk} (x) = \psi(2^jx - k) \]
yielding
\[ \psi_{jk} (x) = \varphi(2(2^jx - k)) - \varphi(2(2^jx - k)-1)\]
\[\implies  \psi_{jk} (x) = \varphi(2^{j+1}x - 2k) - \varphi(2^{j+1}x - 2k-1)\]
which is the desired result.


\subsection*{4.38}


We have \[f(x) = -2 \phi (4x) + 4 \phi(4x-1) + 2 \phi(4x-2) - 3 \phi(4x-3) \in V_2\]
which can be expressed as follows:
\begin{align*}
    f(x) & = -(\phi(2x) + \psi(2x)) + 2(\phi(2x) - \psi(2x)) \\
    &+ (\phi(2x-1) + \psi(2x-1) ) - \frac{3}{2}(\phi(2x-1) - \psi(2x-1) ) \\
    & = \psi(2x) - 3 \psi(2x) - \frac{1}{2} \phi(2x-1) + \frac{5}{2} \psi(2x-1) \\
    & = \frac{1}{2} (\phi(x) + \psi) - 3 \psi(2x) - \frac{1}{4}(\phi(x) - \psi(x) ) + \frac{5}{2}-\psi(2x-1) \\
    & = \frac{1}{4}\phi(x) + \frac{3}{4}\psi(x) + (-3\psi(2x)+ \frac{5}{2} \psi(2x-1) )
\end{align*}

Now, we have that
\begin{align*}
    &\frac{1}{4} \phi(x) \in V_0\\
    &\frac{3}{4}\psi(x) \in W_0\\
    &-3\psi(2x) + \frac{5}{2} \psi (2x-1) ) \in W_1
\end{align*}

\subsection*{4.39}
We have that
\[f(x) = 2 \varphi (4x) + 3 \varphi(4x-1) +  \varphi(4x-2) - 3 \varphi(4x-3) \in V_2 \]
which can be expressed as follows:
\begin{align*}
    f(x) & = (\varphi(2x) + \psi(2x)) + \frac{3}{2}(\varphi(2x) - \psi(2x)) + \\
    & \quad \quad \frac{1}{2}(\varphi(2x-1) + \psi(2x-1) ) - \frac{3}{2}(\varphi(2x-1) - \psi(2x-1) ) \\
    & = \frac{5}{2} \varphi(2x) - \frac{1}{2} \psi(2x) - \varphi(2x-1) + 2 \psi(2x-1) \\
    & = \frac{1}{2} (\varphi(x) + \psi(x)) - \frac{1}{2} \psi(2x) - (\varphi(x) - \psi(x) ) + 2 \psi(2x-1) \\
    & = \frac{1}{4} \varphi(x) + \frac{9}{4}\psi(x) + (-\frac{1}{2}\psi(2x)+ 2 \psi(2x-1) )
\end{align*}
Now, we have that
\[
    \left(\frac{1}{4} \varphi(x)\right) \in V_0 \quad
    \left(\frac{9}{4}\psi(x)\right) \in W_0 \quad
    \left(-\frac{1}{2}\psi(2x)+ 2 \psi(2x-1) \right) \in W_1
\]

\subsection*{4.40}

\begin{lstlisting}
#########################
import numpy as np
from matplotlib import pyplot as plt

def sample(f, n):
    sampling = []
    for k in xrange(2**n+1):
        sampling.append(f(k/2.**n))
    return sampling

def mother_approximation(f, n):
    def psi(x):
        if x < .5 and x >= 0:
            val = 1.
            return val
        elif x < 1 and x >= .5:
            val = -1.
            return val
        else:
            val = 0.
            return val

    def daughter(x):
        value = 0.
        for k, sam in enumerate(sample(f, n)):
            value += sam * psi(2**n * x - k)
        return value

    return wn

def father_approximation(f, n):
    def phi(x):
        if x < 1 and x >= 0:
            val = 1.
            return val
        else:
            val = 0.
            return val

    def son(x):
        value = 0.
        for k, sam in enumerate(sample(f, n)):
            value += sam * phi(2**n * x - k)
        return value

    return fn


def test():
    f = lambda x: (np.sin(2.*np.pi*x - 5.))/(np.sqrt(np.abs(x - (np.pi/20.))))
    
    x = np.linspace(0,1,50)
    for l in xrange(1,11):
        print l
        fn = fn_approx(f, l)
        fn_x = []
        for val in x:
            fn_x.append(fn(val))

        plt.subplot(121)
        plt.plot(x, f(x))
        
        plt.subplot(122)
        plt.plot(x, fn_x)
        plt.show()

#Part 2

def coeff(f, n, sampling):
    c_k = []
    b_k = []
    for k in xrange(len(sampling)):
        c_k.append((f((2.*k)/(2.**n+1)) + f((2.*k + 1.)/(2.**n+1)))/2.)
        b_k.append((f((2.*k)/(2.**n+1)) - f((2.*k + 1.)/(2.**n+1)))/2.)

def wavelet_decomp(f, n, l):
    l_list = np.linspace(l, (n-1), (n-l)+1)
    wn_list = []
    for ll in l_list:
        wn_list.append(wn_approx(f, ll)
    fl = fn_approx(f, l)

    def fn(x):
        value = fl(x)
        for wn in wn_list:
            value += wn(x)
        return value

    plt.plot(fl(x))
    plt.plot(wn_list[0](x))
    plt.show()

    return fn
        
def test2():
    f = lambda x: (np.sin(2.*np.pi*x - 5.))/(np.sqrt(np.abs(x - (np.pi/20.))))
    n = 10
    ls = [0,1,2,3,4,5,6,7,8,9]
    x = np.linspace(0,1,50)
    for l in ls:
        wavelet_decomp(f, n, l)
     
\end{lstlisting}
        This is different because it has incorporated the mother function as well. 

\end{document}
