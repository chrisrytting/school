\documentclass[letterpaper,12pt]{article}

\usepackage{threeparttable}
\usepackage{geometry}
\geometry{letterpaper,tmargin=1in,bmargin=1in,lmargin=1.25in,rmargin=1.25in}
\usepackage[format=hang,font=normalsize,labelfont=bf]{caption}
\usepackage{amsmath}
\usepackage{mathrsfs}
\usepackage{multirow}
\usepackage{array}
\usepackage{delarray}
\usepackage{listings}
\usepackage{amssymb}
\usepackage{amsthm}
\usepackage{lscape}
\usepackage{natbib}
\usepackage{setspace}
\usepackage{float,color}
\usepackage[pdftex]{graphicx}
\usepackage{pdfsync}
\usepackage{verbatim}
\usepackage{placeins}
\usepackage{geometry}
\usepackage{pdflscape}
\synctex=1
\usepackage{hyperref}
\hypersetup{colorlinks,linkcolor=red,urlcolor=blue,citecolor=red}
\usepackage{bm}


\theoremstyle{definition}
\newtheorem{theorem}{Theorem}
\newtheorem{acknowledgement}[theorem]{Acknowledgement}
\newtheorem{algorithm}[theorem]{Algorithm}
\newtheorem{axiom}[theorem]{Axiom}
\newtheorem{case}[theorem]{Case}
\newtheorem{claim}[theorem]{Claim}
\newtheorem{conclusion}[theorem]{Conclusion}
\newtheorem{condition}[theorem]{Condition}
\newtheorem{conjecture}[theorem]{Conjecture}
\newtheorem{corollary}[theorem]{Corollary}
\newtheorem{criterion}[theorem]{Criterion}
\newtheorem{definition}{Definition} % Number definitions on their own
\newtheorem{derivation}{Derivation} % Number derivations on their own
\newtheorem{example}[theorem]{Example}
\newtheorem{exercise}[theorem]{Exercise}
\newtheorem{lemma}[theorem]{Lemma}
\newtheorem{notation}[theorem]{Notation}
\newtheorem{problem}[theorem]{Problem}
\newtheorem{proposition}{Proposition} % Number propositions on their own
\newtheorem{remark}[theorem]{Remark}
\newtheorem{solution}[theorem]{Solution}
\newtheorem{summary}[theorem]{Summary}
\bibliographystyle{aer}
\newcommand\ve{\varepsilon}
\renewcommand\theenumi{\roman{enumi}}
\newcommand\norm[1]{\left\lVert#1\right\rVert}

\begin{document}

\title{Math 344 Homework 5.3}
\author{Chris Rytting}
\maketitle

\subsection*{5.16}


Suppose that $\{x_i\}_{i=0}^\infty$ and  $\{y_i\}_{i=0}^\infty$ be Cauchy. Then 
\[d(x,x_n)<\frac{\epsilon}{2}\]
and 
\[d(y,y_n) < \frac{\epsilon}{2}\] 
for some $N$. Note, by triangle inequality we have
\[|d(x_k,y_k) - d(x_j, y_j)| \leq |d(x_k, x_j)+d(y_k,y_j)| < \frac{\epsilon}{2} + \frac{\epsilon}{2} = \epsilon\]
which is the desired result.
 
\subsection*{5.17}
Suppose $\{x_i\}_{i=0}^\infty$ has a closer point $x \in X$. By Remark 5.2.33, there is a subsequence convergent to $x$.\\\\
Furthermore, by Proposition 5.3.8, the whole Cauchy sequence converges to the unique cluster point referred to in Prop 5.2.30.

\subsection*{5.18 (i)}
$x^3$ is uniformly continuous on the interval $(0,1)$, but not on the latter.
\subsection*{5.18 (ii)}
$\frac{\sin(x)}{x} $ is uniformly and Lipshitz continuous on both intervals.
\subsection*{5.18 (iii)}
    $x \log(x)$ is uniformly continuous on the inverval $(0,1)$, but not on the latter.
Let $\epsilon >0$ be arbitrary, and let $\delta = \frac{\epsilon}{3}$. Then when $|x-y| < \delta$, we have
\[ |x^3 - x^3 | = |x-y|(x^2+xy +y^2)<3|x-y| < \frac{3\epsilon}{3} = \epsilon \implies \text{ uniformly continuous}\]
On the other hand, for the other interval, note that for any fixed difference in $|x-y|$, 
there exists an $x$ such that,
\[|x^2 - y^2| = x^2|1-\left( \frac{y}{x} \right)^2| \not < \epsilon \]


\subsection*{5.19}

Let $f: X \to Y$ be a bounded linear transformation. Note that for all unit vectors, 
\[\sup \|f \mathbf{x}\| = \|f\| < M, \quad M \in \mathbb{N}\]
Let $\delta = \frac{\epsilon}{\|f\|} $. Then
\[\|f \mathbf{x} \| \leq \|f\| \|\mathbf{x}\|\]
 
\[\|f \mathbf{x} - f\mathbf{y} \| = \|f(\mathbf{x}- \mathbf{y})\| \leq \|f\|\|x-y\| < \epsilon\]

which is the desired result.

\subsection*{5.20}


By way of contradiction, assume that there exists a Cauchy sequence in $Z$ that does not converge to something in $Z$. Then, since $Y$ is dense in $Z$, every $z \in Z$ is either a limit point of $Y$ or found in $Y$. \\\\
Now, let $\{z_i\}_{i=0}^\infty$ be this sequence that does not converge in $Z$. We can see that $\exists j,n$ such that
\[
d(z_j,z_n) = d(d(y_{j,i},y_{j,k},d(y_{n,m},y_{n,p})<\epsilon
\]
By Exercise 5.16, distance is a real number, but is arbitrarily small in this case, and a limit point of $Y$, therefore we know that it is in $Z$, and the sequence converges to it in $Z$.\\\\
We have a contradiction and the desired result.



\subsection*{5.21 (i)}


        Suppose $f(B)$ is not bounded, then there exists an 
        \[\mathbf{x} \in B0 \text{ such that } f(x) > M \forall M \in \mathbb{N}\] Then, given $d(x,y) < \delta$, 
        \[|f(x)-f(y)| \not < \epsilon \forall \epsilon > 0\] 
        Then by contrapositive, we have that if $f$ is uniformly continuous, $f(B)$ is bounded.
\subsection*{5.21 (ii)}
    $f(\mathbf{x}) = \sqrt{\mathbf{x}}$ is continuous, but unbounded.

\subsection*{5.22}
    
    
No. Let $X$ be $\mathbb{R}$, and the metric on $X$ be 
\[d(x,y) = \frac{1}{2}, x \neq y, 0 \text{ otherwise}\] 
Consider as a counterexample, the funciton $f(x) = x$ which is uniformly continious on this metric yet unbounded.

\end{document}
