\documentclass[letterpaper,12pt]{article}

\usepackage{threeparttable}
\usepackage{geometry}
\geometry{letterpaper,tmargin=1in,bmargin=1in,lmargin=1.25in,rmargin=1.25in}
\usepackage[format=hang,font=normalsize,labelfont=bf]{caption}
\usepackage{amsmath}
\usepackage{mathrsfs}
\usepackage{multirow}
\usepackage{array}
\usepackage{delarray}
\usepackage{listings}
\usepackage{amssymb}
\usepackage{amsthm}
\usepackage{lscape}
\usepackage{natbib}
\usepackage{setspace}
\usepackage{float,color}
\usepackage[pdftex]{graphicx}
\usepackage{pdfsync}
\usepackage{verbatim}
\usepackage{placeins}
\usepackage{geometry}
\usepackage{pdflscape}
\synctex=1
\usepackage{hyperref}
\hypersetup{colorlinks,linkcolor=red,urlcolor=blue,citecolor=red}
\usepackage{bm}


\theoremstyle{definition}
\newtheorem{theorem}{Theorem}
\newtheorem{acknowledgement}[theorem]{Acknowledgement}
\newtheorem{algorithm}[theorem]{Algorithm}
\newtheorem{axiom}[theorem]{Axiom}
\newtheorem{case}[theorem]{Case}
\newtheorem{claim}[theorem]{Claim}
\newtheorem{conclusion}[theorem]{Conclusion}
\newtheorem{condition}[theorem]{Condition}
\newtheorem{conjecture}[theorem]{Conjecture}
\newtheorem{corollary}[theorem]{Corollary}
\newtheorem{criterion}[theorem]{Criterion}
\newtheorem{definition}{Definition} % Number definitions on their own
\newtheorem{derivation}{Derivation} % Number derivations on their own
\newtheorem{example}[theorem]{Example}
\newtheorem{exercise}[theorem]{Exercise}
\newtheorem{lemma}[theorem]{Lemma}
\newtheorem{notation}[theorem]{Notation}
\newtheorem{problem}[theorem]{Problem}
\newtheorem{proposition}{Proposition} % Number propositions on their own
\newtheorem{remark}[theorem]{Remark}
\newtheorem{solution}[theorem]{Solution}
\newtheorem{summary}[theorem]{Summary}
\bibliographystyle{aer}
\newcommand\ve{\varepsilon}
\renewcommand\theenumi{\roman{enumi}}
\newcommand\norm[1]{\left\lVert#1\right\rVert}

\begin{document}

\title{Math 320 Homework 1.8}
\author{Chris Rytting}
\maketitle

\subsection*{1.44 (i)}

\begin{align*}
\binom a0 = \frac{\Gamma (a+1)}{\Gamma (1) \Gamma(a + 1)}
\binom a0 = \frac{\Gamma (a+1)}{0! \Gamma(a + 1)}
\binom a0 = \frac{\Gamma (a+1)}{ \Gamma(a + 1)}
\end{align*}




\subsection*{1.44 (ii)}

\begin{align*}
\binom a{b+1} = \frac{\Gamma(a+1)}{\Gamma(a-b) \Gamma(b+2)} 
\end{align*}

Now, note that $\Gamma(n) = (n-1)\Gamma(n-1) \implies  \Gamma(a-b) = \frac{\Gamma(a-b+1)}{a-b} $, and we have that
\begin{align*}
\binom {a}{b+1} = \frac{\Gamma(a+1)}{\Gamma(a-b) \Gamma(b+2)} &=\frac{\Gamma(a+1) (a-b)}{\Gamma(a-b+1) \Gamma(b+1) (b+1)}
\\&= \binom a{b+1} \frac{a-b}{b+1}
\end{align*}

\subsection*{1.44 (iii)}

\begin{align*}
\binom {a-1}{b-1}+ \binom {a-1}b &= \frac{\Gamma(a)}{\Gamma(b) \Gamma(a-b+1)} + \frac{\Gamma(a)}{\Gamma(b+1) \Gamma(a-b)} 
\\&= \frac{\Gamma(a)b}{\Gamma(b+1) \Gamma(a-b+1)} + \frac{(a-b)\Gamma(a)}{\Gamma(b+1) \Gamma(a-b+1)} 
\\&= \frac{\Gamma(a)b + \Gamma(a)a - \Gamma(a)b}{\Gamma(b+1) \Gamma(a-b+1)} 
\\&= \frac{ \Gamma(a)a }{\Gamma(b+1) \Gamma(a-b+1)} 
\\&= \frac{ \Gamma(a+1) }{\Gamma(b+1) \Gamma(a-b+1)} 
\\&= \binom ab 
\end{align*}

\subsection*{1.44 (iv)}
\begin{align*}
    \binom ak &= \frac{\Gamma(a+1)}{\Gamma(a-k+1) \Gamma(k+1)} 
    \\ &= \frac{\Gamma(a+1)}{\Gamma(a-k+1) k!} 
    \\ &= \frac{a\Gamma(a)}{(a-k)\Gamma(a-k) k!} 
    \\ &= \frac{a(a-1)\Gamma(a-1)}{(a-k)(a-k-1)\Gamma(a-k-1) k!} 
\end{align*}
Proceeding inductively, since $k \in \mathbb{N} $, we know that eventually we will have 

\begin{align*}
    \\ &= \frac{a(a-1)(a-2)\dots(a-k+1)(a-k)(a-k-1) \Gamma(a-k-1)}{(a-k)(a-k-1)\Gamma(a-k-1) k!} 
\end{align*}
It should be clear that the terms following and including $(a-k)$ in the numerator and denominator products will cancel out, leading to 

\begin{align*}
    \\ &= \frac{a(a-1)(a-2)\dots(a-k+1)}{k!} 
\end{align*}
Which is the desired result.

\subsection*{1.45 (i)}

\begin{align*}
    (\int^{\infty}_{-\infty}  e^{x^2/2})^2 dx  &= \int^{\infty}_{-\infty} \int^{\infty}_{-\infty} e^{-(x^2 + y^2)/2} dx dy
    \\&= \int^{2 \pi}_{0} \int^{\infty}_{0} e^{-r^2/2} dr d\theta  \text{ where } r^2 = x^2 + y^2
    \\&= \int^{2 \pi}_{0} \int^{\infty}_{0} e^{-r^2/2} dr d\theta
\end{align*}
Now let $u = -\frac{r^2}{2}$, and we have
\begin{align*}
    \\&= \int^{2 \pi}_{0} \int^{\infty}_{0} e^{-u} du d\theta
    \\&= \int^{2 \pi}_{0} \int^{\infty}_{0} e^{-u} du d\theta
    \\&= \int^{2 \pi}_{0} \Big( -e^{-u} \Big|^\infty_0 \Big) d\theta 
    \\&= \int^{2 \pi}_{0} 1 d\theta 
    \\&= \theta \Big|^{2 \pi}_{0}
    \\&= 2 \pi
\end{align*}
\subsection*{1.45 (ii)}

\begin{align*}
\Gamma(x) &= \frac{1}{2{x-1}} \int^{\infty}_{0}e^{-\frac{u^2}{2}} u^{2x-1}du
\\&= \frac{1}{2^{x-1}} \int^{\infty}_{0}e^{-\frac{u^2}{2}} u^{2x-1}du
\\&= \frac{1}{2^{x-1}} \int^{\infty}_{0}e^{-\frac{u^2}{2}} u^{2x-1}du
\end{align*}
Now, letting $t = \frac{u^2}{2}$, we have
\begin{align*}
    \\&= \frac{1}{{2^{x-1}}} \int^{\infty}_{0}e^{-t} u^{2x-2}dt
\\&= \frac{1}{2^{x-1}} \int^{\infty}_{0}e^{-t} u^{2x-2}dt
\\&= \frac{1}{2^{x-1}} \int^{\infty}_{0}e^{-t} 2^{x-1} (\frac{u^2}{2})^{x-1}dt
\\&= \frac{2^{x-1}}{2^{x-1}} \int^{\infty}_{0}e^{-t}  t^{x-1}dt
\end{align*}
Which is equal to the definition of $\Gamma(x)$, yielding the desired result.

\subsection*{1.45 (iii)}
By $1.45 (ii)$
\begin{align*}
    \Gamma(\frac{1}{2}) &=\frac{1}{a^{2x-1}} \int^{\infty}_{0}e^{-\frac{u^2}{2}} u^{2x-1}du
    \\&=\frac{1}{2^-1/2} \int^{\infty}_{0}e^{-\frac{u^2}{2}} u^{0}du
    \\&= \sqrt{2} \int^{\infty}_{0}e^{-\frac{u^2}{2}} du
    \\&= \sqrt{2} \int^{\infty}_{0}e^{-\frac{u^2}{2}} du
    \\&= \sqrt{2} \int^{\infty}_{0}\int^{\infty}_{0}e^{-\frac{x^2 +y^2}{2}} du
    \\&= \sqrt{2} \int^{\pi/2}_{0}\int^{\infty}_{0}e^{-\frac{r^2}{2}} r dr d\theta
    \\&= \sqrt{2} \int^{\pi/2}_{0}\int^{\infty}_{0}e^{-\frac{r^2}{2}} r dr d\theta
\end{align*}
If we let $u = \frac{r^2}{2}$
\begin{align*}
    \\&= \sqrt{2} \int^{\pi/2}_{0}\int^{\infty}_{0}e^{-u} r du d\theta
    \\&= \sqrt{2} \int^{\pi/2}_{0}\int^{\infty}_{0}e^{-u}  du d\theta
    \\&= \sqrt{2} \int^{\pi/2}_{0} 1 d\theta
    \\&= \sqrt{2} \int^{\pi/2}_{0} 1 d\theta
    \\&= \sqrt{2} \frac{\pi}{2} 1 d\theta
\end{align*}

\subsection*{1.45 (iv)}
\begin{align*}
    \int^{\infty}_{0} e^{-xt^2} dt &= \Big( \int^{\pi}{0} e^{-xt^2} dt\Big)^2
    \\&= \Big( \int^{\infty}_{0} e^{-xt^2} dt\Big)^2
    \\&= \int^{\infty}_{0} \int^{\infty}{0} e^{-x(m^2 + n^2)} dm dn
\end{align*}
    Let $r^2 = m^2 + n^2$, we have  
\begin{align*}
    \\&= \int^{\pi/2}_{0} \int^{\infty}{0} e^{-xr^2}r dr d\theta
    \\&= \int^{\pi/2}_{0} \int^{\infty}{0} e^{-xr^2}r dr d\theta
\end{align*}
Let $u = xr^2$, and we have that
\begin{align*}
    \\&= \int^{\pi/2}_{0} \int^{\infty}_{0} \frac{e^{-u}}{2x} du d\theta
    \\&=\frac{1}{\sqrt{2x}} \int^{\pi/2}_{0} \int^{\infty}_{0} e^{-u} du d\theta
    \\&=\frac{1}{\sqrt{2x}} \int^{\pi/2}_{0} 1 d\theta
    \\&=\frac{1}{\sqrt{2x}} \sqrt{\frac{\pi}{2}} 
    \\&=\frac{1}{2} \sqrt{\frac{\pi}{x}} 
\end{align*}

\subsection*{1.46}
The desired is equivalent to showing that
\begin{align*}
    &\text{lim}_{x\rightarrow \infty} \frac{\text{Beta} (x,y)}{\Gamma(y) x^{-y}} =1
    \\&= \text{lim}_{x\rightarrow \infty} \frac{\frac{\Gamma(x)\Gamma(y) }{\Gamma(x+y)}}{\frac{\Gamma(y)x^{-y}}{1}} 
    \\&= \text{lim}_{x\rightarrow \infty} \frac{\Gamma(x)\Gamma(y) }{\Gamma(y) \Gamma(x+y) x^{-y}} 
    \\&= \text{lim}_{x\rightarrow \infty} \frac{\Gamma(x) }{ \Gamma(x+y) x^{-y}} 
    \\&= \text{lim}_{x\rightarrow \infty} \frac{\Gamma(x) x^{y}}{ \Gamma(x+y) } 
    \\&= \frac{\int^{x}_{0}e^{-t} t^{x-1} dt x^y}{\int^{x}_{0}e^{-t} t^{x+y-1} dt} 
\end{align*}
Now, examining the denominator of this expression, we have
\begin{align*}
    \int^{x}_{0}e^{-t} t^{x+y-1} dt = \Big( -e^{-t}t^{x+y-1} \Big|^{x}_{0} \Big) + (x+y-1) \int^{x}_{0}e^{-t} t^{x-1 + y-1}
\end{align*}
Resulting in 
\begin{align*}
    \frac{x\cdot x \cdot \dots \cdot \int^{x}_{0}e^{-t} t^{x-1}dt}{(x-1+y)(x-1+y-1)\dots(x-1) \int^{x}_{0}d^{-t}t^{x-1}dt}
\end{align*}
Now, the right-most terms will cancel, and since we are taking the limit with respect to $x \rightarrow \infty$, we will need to differentiate with respect to $x$ alone after encountering $\frac{\infty}{\infty}$. We will need to do so at least $y$ times, meaning that all other $x$ terms go to zero and we will have, effectively, $\frac{x^y}{x^y} = 1$. 

\subsection*{1.47}
\begin{align*}
    \int^{\infty}_{0} e^{-xt} t^p dt
\end{align*}
Now, letting 
\[ u = xt \]
\[ du = x~ dt \]
\[ t = \frac{u}{x} \]
We have
\begin{align*}
    \int^{\infty}_{0} e^{-xt} t^p dt &=\int^{\infty}_{0} \frac{1}{x}e^{-u} (\frac{u}{x})^p du
    \frac{1}{x}\int^{\infty}_{0} e^{-u} (\frac{u}{x})^p du
    \\&=\frac{1}{x}\int^{\infty}_{0} e^{-u} \frac{u^p}{x^p} du
    \\&=\frac{1}{x^{1+p}}\int^{\infty}_{0} e^{-u} u^p du
\end{align*}
And since $\int^{\infty}_{0} e^{-u} u^p du = \Gamma(p+1) $, we have the desired result.

\subsection*{1.48}
We know that $\text{log} () $ strictly increasing on $[1, \infty)$ 
\[ \implies \text{log} (1) <\text{log} (2) < \dots<\text{log} (n-1) <\text{log} (n) \]
 \begin{align*}
     \sum^{n-1}_{k=1}  \text{log} (k)  &= \text{log} (1) + \text{log} (2) + \dots + \text{log} (n-1) 
     \\&= \text{log}(n-1)! 
     \\\sum^{n}_{k=1}  \text{log} (k)  &= \text{log} (1) + \text{log} (2) + \dots + \text{log} (n) 
     \\&= \text{log}(n)! 
 \end{align*}
 Also, we have that
 \begin{align*}
     \int^{n}_{1}\text{log} (k) &= k ( \text{log} (n-1))^n
     \\&= n \text{log} (k) - n + 1
 \end{align*}
 Where $\text{log} (n)$ is a continually increasing funciton, we have that the first summation to $n-1$ is equivalent to the Riemann sums evaluated at the left endpoints of each interval, and the second sum corresponds to the right endpoints, meaning that the first is a lower bound on the true integral, and the second is an upper bound
 \[ \implies \sum^{n-1}_{k=1} \text{log} (k) < \int^{n}_{1} \text{log} (x) dx < \sum^{n}_{k=1} \text{log} (k)\]
 \[ \implies \text{log} (n-1)! < n\text{log} (n) -n+1 <  \text{log} (n!)\]
 Now, notice that if we add $\text{log} (n) $ to each term
 \[ \implies \text{log} (n-1)! + \text{log} (n)  < n\text{log} (n) -n+1+ \text{log} (n) <  \text{log} (n!)+ \text{log} (n)\]
 which yields, combining the inequalities.
 \[ \implies n\text{log} (n) -n+1 <  \text{log} (n!) < n\text{log} (n) -n+1+ \text{log} (n) \]
Which yields the second inequality. As for the third, let us raise every term to the exponent.
\[ e^{n\text{log} (n) -n+1} < e^{ \text{log} (n!) }< e^{n\text{log} (n) -n+1+ \text{log} (n)} \]
\[ \frac{e^{\text{log} (n)^n}}{e^{ n-1}} < e^{ \text{log} (n!) } < \frac{e^{(n+1) \text{log} (n)}}{e^{n-1}} \]
\[ \frac{ (n)^n}{e^{ n-1}} < (n!) <  \frac{n^{(n+1)}  }{e^{n-1}} \]

\subsection*{1.49}
Consider
\begin{align*}
    \int^{1}_{-1} e^{x cosh(t)} dt 
\end{align*}
Let 
\[\alpha = x\]
\[f(t) = -cosh(t)\]
\[f'(t) = -sinh(t)\]
\[f''(t) = -cosh(t)\]
We can find $x_0$ by setting $f'(t)$ equal to $0$.

\[-sinh(x_0) = 0\]
Differentiating, we have
\[-sinh^{-1}(sinh(x_0)) = 0\]
\[ \implies x_0 = 0\]

We have that 
\begin{align*}
    \int^{1}_{-1} e^{x cosh(t)} dt = e^{\alpha f(x_0)} \sqrt{\frac{2 \pi}{ \alpha|f''(x_0) |}}
    &=e^{-1x} \sqrt{\frac{2 \pi}{ x |1 |}}
    \\&=e^{-x} \sqrt{\frac{2 \pi}{ x }}
\end{align*}

\end{document}
