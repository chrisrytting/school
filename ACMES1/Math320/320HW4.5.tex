\documentclass[letterpaper,12pt]{article}

\usepackage{threeparttable}
\usepackage{geometry}
\geometry{letterpaper,tmargin=1in,bmargin=1in,lmargin=1.25in,rmargin=1.25in}
\usepackage[format=hang,font=normalsize,labelfont=bf]{caption}
\usepackage{amsmath}
\usepackage{multirow}
\usepackage{array}
\usepackage{delarray}
\usepackage{amssymb}
\usepackage{amsthm}
\usepackage{lscape}
\usepackage{natbib}
\usepackage{setspace}
\usepackage{float,color}
\usepackage[pdftex]{graphicx}
\usepackage{mathrsfs}  
\usepackage{pdfsync}
\usepackage{verbatim}
\usepackage{placeins} \usepackage{geometry}
\usepackage{pdflscape}
\synctex=1
\usepackage{hyperref}
\hypersetup{colorlinks,linkcolor=red,urlcolor=blue,citecolor=red}
\usepackage{bm}
\usepackage{amssymb}


\theoremstyle{definition}
\newtheorem{theorem}{Theorem}
\newtheorem{acknowledgement}[theorem]{Acknowledgement}
\newtheorem{algorithm}[theorem]{Algorithm}
\newtheorem{axiom}[theorem]{Axiom}
\newtheorem{case}[theorem]{Case}
\newtheorem{claim}[theorem]{Claim}
\newtheorem{conclusion}[theorem]{Conclusion}
\newtheorem{condition}[theorem]{Condition}
\newtheorem{conjecture}[theorem]{Conjecture}
\newtheorem{corollary}[theorem]{Corollary}
\newtheorem{criterion}[theorem]{Criterion}
\newtheorem{definition}{Definition} % Number definitions on their own
\newtheorem{derivation}{Derivation} % Number derivations on their own
\newtheorem{example}[theorem]{Example}
\newtheorem*{exercise}{Exercise} % Number exercises on their own
\newtheorem{lemma}[theorem]{Lemma}
\newtheorem{notation}[theorem]{Notation}
\newtheorem{problem}[theorem]{Problem}
\newtheorem{proposition}{Proposition} % Number propositions on their own
\newtheorem{remark}[theorem]{Remark}
\newtheorem{solution}[theorem]{Solution}
\newtheorem{summary}[theorem]{Summary}
\bibliographystyle{aer}
\newcommand\ve{\varepsilon}
\renewcommand\theenumi{\roman{enumi}}

\title{Math 320 Homework 4.5}
\author{Chris Rytting}

\begin{document}
\maketitle
\subsection*{4.25}

We have that
\begin{align*}
(\textbf{f} \ast \textbf{g})_0  &=  1 \cdot 0 + 4 \cdot 0 + 3 \cdot 1 + 2 \cdot 0 = 3 \\
(\textbf{f} \ast \textbf{g})_2  &=  2 \cdot 0 + 1 \cdot 0 + 4 \cdot 1 + 3 \cdot 0 = 4 \\
(\textbf{f} \ast \textbf{g})_3  &=  3 \cdot 0 + 2 \cdot 0 + 1 \cdot 1 + 4 \cdot 0 = 1 \\
(\textbf{f} \ast \textbf{g})_4  &=  4 \cdot 0 + 3 \cdot 0 + 2 \cdot 1 + 1 \cdot 0 = 2 \\
\end{align*}
\begin{align*}
(\textbf{g} \ast \textbf{h})_0  &=  0 \cdot 1 + 0 \cdot i + 1 \cdot -1 + 0 \cdot -i = -1 \\
(\textbf{g} \ast \textbf{h})_2  &=  0 \cdot 1 + 0 \cdot i + 0 \cdot -1 + 1 \cdot -i = -i \\
(\textbf{g} \ast \textbf{h})_3  &=  1 \cdot 1 + 0 \cdot i + 0 \cdot -1 + 0 \cdot -i = 1 \\
(\textbf{g} \ast \textbf{h})_4  &=  0 \cdot 1 + 1 \cdot i + 0 \cdot -1 + 0 \cdot -i = i \\
\end{align*}
\begin{align*}
(\textbf{g} \ast \textbf{g})_0  &=  0 \cdot 0 + 0 \cdot 0 + 1 \cdot 1 + 0 \cdot 0 = 1 \\
(\textbf{g} \ast \textbf{g})_2  &=  0 \cdot 0 + 0 \cdot 0 + 0 \cdot 1 + 1 \cdot 0 = 0 \\
(\textbf{g} \ast \textbf{g})_3  &=  1 \cdot 0 + 0 \cdot 0 + 0 \cdot 1 + 0 \cdot 0 = 0 \\
(\textbf{g} \ast \textbf{g})_4  &=  0 \cdot 0 + 1 \cdot 0 + 0 \cdot 1 + 0 \cdot 0 = 0 \\
\end{align*}
\begin{align*}
(\textbf{h} \ast \textbf{k})_0  &=  1 \cdot  1 + -i \cdot -1 + -1 \cdot 1 +  i \cdot -1 = 0 \\
(\textbf{h} \ast \textbf{k})_2  &=  i \cdot  1 + 1  \cdot -1 + -i \cdot 1 + -1 \cdot -1 = 0 \\
(\textbf{h} \ast \textbf{k})_3  &=  -1 \cdot 1 + i  \cdot -1 +  1 \cdot 1 + -i \cdot -1 = 0 \\
(\textbf{h} \ast \textbf{k})_4  &=  -i \cdot 1 + -1 \cdot -1 +  i \cdot 1 +  1 \cdot -1 = 0 \\
\end{align*}
\begin{align*}
(\textbf{h} \ast \textbf{h})_0  &=  1 \cdot 1 + -i \cdot i + -1 \cdot -1 + i \cdot -i = 4 \\
(\textbf{h} \ast \textbf{h})_2  &=  i \cdot 1 + 1 \cdot i + -i \cdot -1 + -1 \cdot -i = 4i \\
(\textbf{h} \ast \textbf{h})_3  &=  -1 \cdot 1 + i \cdot i + 1 \cdot -1 + -i \cdot -i = -4 \\
(\textbf{h} \ast \textbf{h})_4  &=  -i \cdot 1 + -1 \cdot i + 1 \cdot -1 + 1 \cdot -i = -4i \\
\end{align*}
And thus we have that
\begin{align*}
(\textbf{f} \ast \textbf{g}) &= \begin{bmatrix} 3 & 4 & 1 & 2 \end{bmatrix}^T\\
(\textbf{g} \ast \textbf{h}) &= \begin{bmatrix} -1 & -i & 1 & i \end{bmatrix}^T\\
(\textbf{g} \ast \textbf{g}) &= \begin{bmatrix} 1 & 0 & 0 & 0 \end{bmatrix}^T\\
(\textbf{h} \ast \textbf{k}) &= \begin{bmatrix} 0 & 0 & 0 & 0 \end{bmatrix}^T\\
(\textbf{h} \ast \textbf{h}) &= \begin{bmatrix} 4 & 4i & -4 & -4i \end{bmatrix}^T
\end{align*}








\subsection*{4.26}


\begin{align*}
    F_4\textbf{f} \odot F_4\textbf{g} &= \begin{bmatrix} 2.5 & .5-.5i & -.5 & .5+.5i \end{bmatrix}^T \\
    F_4\textbf{g} \odot F_4\textbf{h} &= \begin{bmatrix} 0 & -1 & 0 & 0 \end{bmatrix}^T \\
    F_4\textbf{g} \odot F_4\textbf{g} &= \begin{bmatrix} \frac{1}{4} & \frac{1}{4} & \frac{1}{4} & \frac{1}{4} \end{bmatrix}^T \\
    F_4\textbf{h} \odot F_4\textbf{k} &= \begin{bmatrix} 0 & 0 & 0 & 0 \end{bmatrix}^T \\
    F_4\textbf{h} \odot F_4\textbf{h} &= \begin{bmatrix} 0 & 4 & 0 & 0 \end{bmatrix}^T 
\end{align*}
\begin{align*}
F_4^{-1} =
\begin{bmatrix}
    1 & 1 & 1 & 1 \\
    1 & i & -1 & -i \\
    1 & -1 & 1 & -1 \\
    1 & -i & -1 & i
\end{bmatrix}\\
\end{align*}
By multiplying this matrix by each hadamard product, we get the vectors from 4.25.

\subsection*{4.27}

Using zeroes as the coefficients of certain polynomials to get the polynomials to be $2^n$ long, we can use 
FFT to calculate the fourier transform with a temporal complexity of $O(n)$.
Splitting them then into evens and odds $\omega_n$.\\\\
We then have two Discrete fourier transforms that are $n/2$ long. Then we take the hadamard product of these polynomials.
Then we can apply the inverse of the FFT to this vector of coefficents with $O(n)$ temporal complexity. Total temporal compexity is given, then, by
\[T(n) \leq 4T(n) + cn\]
which by the master theorem is $O(n \log(n))$

\subsection*{4.28}


Let 
\begin{align*}
A = \begin{bmatrix}
    a_0 & a_{n-1} & \dots & a_2 &a_1 \\
    a_1 & a_{0} & \dots & a_3 &a_2 \\
    \vdots & \vdots & & \vdots & \vdots\\
    a_{n-2} & a_{n-3} & \dots & a_0 &a_{n-1} \\
    a_{n-1} & a_{n-2} & \dots & a_1 &a_0 \\
\end{bmatrix} \quad
\mathbf{g} = \begin{bmatrix} g_0  \\
                 g_1  \\
                 \vdots \\
                 g_{n-1}\\
\end{bmatrix} \quad
\mathbf{a} = \begin{bmatrix} a_0  \\
                 a_1  \\
                 \vdots \\
                 a_{n-1}\\
\end{bmatrix}
\end{align*}
Note, 
\[A \mathbf{g} = 
\begin{bmatrix}
    a_0 g_0 & a_{n-1}g_1 & \dots & a_2g_{n-1} &a_1g_n \\
    a_1 g_0 & a_{0}  g_1 & \dots & a_3 g_{n-1} &a_2 g_n \\
    \vdots & \vdots & & \vdots & \vdots\\
    a_{n-2}g_0 & a_{n-3}g_1 & \dots & a_0g_{n-1} &a_{n-1}g_n \\
    a_{n-1} g_0& a_{n-2}g_1 & \dots & a_1g_{n-1} &a_0 g_n\\
\end{bmatrix} \]
and, by definition of the convolution, we have \[\mathbf{f}*\mathbf{g} =\begin{bmatrix}
    a_0 g_0 & a_{n-1}g_1 & \dots & a_2g_{n-1} &a_1g_n \\
    a_1 g_0 & a_{0}  g_1 & \dots & a_3 g_{n-1} &a_2 g_n \\
    \vdots & \vdots & & \vdots & \vdots\\
    a_{n-2}g_0 & a_{n-3}g_1 & \dots & a_0g_{n-1} &a_{n-1}g_n \\
    a_{n-1} g_0& a_{n-2}g_1 & \dots & a_1g_{n-1} &a_0 g_n\\
\end{bmatrix}\]
Thus, they are identical.

\subsection*{4.29}

We have that
 \[ p(\lambda) = \lambda^4 + 8\lambda^3 + 20\lambda^2 + 16\lambda = 0\] yielding \[\lambda_1 = -4, \lambda_2 = -2 = \lambda_3, \lambda_4 = 0\]
with eigenvectors
\[v_1 = \begin{bmatrix}-1 \\
                 1 \\
                 -1\\
                 1\\
\end{bmatrix} \quad
v_2 = \begin{bmatrix}0 \\
                 -1 \\
                 0\\
                 1\\
\end{bmatrix} \quad
v_3 = \begin{bmatrix}-1 \\
                 0 \\
                 1\\
                 0\\
\end{bmatrix} \quad
v_4 = \begin{bmatrix}1 \\
                 1 \\
                 1\\
                 1\\
\end{bmatrix} \quad\]

\end{document}

