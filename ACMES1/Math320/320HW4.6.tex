\documentclass[letterpaper,12pt]{article}

\usepackage{threeparttable}
\usepackage{geometry}
\geometry{letterpaper,tmargin=1in,bmargin=1in,lmargin=1.25in,rmargin=1.25in}
\usepackage[format=hang,font=normalsize,labelfont=bf]{caption}
\usepackage{amsmath}
\usepackage{mathrsfs}
\usepackage{multirow}
\usepackage{array}
\usepackage{delarray}
\usepackage{listings}
\usepackage{amssymb}
\usepackage{amsthm}
\usepackage{lscape}
\usepackage{natbib}
\usepackage{setspace}
\usepackage{float,color}
\usepackage[pdftex]{graphicx}
\usepackage{pdfsync}
\usepackage{verbatim}
\usepackage{placeins}
\usepackage{geometry}
\usepackage{pdflscape}
\synctex=1
\usepackage{hyperref}
\hypersetup{colorlinks,linkcolor=red,urlcolor=blue,citecolor=red}
\usepackage{bm}


\theoremstyle{definition}
\newtheorem{theorem}{Theorem}
\newtheorem{acknowledgement}[theorem]{Acknowledgement}
\newtheorem{algorithm}[theorem]{Algorithm}
\newtheorem{axiom}[theorem]{Axiom}
\newtheorem{case}[theorem]{Case}
\newtheorem{claim}[theorem]{Claim}
\newtheorem{conclusion}[theorem]{Conclusion}
\newtheorem{condition}[theorem]{Condition}
\newtheorem{conjecture}[theorem]{Conjecture}
\newtheorem{corollary}[theorem]{Corollary}
\newtheorem{criterion}[theorem]{Criterion}
\newtheorem{definition}{Definition} % Number definitions on their own
\newtheorem{derivation}{Derivation} % Number derivations on their own
\newtheorem{example}[theorem]{Example}
\newtheorem{exercise}[theorem]{Exercise}
\newtheorem{lemma}[theorem]{Lemma}
\newtheorem{notation}[theorem]{Notation}
\newtheorem{problem}[theorem]{Problem}
\newtheorem{proposition}{Proposition} % Number propositions on their own
\newtheorem{remark}[theorem]{Remark}
\newtheorem{solution}[theorem]{Solution}
\newtheorem{summary}[theorem]{Summary}
\bibliographystyle{aer}
\newcommand\ve{\varepsilon}
\renewcommand\theenumi{\roman{enumi}}
\newcommand\norm[1]{\left\lVert#1\right\rVert}

\begin{document}

\title{Math 320 Homework 4.6}
\author{Chris Rytting}
\maketitle

\subsection*{4.30}

By exercise 4.10 and example 4.2.6, we have that
\begin{align*}
    \mathscr{F}^{-1}( \text{sinc} (at)) =  (\text{rect}_a(\xi) ) = \frac{1}{2    \pi} \hat f(-t) = \frac{1}{2 \pi} \mathscr{F}(\text{sinc} (-at)
\end{align*}
but since sinc is even, we have
\begin{align*}
    \text{rect}_a(\xi)  = \frac{1}{2    \pi} \hat f(-t) = \frac{1}{2 \pi} \mathscr{F}(\text{sinc} (-at) = \frac{1}{2 \pi} \mathscr{F}(\text{sinc} (at))
\end{align*}
\[\implies 2 \pi \text{rect}_a(\xi) = \mathscr{F}(\text{sinc} (at)\]

\subsection*{4.31}

We need to shift $f$ by $\frac{b+a}{2}$. To do so, we let 
\[g(t) = e^{-it (\frac{b+a}{2}} f(t) \]
where $g$ is a symmetric version of $f$.\\\\
Thus, we have that 
\[g(t) = \sum^{\infty}_{-\infty} g(t_k) \text{sinc} (\frac{b-a}{2}t - k \pi)\]
Therefore,
\[f(t) = \sum^{\infty}_{-\infty} e^{it (\frac{b+a}{2}} e^{-it_k (\frac{b+a}{2}} f(t_k) \text{sinc} (\frac{b-a}{2}t - k \pi) \]

\[\implies f(t) = \sum^{\infty}_{-\infty} e^{i(t-t_k) \frac{b+a}{2}} f(t_k) \text{sinc} (\frac{b-a}{2}t - k \pi) \] 
which is the desired result.
\subsection*{4.32}

Suppose to the contrary that this holds for $N=1$. Then we have that

\begin{align*}
    \text{sin} (t)&= \sum^{\infty}_{-\infty} \text{sin} (t_k) \text{sinc} (Nt - \pi k)  \\
    &= \sum^{\infty}_{-\infty} \text{sin} (\frac{k \pi}{N}) \text{sinc} (Nt - \pi k)  \\
    &= \sum^{\infty}_{-\infty} \text{sin} (k \pi) \text{sinc} (Nt - \pi k)  \\
\end{align*}
However, since $k \in \mathbb{Z}$, we have that $\text{sin} (k\pi) = 0 \quad \forall k$, yielding that
\[\text{sin} (t)  \sum^{\infty}_{-\infty} \text{sin} (k \pi) \text{sinc} (Nt - \pi k) = 0 \forall t\]
this is a contradiction since for 
\[\text{sin} (\frac{\pi}{2})  \neq 0\]







\end{document}
