\documentclass[letterpaper,12pt]{article}

\usepackage{threeparttable}
\usepackage{geometry}
\geometry{letterpaper,tmargin=1in,bmargin=1in,lmargin=1.25in,rmargin=1.25in}
\usepackage[format=hang,font=normalsize,labelfont=bf]{caption}
\usepackage{amsmath}
\usepackage{mathrsfs}
\usepackage{multirow}
\usepackage{array}
\usepackage{delarray}
\usepackage{listings}
\usepackage{amssymb}
\usepackage{amsthm}
\usepackage{lscape}
\usepackage{natbib}
\usepackage{setspace}
\usepackage{float,color}
\usepackage[pdftex]{graphicx}
\usepackage{pdfsync}
\usepackage{verbatim}
\usepackage{placeins}
\usepackage{geometry}
\usepackage{pdflscape}
\synctex=1
\usepackage{hyperref}
\hypersetup{colorlinks,linkcolor=red,urlcolor=blue,citecolor=red}
\usepackage{bm}


\theoremstyle{definition}
\newtheorem{theorem}{Theorem}
\newtheorem{acknowledgement}[theorem]{Acknowledgement}
\newtheorem{algorithm}[theorem]{Algorithm}
\newtheorem{axiom}[theorem]{Axiom}
\newtheorem{case}[theorem]{Case}
\newtheorem{claim}[theorem]{Claim}
\newtheorem{conclusion}[theorem]{Conclusion}
\newtheorem{condition}[theorem]{Condition}
\newtheorem{conjecture}[theorem]{Conjecture}
\newtheorem{corollary}[theorem]{Corollary}
\newtheorem{criterion}[theorem]{Criterion}
\newtheorem{definition}{Definition} % Number definitions on their own
\newtheorem{derivation}{Derivation} % Number derivations on their own
\newtheorem{example}[theorem]{Example}
\newtheorem{exercise}[theorem]{Exercise}
\newtheorem{lemma}[theorem]{Lemma}
\newtheorem{notation}[theorem]{Notation}
\newtheorem{problem}[theorem]{Problem}
\newtheorem{proposition}{Proposition} % Number propositions on their own
\newtheorem{remark}[theorem]{Remark}
\newtheorem{solution}[theorem]{Solution}
\newtheorem{summary}[theorem]{Summary}
\bibliographystyle{aer}
\newcommand\ve{\varepsilon}
\renewcommand\theenumi{\roman{enumi}}
\newcommand\norm[1]{\left\lVert#1\right\rVert}

\begin{document}

\title{Math 344 Homework 5.5}
\author{Chris Rytting}
\maketitle

\subsection*{5.30}


If we have that 
\[(f_n)_{n=0}^\infty\] is converges uniformly, then \[\exists N \in \mathbb{N} \text{ s.t. } |f_n(x) - f(x)| < \epsilon \quad \forall \mathbf{x} \in X\]
And by Corollary 5.2.35. 
\[f(x) = lim_{n\to \infty} f_n(x)\]
\subsection*{5.31 (i)}


Let
        \[
            f(x) = 
            \begin{cases}
                1 & \text{if } x = \frac{\pi}{2}\\
                0 & \text{else }
            \end{cases}
            \]
            for $x\in[0,\frac{\pi}{2}]$.
            We note that $\sin(x) < 1 \quad \forall ~x~ \in [0,\pi]\smallsetminus \{\frac{\pi}{2}\}$\\
            Therefore $\sin^n(x) \to 0$ and $\sin^n(\frac{\pi}{2})=1 \quad \forall n$.\\
            Therefore $\sin^n(x) \to \begin{cases} 
                1 & \text{if } x = \frac{\pi}{2}\\
                0 & \text{otherwise }
            \end{cases}$

\subsection*{5.31 (iii)}
            Let 
\[\epsilon = \frac{1}{2}, \nexists \delta \text{ s.t. } \forall x \in B(\frac{\pi}{2}, \delta) \implies \|sin^n(x) - f(x)\|_\infty < \frac{1}{2}\] Since every epsilon potential ball contains the point $\sin^n(\frac{\pi}{2}) = 1$, it is not uniformly convergent.

\subsection*{5.31 (iv)}
        Sequence not Cauchy.

\subsection*{5.32 (i)}
Note, by ratio test, we have
\begin{align*}
    \left\|(-1)^{k+1} \frac{A^{2(k+1)}}{(2(k+1))!} \right\| \left\|(-1)^k \frac{A^{2k}}{(2k)!}\right\|^{-1} 
    & = \left\| \frac{A^{2k+2}}{(2k+2)!}\right\| \left\| \frac{A^{2k}}{(2k)!}\right\|^{-1} \\ 
    &= \left\| \frac{A^{2}}{(2k+2)(2k+1)} \right\|
\end{align*}
which approaches 0 as $k\rightarrow \infty$.
\\
\subsection*{5.32 (ii)}
Note, by ratio test, we have
\begin{align*}
    \left\|(-1)^{k+1} \frac{A^{2(k+1)+1}}{(2(k+1)+1)!} \right\| \left\|(-1)^k \frac{A^{2k+1}}{(2k+1)!}\right\|^{-1} 
    & = \left\| \frac{A^{2k+3}}{(2k+3)!}\right\| \left\| \frac{A^{2k+1}}{(2k+1)!}\right\|^{-1} \\ 
    &= \left\| \frac{A^{2}}{(2k+3)(2k+2)} \right\|
\end{align*}
which approaches 0 as $k\rightarrow \infty$.
\\
\subsection*{5.32 (iii)}
Note, by ratio test, we have
\begin{align*}
    \left\|(-1)^{k} \frac{A^{k+1}}{k+1} \right\| \left\|(-1)^{k-1} \frac{A^{k}}{k}\right\|^{-1} 
    & = \left\|\frac{A^{k+1}}{k+1} \right\| \left\|\frac{A^{k}}{k}\right\|^{-1} \\ 
    &= \left\| \frac{kA}{k+1} \right\|\\
    &= \left\| \frac{k}{k+1} \right\|  \left\|A \right\|\\
&\text{ and as $k \to \infty $}\\
    &= 1  \left\|A \right\|\\
\end{align*}
Since $\|A\| < 1$ as $k\rightarrow \infty$ we know that this value is less than one and by the ratio test converges absolutely.

\subsection*{5.33}
Absolute value of sum is going to zero, and can can always be expressed as less than $\varepsilon$, however, absolute value of summation diverges.\\\\

\[\left(-1 + \frac{1}{2}\right) + \left(\frac{-1}{3} + \frac{1}{4} + \frac{1}{6}\right)\dots\]
adding sufficient terms so that each term is greater than $\frac{1}{10}$, justified by the fact that
\[\sum^\infty_{n=1}\frac{1}{2n} \] 
diverges. Then each term will be greater than $\frac{1}{10}$, and we have that the sum is greater than
\[\left(\frac{1}{10}\right) + \left(\frac{1}{10}\right) + \left(\frac{1}{10}\right) + \left(\frac{1}{10}\right) \dots = \infty\]



\subsection*{5.34}
Assume to the contrary that $(I-A)$ is singular. Therefore, $\exists \textbf{x} \neq 0 $ such that
\[(I-A) \textbf{x} = 0\]
so $A \textbf{x} = \textbf{x}$. Now, scale $ \textbf{x} $ such that $\| \textbf{x} \| = 1 $. Therefore, $\|A \textbf{x} \| = 1  $. Thus 
\[\text{sup}_{\|x\| = 1} \|A \textbf{x}\| \geq 1\]
which is a contradiction. Now by definition, let
\[ \|(I - A)^{-1} \| = \text{sup}_{\|y\| =1}  \|(I-A)^{-1} \textbf{y} \| \]
Now, let $x_y = (I-A)^{-1} \textbf{y} $ or $y = (I-A) x_y$. Then 
\[ \text{sup}_{\|y\| = 1} \| \textbf{x}_y  \|\]
Now we get 
\begin{align*}
    1 &= \| \textbf{y} \| \\ 
    &= \| (I-A) \textbf{x}_y  \| \\ 
    &= \| \textbf{x}_y - A \textbf{x}_y  \| \\ 
    &\geq \| \textbf{x}_y\| - \|A \textbf{x}_y  \| \\ 
    &\geq \| \textbf{x}_y\| - \|A\|\| \textbf{x}_y  \| \\ 
    &= ( 1 - \|A\|)( \textbf{x}_y )  \\ 
\end{align*}
Therefore, we can say
\[ \textbf{x}_y \leq \frac{1}{1 - \|A\|} = (1 - \|A\|)^{-1} \]
\[\implies \|(I-A)^{-1} \textbf{y} \|  < (1-\|A\|)^{-1} \]
\[ \implies \|(I-A)^{-1} \|\leq (1-\|A\|)^{-1}\]
which is the desired result.

\subsection*{5.35}
We know that $\|A\|^{-1} < M$, and $\|E\|< \frac{1}{\|A\|^{-1}}\|$. Suppose to the contrary
\begin{align*}
    \|A+E\|^{-1} &\leq \frac{\|A\|^{-1}}{1-\|E\| \|A^{-1}\|} \\
    \frac{\|A+E\|^{-1}}{\|A\|^{-1}} &\leq 1- \frac{1}{1-\|E\| M} \\
    \frac{\|A\|^{-1}}{\|A+E\|^{-1}}&\geq 1- \frac{1-\|E\| M}{1}\\
    \frac{\|A\|^{-1}}{\|A+E\|^{-1}} + M \|E\| &\leq 1 \\
   \implies  \frac{1}{M} &< \|A+E\| + \frac{1}{\|A^{-1}\|} \\
    \frac{1}{M} - \frac{1}{\|A^{-1}\|} &< \|A+E\|\\
    \implies 
    \|A+E\| &< 0
\end{align*}
a contradiction.




\end{document}
