\documentclass[letterpaper,12pt]{article}

\usepackage{threeparttable}
\usepackage{geometry}
\geometry{letterpaper,tmargin=1in,bmargin=1in,lmargin=1.25in,rmargin=1.25in}
\usepackage[format=hang,font=normalsize,labelfont=bf]{caption}
\usepackage{amsmath}
\usepackage{mathrsfs}
\usepackage{dsfont}
\usepackage{multirow}
\usepackage{array}
\usepackage{delarray}
\usepackage{listings}
\usepackage{amssymb}
\usepackage{amsthm}
\usepackage{lscape}
\usepackage{natbib}
\usepackage{setspace}
\usepackage{float,color}
\usepackage[pdftex]{graphicx}
\usepackage{pdfsync}
\usepackage{verbatim}
\usepackage{placeins}
\usepackage{geometry}
\usepackage{pdflscape}
\synctex=1
\usepackage{hyperref}
\hypersetup{colorlinks,linkcolor=red,urlcolor=blue,citecolor=red}
\usepackage{bm}


\theoremstyle{definition}
\newtheorem{theorem}{Theorem}
\newtheorem{acknowledgement}[theorem]{Acknowledgement}
\newtheorem{algorithm}[theorem]{Algorithm}
\newtheorem{axiom}[theorem]{Axiom}
\newtheorem{case}[theorem]{Case}
\newtheorem{claim}[theorem]{Claim}
\newtheorem{conclusion}[theorem]{Conclusion}
\newtheorem{condition}[theorem]{Condition}
\newtheorem{conjecture}[theorem]{Conjecture}
\newtheorem{corollary}[theorem]{Corollary}
\newtheorem{criterion}[theorem]{Criterion}
\newtheorem{definition}{Definition} % Number definitions on their own
\newtheorem{derivation}{Derivation} % Number derivations on their own
\newtheorem{example}[theorem]{Example}
\newtheorem{exercise}[theorem]{Exercise}
\newtheorem{lemma}[theorem]{Lemma}
\newtheorem{notation}[theorem]{Notation}
\newtheorem{problem}[theorem]{Problem}
\newtheorem{proposition}{Proposition} % Number propositions on their own
\newtheorem{remark}[theorem]{Remark}
\newtheorem{solution}[theorem]{Solution}
\newtheorem{summary}[theorem]{Summary}
\bibliographystyle{aer}
\newcommand\ve{\varepsilon}
\renewcommand\theenumi{\roman{enumi}}
\newcommand\norm[1]{\left\lVert#1\right\rVert}

\begin{document}

\title{Math 344 Homework }
\author{Chris Rytting}
\maketitle

\subsection*{5.42}


By (i) of 5.7.13, 
\[ \left\| \int^b_a f(t)dt\right\| \leq (b-a)\sup_{t\in[a,b]} \|f(t)\| \]
Now, $\|f(t)\|$ is a real valued function, so
\[\sup_{t\in[a,b]} \|f(t)\| \leq \|f(t)\|\]
\begin{align*}
\implies \int_a^b \|f(t)\|dt & = (b-a) \|f(t)\|\\
\left\|\int^b_a f(t) dt \right\| & \leq (b-a) \sup_{t\in[a,b]} \|f(t)\| \leq (b-a) \|f(t)\|
\end{align*}

\subsection*{5.43}
\begin{align*}
    \int_\alpha^\beta f(t) dt & = \int_\alpha ^\gamma f(t) f(t) \mathds{1}_{[\alpha,\gamma]}
\end{align*}
by (i) of 5.7.13,
\[\int^\beta _\alpha f(t) dt = \int_\alpha ^\gamma f(t)dt + \int_\gamma ^\beta f(t)dt \]


\subsection*{5.44}

Let 
\[\delta (x, y) < \delta, \quad \epsilon = \frac{\delta}{M}\]
Now we have that
\[ F(x) = \int_a^b f(x)dx\]
is a bounded and linear transformation since it is continuous.
\[\implies\int_a^b f(t)dt<M\]
since $M>0$. This yields
\[\big|\int_a^b f(x)dt - \int_a^b f(y)dt\big|=\left|F(x) - F(y)\right|<M\delta<\epsilon\]

\subsection*{5.45}
Let 
\[d(x,y)<\delta = \frac{\epsilon}{2}\] Then
\[|x^{1/3}-y^{1/3}|\leq 1\]
\[\implies |x^{1/3}- y^{1/3}| < \frac{|x-y|\epsilon}{2} < \epsilon\]
We know that $x,y\in [-1,1]$. However,
\[f(x)= \frac{x^{-2/3}}{3}\]
So we have 
\[\lim_{x\to 0^-}f(x) = - \infty\] and
\[\lim_{x\to 0^+}f(x) = \infty\]
derivative is not bounded though, so it is not in the space of continuous functions.

\subsection*{5.46}

There exists no finite partition such that 
\[f(x) = \begin{cases} 0 & \text{ if } x \neq 0\\
    1 &\text{ if } x =0
\end{cases}\]
This function is not in the space of step functions.  
\[ f \not \in S([a,b];X)\]
and the regulated integral is undefined.
\end{document}
