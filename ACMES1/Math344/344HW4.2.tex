\documentclass[letterpaper,12pt]{article}

\usepackage{threeparttable}
\usepackage{geometry}
\geometry{letterpaper,tmargin=1in,bmargin=1in,lmargin=1.25in,rmargin=1.25in}
\usepackage[format=hang,font=normalsize,labelfont=bf]{caption}
\usepackage{amsmath}
\usepackage{mathrsfs}
\usepackage{multirow}
\usepackage{array}
\usepackage{delarray}
\usepackage{listings}
\usepackage{amssymb}
\usepackage{amsthm}
\usepackage{lscape}
\usepackage{natbib}
\usepackage{setspace}
\usepackage{float,color}
\usepackage[pdftex]{graphicx}
\usepackage{pdfsync}
\usepackage{verbatim}
\usepackage{placeins}
\usepackage{geometry}
\usepackage{pdflscape}
\synctex=1
\usepackage{hyperref}
\hypersetup{colorlinks,linkcolor=red,urlcolor=blue,citecolor=red}
\usepackage{bm}


\theoremstyle{definition}
\newtheorem{theorem}{Theorem}
\newtheorem{acknowledgement}[theorem]{Acknowledgement}
\newtheorem{algorithm}[theorem]{Algorithm}
\newtheorem{axiom}[theorem]{Axiom}
\newtheorem{case}[theorem]{Case}
\newtheorem{claim}[theorem]{Claim}
\newtheorem{conclusion}[theorem]{Conclusion}
\newtheorem{condition}[theorem]{Condition}
\newtheorem{conjecture}[theorem]{Conjecture}
\newtheorem{corollary}[theorem]{Corollary}
\newtheorem{criterion}[theorem]{Criterion}
\newtheorem{definition}{Definition} % Number definitions on their own
\newtheorem{derivation}{Derivation} % Number derivations on their own
\newtheorem{example}[theorem]{Example}
\newtheorem{exercise}[theorem]{Exercise}
\newtheorem{lemma}[theorem]{Lemma}
\newtheorem{notation}[theorem]{Notation}
\newtheorem{problem}[theorem]{Problem}
\newtheorem{proposition}{Proposition} % Number propositions on their own
\newtheorem{remark}[theorem]{Remark}
\newtheorem{solution}[theorem]{Solution}
\newtheorem{summary}[theorem]{Summary}
\bibliographystyle{aer}
\newcommand\ve{\varepsilon}
\renewcommand\theenumi{\roman{enumi}}
\newcommand\norm[1]{\left\lVert#1\right\rVert}

\begin{document}

\title{Math 344 Homework 4.2}
\author{Chris Rytting}
\maketitle

\subsection*{4.7 (i)}
No element of this set is a linear combination of other elements of the set, so we have that the set is linearly independent, and we know by assumption that the set spans $C^{\infty}$. Therefore, it forms a basis.

\subsection*{4.7 (ii)}
\[
D = 
\begin{bmatrix}
    0 & -1 & 0 & 0 \\
    1 & 0 & 0 & 0 \\
    0 & 0 & 0 & -2 \\
    0 & 0 & 2 & 0 \\
\end{bmatrix}
\]

\subsection*{4.7(iii)}

Consider one space spanned by $sin(2x)$ and $cos(2x)$. Upon multiplying $D$ by the vector representations of these spans, namely
\[ \left\{ 
\begin{bmatrix}
    0\\0\\1\\0
\end{bmatrix}\right\}
\text{ and }
\left\{ 
\begin{bmatrix}
    0\\0\\0\\1
\end{bmatrix}\right\}\]
we will have a result in the same span, showing that these subspaces are $D$-invariant.\\\\


Consider another space spanned by $sin(x)$ and $cos(x)$. Upon multiplying $D$ by the vector representations of these spans, namely
\[ \left\{ 
\begin{bmatrix}
    1\\0\\0\\0
\end{bmatrix}\right\}
\text{ and }
\left\{ 
\begin{bmatrix}
    0\\1\\0\\0
\end{bmatrix}\right\}\]
we will have a result in the same span, showing that these subspaces are $D$-invariant.






\subsection*{4.8}

Consider a unidimensional subspace span$\{(0,0,\dots,a_n,0,\dots)\}$ of the vector space $\ell^{\infty}$ and the right shift operated  span$\{(0,0,\dots,0,a_n,\dots)\}$. There exists no linear combination of elements from the first subspace that results in an element from the second, and therefore it is not invariant. 


\subsection*{4.9}


It suffices to show that $T'(a') = T'(b')$ if $a'=b'$. Now, let $a' = a + W$ and $b' = b + W$ where $a',b' \in V/W$. We will consider two cases, that $a,b \in W$ and that $a,b \not \in W$. \\\\
Let $a,b \in W$. Then 
\[T'(a') = T'(a + W) = T(a) + W\]
and 
\[T'(b') = T'(b + W) = T(b) + W\]
and we have that, since by the definition of an invariant subspace, $T(a) \in W$ and $T(b) \in W$, that $T(a) + W = T(b) + W$ are in the same equivalence class and we have the desired result.\\\\

Now let $a,b \not\in W$. We still have, though, that 
\[a' = a + W = b' = b+W\] 
Now, there exists a vector $c$ such that, since $a,b$ are in the same equivalence class, if we let 
\[a'' = a - c + W\]
and
\[b'' = b - c + W\]
we still have that $a'' = b''$
and that 
\[T'(a'') = T'(a - c + W) = T(a) - T(c) + W\]
and 
\[T'(b'') = T'(b - c + W) = T(b) - T(c) + W\]
and, by the same logic as the other case, we have that
\[T(a) - T(c) + W = T(b) - T(c) + W\]
\[\implies T(a)  + W = T(b)  + W\]
Which is the desired result.




\subsection*{4.10}

\[(i) \implies (ii)\]
For $w_1 \in W_1$ and $w_2 \in W_2$ note that
\[ R^2(w_1 + w_2) = R(w_1 - w_2) = w_1 + w_2 \]
Which is the desired result.
\\

\[(ii) \implies (iii)\]
To find the nullspace fo $(R-I)$, we note that
\[ (R-I)v = 0 \implies Rv - v = 0 \implies Rv = v \]
yielding
\[ \mathscr{N}(R - I) = \{v \in V | Rv = v \} \]
and 
\[ (R+I)v = 0 \implies Rv + v = 0 \implies Rv = -v \]
yielding
\[ \mathscr{N}(R + I) = \{v \in V | Rv = -v \} \]
It is clear that 
\[ \mathscr{N}(R + I) \cap
\mathscr{N}(R + I) = \{ 0\}\]
Now, to show that these two spaces comprise all of $V$, we consider the vector 
\[v\in V = \frac{1}{2}(I-R)v +  \frac{1}{2}(I+R)v\]
and we have that 
\[(R-I) (\frac{1}{2} (I + R)v) = \frac{1}{2}(R + R^2 - I - R)v =\frac{1}{2}(R + I - I - R)v = 0\]
and
\[(R+I) (\frac{1}{2} (I - R)v) = \frac{1}{2}(R - R^2 + I - R)v =\frac{1}{2}(R - I + I - R)v = 0\]
implying that the first and second terms of $v$ are in the null space of $(R-I)$ and $(R+I)$, respectively. However, any $v\in V$ can be expressed as a linear combination of these nullspaces, implying that they span the whole space, and we have the desired result that they are complimentary.


\[(iii) \implies (i)\]
Let $W_1 = \mathscr{N}(R-I)$ and $W_2 = \mathscr{N}(R+I)$. We know by part (ii), that 
\[R(w_1) = w_1 \quad w_1 \in W_1\] and that 
\[R(w_2) = -w_2 \quad w_2 \in W_2\] 
Now, $R$ is a linear operator, so we have that $R(w_1 + w_2) = R(w_1) + R(w_2)= w_1 - w_2$. Therefore, $R$ is a reflection.



\subsection*{4.11 (i)}
\[ \text{span} \left\{ \begin{bmatrix} 1 \\ 3 \end{bmatrix} \right\} \quad \text{span} \left\{ \begin{bmatrix} -3 \\ 1 \end{bmatrix} \right\} \]


\subsection*{4.11 (ii)}

\[ T = \left\{ \begin{bmatrix} 1 \\ 3 \end{bmatrix}, \begin{bmatrix} -3 \\ 1 \end{bmatrix} \right\} 
\quad
L_{TT} = \begin{bmatrix} 1 & 0 \\ 0 & -1 \end{bmatrix}
\] 
\subsection*{4.11 (iii)}
\[
C_{ST} = \begin{bmatrix} 1 & 3 \\ 3 & -1 \end{bmatrix}
\quad
C_{TS} = \begin{bmatrix} 1/10 & 3/10 \\ 3/10 & -1/10 \end{bmatrix}
\]
\subsection*{4.11 (iv)}
\[   
L = C_{ST}L_{TT}C_{TS} = 
\begin{bmatrix} 1 & 3 \\ 3 & -1 \end{bmatrix}
\begin{bmatrix} 1 & 0 \\ 0 & -1 \end{bmatrix}
\begin{bmatrix} 1/10 & 3/10 \\ 3/10 & -1/10 \end{bmatrix}
=
\begin{bmatrix} -4/5 & 3/5 \\ 3/5 & 4/5 \end{bmatrix}
\]





\end{document}
