\documentclass[letterpaper,12pt]{article}

\usepackage{threeparttable}
\usepackage{geometry}
\geometry{letterpaper,tmargin=1in,bmargin=1in,lmargin=1.25in,rmargin=1.25in}
\usepackage[format=hang,font=normalsize,labelfont=bf]{caption}
\usepackage{amsmath}
\usepackage{mathrsfs}
\usepackage{multirow}
\usepackage{array}
\usepackage{delarray}
\usepackage{listings}
\usepackage{amssymb}
\usepackage{amsthm}
\usepackage{lscape}
\usepackage{natbib}
\usepackage{setspace}
\usepackage{float,color}
\usepackage[pdftex]{graphicx}
\usepackage{pdfsync}
\usepackage{verbatim}
\usepackage{placeins}
\usepackage{geometry}
\usepackage{pdflscape}
\synctex=1
\usepackage{hyperref}
\hypersetup{colorlinks,linkcolor=red,urlcolor=blue,citecolor=red}
\usepackage{bm}


\theoremstyle{definition}
\newtheorem{theorem}{Theorem}
\newtheorem{acknowledgement}[theorem]{Acknowledgement}
\newtheorem{algorithm}[theorem]{Algorithm}
\newtheorem{axiom}[theorem]{Axiom}
\newtheorem{case}[theorem]{Case}
\newtheorem{claim}[theorem]{Claim}
\newtheorem{conclusion}[theorem]{Conclusion}
\newtheorem{condition}[theorem]{Condition}
\newtheorem{conjecture}[theorem]{Conjecture}
\newtheorem{corollary}[theorem]{Corollary}
\newtheorem{criterion}[theorem]{Criterion}
\newtheorem{definition}{Definition} % Number definitions on their own
\newtheorem{derivation}{Derivation} % Number derivations on their own
\newtheorem{example}[theorem]{Example}
\newtheorem{exercise}[theorem]{Exercise}
\newtheorem{lemma}[theorem]{Lemma}
\newtheorem{notation}[theorem]{Notation}
\newtheorem{problem}[theorem]{Problem}
\newtheorem{proposition}{Proposition} % Number propositions on their own
\newtheorem{remark}[theorem]{Remark}
\newtheorem{solution}[theorem]{Solution}
\newtheorem{summary}[theorem]{Summary}
\bibliographystyle{aer}
\newcommand\ve{\varepsilon}
\renewcommand\theenumi{\roman{enumi}}
\newcommand\norm[1]{\left\lVert#1\right\rVert}

\begin{document}

\title{Math 344 Homework 3.3}
\author{Chris Rytting}
\maketitle



\subsection*{Exercise 3.12}
If this process is applied to a linearly dependent then this process will yield zero vectors because they are linear combinations of each other. 

\subsection*{3.13}
Gram-Schmidt yields the basis
\[B=
\begin{bmatrix}
    \frac{1}{\sqrt{2}}&\frac{1}{\sqrt{2}}\\
    \frac{1}{\sqrt{2}}&-\frac{1}{\sqrt{2}}
\end{bmatrix}
\]
By Theorem 3.2.3, we have that each $a_i$, or in other words, the coefficients that we multiply the vectors $x_i$(in this case, the vectors of $B$) by to yield $x (\text{in this case }(2e_1 + 3e_2) = 
\begin{bmatrix}
    2\\3
\end{bmatrix})$, is given by $\langle x,x_i \rangle = a_i$.
This yields $a_1 = \frac{5}{\sqrt{2}} \text{ and } a_2 = \frac{-1}{\sqrt{2}}$, and we have that
\[
\begin{bmatrix}
    \frac{1}{\sqrt{2}} & \frac{1}{\sqrt{2}}\\
    \frac{1}{\sqrt{2}} & \frac{-1}{\sqrt{2}}
\end{bmatrix}
\begin{bmatrix}
    \frac{5}{\sqrt{2}}\\
    \frac{-1}{\sqrt{2}} 
\end{bmatrix}
=
\begin{bmatrix}
    2\\
    3
\end{bmatrix}
\]




\subsection*{Exercise 3.14}
The set of orthogonal vectors for this set are:
\[E=\{\frac{1}{\sqrt{\pi}},\frac{x}{\sqrt{\frac{\pi}{2}}},\frac{x^2-\frac{1}{2}}{\sqrt{\frac{\pi}{8}}},\frac{x^3-\frac{3}{4}x}{\sqrt{\frac{7\pi}{8}}}\}\]
These are all orthogonal to each other.




\end{document}










