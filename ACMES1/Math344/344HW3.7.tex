\documentclass[letterpaper,12pt]{article}

\usepackage{threeparttable}
\usepackage{geometry}
\geometry{letterpaper,tmargin=1in,bmargin=1in,lmargin=1.25in,rmargin=1.25in}
\usepackage[format=hang,font=normalsize,labelfont=bf]{caption}
\usepackage{amsmath}
\usepackage{multirow}
\usepackage{array}
\usepackage{delarray}
\usepackage{amssymb}
\usepackage{amsthm}
\usepackage{lscape}
\usepackage{natbib}
\usepackage{setspace}
\usepackage{float,color}
\usepackage[pdftex]{graphicx}
\usepackage{mathrsfs}  
\usepackage{pdfsync}
\usepackage{verbatim}
\usepackage{placeins} \usepackage{geometry}
\usepackage{pdflscape}
\synctex=1
\usepackage{hyperref}
\hypersetup{colorlinks,linkcolor=red,urlcolor=blue,citecolor=red}
\usepackage{bm}
\usepackage{amssymb}


\theoremstyle{definition}
\newtheorem{theorem}{Theorem}
\newtheorem{acknowledgement}[theorem]{Acknowledgement}
\newtheorem{algorithm}[theorem]{Algorithm}
\newtheorem{axiom}[theorem]{Axiom}
\newtheorem{case}[theorem]{Case}
\newtheorem{claim}[theorem]{Claim}
\newtheorem{conclusion}[theorem]{Conclusion}
\newtheorem{condition}[theorem]{Condition}
\newtheorem{conjecture}[theorem]{Conjecture}
\newtheorem{corollary}[theorem]{Corollary}
\newtheorem{criterion}[theorem]{Criterion}
\newtheorem{definition}{Definition} % Number definitions on their own
\newtheorem{derivation}{Derivation} % Number derivations on their own
\newtheorem{example}[theorem]{Example}
\newtheorem*{subsection*}{Exercise} % Number subsection*s on their own
\newtheorem{lemma}[theorem]{Lemma}
\newtheorem{notation}[theorem]{Notation}
\newtheorem{problem}[theorem]{Problem}
\newtheorem{proposition}{Proposition} % Number propositions on their own
\newtheorem{remark}[theorem]{Remark}
\newtheorem{solution}[theorem]{Solution}
\newtheorem{summary}[theorem]{Summary}
\bibliographystyle{aer}
\newcommand\ve{\varepsilon}
\renewcommand\theenumi{\roman{enumi}}

\title{Math 344 Homework 3.7}
\author{Chris Rytting}


\begin{document}
\maketitle
\subsection*{3.37}


\[D = \begin{bmatrix} 
0&1&0\\
0&0&2\\
0&0&0\\
\end{bmatrix}
D^*  = \begin{bmatrix} 
0&0&0\\
1&0&0\\
0&2&0\\
\end{bmatrix}\]\\
\smallskip\\
\[\mathscr{N}(D) = \text{span}\{(1,0,0)^T\}\]\\
\[\mathscr{R}(D) = \text{span}\{(0,1,0)^T,(0,0,1)^T\} \]\\
\[\mathscr{N}(D^*) = \text{span}\{(0,0,1)^T\}\]\\
\[\mathscr{R}(D^*) =  \text{span}\{(0,0,1)^T,(0,1,0)^T\}\]

\subsection*{3.38}
\[A=\begin{bmatrix}
1 &1&1& 0 \\
0&0&0& 0\\
2&2&2& 0
\end{bmatrix}\]\\
\[A^* = A^H =\begin{bmatrix}
1 &0&2\\
1&0&2\\
1&0&2\\
0&0&0
\end{bmatrix}\]\\
\[\mathscr{N}(A) = \text{span}\{(0,0,0,1)^T, (1,0,-1,0)^T, (-1,1,0,0)^T\}\]\\
\[\mathscr{R}(A) = \text{span}\{(1,0,0,0)^T\}\]\\
\[\mathscr{R}(A^*) = \text{span}\{(1,0,0)^T\}\]\\
\[\mathscr{N}(A^*) = \text{span}\{(0,1,0)^T, (2,0,-1)^T\} \]\\

\subsection*{3.39 (i)}
\begin{align*}
    \langle \mathbf{w}, (S+T)\mathbf{n} \rangle_w &= \langle \mathbf{w}, S\mathbf{v}+T\mathbf{v} \rangle_w \\
    & = \langle \mathbf{w}, S\mathbf{v} \rangle_w + \langle \mathbf{w}, T\mathbf{v} \rangle_w \\
    & = \langle S^*\mathbf{w}, \mathbf{v} \rangle_v +\langle T^*\mathbf{w}, \mathbf{v} \rangle_v \\
    & = \langle (S^* + T^*)\mathbf{w}, \mathbf{v} \rangle_v
\end{align*}\\
And we have that $(S+T)^* = S^* + T^*$
\begin{align*}
    \langle \mathbf{w}, \alpha  T\mathbf{v} \rangle_w &= \langle \bar \alpha \mathbf{w}, T\mathbf{v} \rangle_w \\
    & = \langle \bar \alpha  \mathbf{w}, T^*\mathbf{v} \rangle_v
\end{align*}
\\
And we have that $(\alpha T)^* = \bar \alpha T ^*$\\
\subsection*{3.39 (ii)}
\begin{align*}
    \langle \mathbf{w}, S\mathbf{v} \rangle_w & = \langle S^*\mathbf{w}, \mathbf{v} \rangle_v = \langle \mathbf{w}, (S^*)^*\mathbf{v} \rangle_w
\end{align*}
\\
And we have that $(S^*)^* = S$\\
\subsection*{3.39 (iii)}
\begin{align*}
    \langle \mathbf{v}, ST\mathbf{v} \rangle & = \langle S^*\mathbf{v}, T\mathbf{v} \rangle = \langle T^*S^*\mathbf{v}, \mathbf{v} \rangle 
\end{align*}
\\
And we have that $(ST)^* = T^*S^*$\\
\subsection*{3.39 (iv)}
\begin{align*}
    \langle \mathbf{v}, \mathbf{v} \rangle & = \langle TT^{-1}\mathbf{v}, \mathbf{v} \rangle \\
    & = \langle T^{-1}\mathbf{v}, T^*\mathbf{v} \rangle \\
    & = \langle \mathbf{v}, (T^{-1})^*T^*\mathbf{v} \rangle  = \langle \mathbf{v}, \mathbf{v} \rangle 
\end{align*}\\
And we have that $(T^{-1})^* T^* = I$, implying $(T^{-1})^* = (T^*)^{-1}$.

\subsection*{3.40}
We know that $L^*:W\rightarrow V$ and $\mathbf{v}\in \mathscr{R}(L^*)^\bot$ iff 
\[\langle \mathbf{v}, L^* \mathbf{w} \rangle = 0 \forall \mathbf{w} \in W\]
This is true iff 
\[\langle L\mathbf{v}, \mathbf{w} \rangle =0 \quad \forall \mathbf{v} \in V\]
which happens iff 
\[\mathbf{v} \in \mathscr{N}(L)\]
Therefore, $\mathscr{N}(L) = \mathscr{R}(L^*)^{\bot}$, and by Lemma 3.7.18, $\mathscr{N}(L)^\bot = \mathscr{R}(L^*)$


\subsection*{3.41 (i)}
It is sufficient to show $\langle Y,AX\rangle = \langle A^HY,X\rangle$\\
\begin{align*}
\langle Y, AX\rangle = tr(Y^HAX) = tr(A^HY)^HX)= \langle A^HY,X\rangle
\end{align*}
\subsection*{3.41 (ii)}
\begin{align*}
\langle V,WA \rangle = tr(V^HWA) = tr(AV^HW) = tr((VA^H)^HW)=\langle VA^*,W\rangle
\end{align*}
\subsection*{3.41 (iii)}
\begin{align*}
    \langle (T_A(B))^*,C\rangle & = \langle B,T_A(C)\rangle\\
    & = \langle B, AC-CA \rangle = tr(B^H(AC_CA))  \\ 
    & =tr(B^HAC)-tr(B^HCA)=\langle B,AC\rangle -\langle B,CA\rangle\\
    & = \langle A^*B,C\rangle - \langle BA^*,C\rangle \\
    & = tr(B^H(A^*)^HC)-tr((A^*)^HB^HC) = tr(B^H(A^*)^HC-(A^*)^HB^HC)\\ 
    & = \langle A^*B-BA^*,C\rangle\\
    & = \langle T_{A^*}(B),C\rangle \\
\end{align*}\\
And we have that $(T_A)^* = T_{A^*}$

\subsection*{3.42}
Note that $A^* = A^H$, and if there exists a solution to 
\[A \mathbf{x} = \mathbf{b}  \implies \mathbf{x} \in \mathscr{R}(A)\]
Otherwise, it would be the case that 
\[\mathbf{x} \in \mathscr{R}(A)^\bot\]
because they are complement subspaces. By Theorem 3.7.21, we know 
\[\mathscr{R}(A)^\bot = \mathscr{N}(A^H)\]
Therefore, we have that \[\mathbf{x} \in \mathscr{R}(A) \quad \text{ or }\quad \mathbf{x} \in \mathscr{N}(A^H)\]



\subsection*{3.43}

If $A$ and $B$ are arbitrary matrices in the spaces $\text{Sym}_n(\mathbb{R})$ and $\text{Skew}_n(\mathbb{R})$, respectively, then So $A^T = A$ and $B^T = -B$. We also know that the following is true:
\[tr(CD) = tr(DC) \quad\text{and}\quad tr(C) = tr(C^T)\]
\[\implies \langle A,B \rangle = \text{tr}(A^TB) = \text{tr}((-1)A^TB^T) =  -\text{tr}(A^TB^T) = -\text{tr}((BA)^T) =  -\text{tr}(AB)\] but $\text{tr}(AB) = -\text{tr}(AB)$ only holds iff $\text{tr}(AB) =  0$. \\\\
Therefore, the set of symme\text{tr}ic ma\text{tr}ices in $M$ is orthogonal to the set of skew ma\text{tr}ices.



\end{document}

