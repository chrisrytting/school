\documentclass[letterpaper,12pt]{article}

\usepackage{threeparttable}
\usepackage{geometry}
\geometry{letterpaper,tmargin=1in,bmargin=1in,lmargin=1.25in,rmargin=1.25in}
\usepackage[format=hang,font=normalsize,labelfont=bf]{caption}
\usepackage{amsmath}
\usepackage{mathrsfs}
\usepackage{multirow}
\usepackage{array}
\usepackage{delarray}
\usepackage{listings}
\usepackage{amssymb}
\usepackage{amsthm}
\usepackage{lscape}
\usepackage{natbib}
\usepackage{setspace}
\usepackage{float,color}
\usepackage[pdftex]{graphicx}
\usepackage{pdfsync}
\usepackage{verbatim}
\usepackage{placeins}
\usepackage{geometry}
\usepackage{pdflscape}
\synctex=1
\usepackage{hyperref}
\hypersetup{colorlinks,linkcolor=red,urlcolor=blue,citecolor=red}
\usepackage{bm}


\theoremstyle{definition}
\newtheorem{theorem}{Theorem}
\newtheorem{acknowledgement}[theorem]{Acknowledgement}
\newtheorem{algorithm}[theorem]{Algorithm}
\newtheorem{axiom}[theorem]{Axiom}
\newtheorem{case}[theorem]{Case}
\newtheorem{claim}[theorem]{Claim}
\newtheorem{conclusion}[theorem]{Conclusion}
\newtheorem{condition}[theorem]{Condition}
\newtheorem{conjecture}[theorem]{Conjecture}
\newtheorem{corollary}[theorem]{Corollary}
\newtheorem{criterion}[theorem]{Criterion}
\newtheorem{definition}{Definition} % Number definitions on their own
\newtheorem{derivation}{Derivation} % Number derivations on their own
\newtheorem{example}[theorem]{Example}
\newtheorem{exercise}[theorem]{Exercise}
\newtheorem{lemma}[theorem]{Lemma}
\newtheorem{notation}[theorem]{Notation}
\newtheorem{problem}[theorem]{Problem}
\newtheorem{proposition}{Proposition} % Number propositions on their own
\newtheorem{remark}[theorem]{Remark}
\newtheorem{solution}[theorem]{Solution}
\newtheorem{summary}[theorem]{Summary}
\bibliographystyle{aer}
\newcommand\ve{\varepsilon}
\renewcommand\theenumi{\roman{enumi}}
\newcommand\norm[1]{\left\lVert#1\right\rVert}

\begin{document}

\title{Math 344 Homework 4.1}
\author{Chris Rytting}
\maketitle

\subsection*{4.1}
If $\lambda$ is an eigenvalue of $A$, then we have that 
\begin{align*}
    Ax &= \lambda x \\
    (Ax)^k &= (\lambda x)^k \\
    A^kx^k &= \lambda^k x^k \\
\end{align*}
If $A$ is nilpotent, though, we have that $A^k = 0$, which implies that
\[0x^k = \lambda^k x^k\]
Which only holds if and only if $\lambda = 0 $, since $x$ is nonzero by the definition 4.1.1.

\subsection*{4.2}
Note that 
\[ D = 
\begin{bmatrix}
    0 & 1 & 0 \\
    0 & 0 & 2 \\
    0 & 0 & 0 \\
\end{bmatrix}
\]
Now, we find eigenvalues by the following calculation
\begin{align*}
    p_A(z) &= \text{det} (z I - A) \\
&= \text{det} \left(  
\begin{bmatrix}
    z & 0 & 0 \\
    0 & z & 0 \\
    0 & 0 & z \\
\end{bmatrix}
-
\begin{bmatrix}
    0 & 1 & 0 \\
    0 & 0 & 2 \\
    0 & 0 & 0 \\
\end{bmatrix}\right)\\
&= \text{det}
\begin{bmatrix}
    z & -1 & 0 \\
    0 & z & -2 \\
    0 & 0 & z \\
\end{bmatrix}\\
&= z^3\\
\implies z &= 0\\
\end{align*}
Now, we have there is one eigenvalue equal to zero with algebraic multiplicity of 3 and geometric multiplicity of one, the eigenvector being 
\[
\begin{bmatrix}
    1\\0\\0
\end{bmatrix}
\]

\subsection*{4.3}
Let $A$ be as follows:
\[A = 
\begin{bmatrix}
    a & b \\
    c & d \\
\end{bmatrix}
\]


The characteristic polynomial is given by 
\begin{align*}
p_A(\lambda) &= \text{det}(\lambda I - A)\\
&= \text{det} \left(
\begin{bmatrix}
    \lambda & 0 \\
    0 & \lambda \\
\end{bmatrix}
- 
\begin{bmatrix}
    a & b \\
    c & d \\
\end{bmatrix}\right)
&=
\text{det}
\begin{bmatrix}
    \lambda-a & b \\
    c & \lambda-d \\
\end{bmatrix}
&= (\lambda-d)(\lambda-a) - bc \\
&= \lambda^2 - \lambda d - a\lambda + ad - bc\\
&= \lambda^2 - \lambda d - a\lambda + ad - bc\\
&= \lambda^2 - \lambda(d + a) + ad - bc\\
&= \lambda^2 - \lambda \text{tr}(A) + \text{det}(A)\\
\end{align*}
as desired.

\subsection*{4.4 (i)}
Let 
\[A =
\begin{bmatrix}
   a & b \\ 
   c & d
\end{bmatrix}  \]

Since $A$ is Hermitian, $c=b$, and using the characteristic polynomial found in 4.3, we have that
\begin{align*}
 \lambda &= \frac{(a+d) \pm \sqrt{(a+d)^2 - 4(1)(ad-b^2)    }   }{    2} \\&= \frac{a+d}{2}+\frac{1}{2}\sqrt{a^2 + 2ad + d^2 -4ad + 4b^2} \\&= \frac{a+d}{2}+\frac{1}{2}\sqrt{a^2 - 2ad + d^2 + 4b^2 } \\&= \frac{a+d}{2}+\frac{1}{2}\sqrt{(a-d)^2 + 4b^2  }
\end{align*}
 Since all elements of $A$ are real, and $(a-d)^2 + 4b^2 $ will be non-negative, we know that no eigenvalue will be imaginary, implying that all will be real. 

\subsection*{4.4 (ii)}


As $A$ is skew-Hermitian, we know that $-a = \overline a$ and that $-d = \overline d$, implying that $A$ has strictly imaginary numbers on the diagonal. Moreover, $-b = \overline c$ and $-c = \overline b$, implying a matrix $A$ of the form
\[A = 
\begin{bmatrix}
    wi & x + yi \\
    -x + yi  &vi \\
\end{bmatrix}\]
where $x,y,v,w \in \mathbb{R}$
Using the characteristic polynomial and the quadratic formula, we have that
\begin{align*}
 \lambda &= \frac{(a+d) \pm \sqrt{(a+d)^2 - 4(1)(ad-b^2)    }   }{    2} \\&= \frac{a+d}{2}+\frac{1}{2}\sqrt{a^2 + 2ad + d^2 -4ad + 4b^2} \\&= \frac{a+d}{2}+\frac{1}{2}\sqrt{a^2 - 2ad + d^2 + 4b^2 } \\&= \frac{a+d}{2}+\frac{1}{2}\sqrt{(a-d)^2 + 4b^2  }
\end{align*}
Since $(w-v)^2 +4x^2  +4y^2 \in \mathbb{R}_+$ and ${w+v} \over 2$ is real, the sum of these will be real, and a real number multiplied by an imaginary number will be strictly imaginary. 

\subsection*{4.5}
If $A(c-\lambda I)^{-1}B^H$ has an eigenvalue of 1, we have that 
\[(A(c-\lambda I)^{-1}B^H)\textbf{y} = \textbf{y}\]
for some $\textbf{y} \in \mathbb{F}^m$. If $\textbf{x}$ is an eigenvector and $\lambda$ is its eigenvalue  $C-B^HA$ then we have that \[(C-B^HA)\textbf{x} = \lambda \textbf{x}\] Then we have the following:
\begin{align*}
        (C-B^HA)\textbf{x} &= \lambda \textbf{x} \\
            C\textbf{x}-B^HA\textbf{x} &= \lambda \textbf{x} \\
                B^HA\textbf{x} &= C\textbf{x} - \lambda \textbf{x} \\
                    B^HA\textbf{x} &= (C- \lambda I)\textbf{x} \\
                        (C- \lambda I)^{-1} B^HA\textbf{x} &= \textbf{x} \\
                            A(C- \lambda I)^{-1} B^HA\textbf{x} &= A\textbf{x} 
                        \end{align*}
                        Now, letting $\textbf{y} = A\textbf{x}$, we have that \[(A(c-\lambda I)^{-1}B^H)\textbf{y} = \textbf{y}\]
                        Therefore, 1 is an eigenvalue of $A(c-\lambda I)^{-1}B^H$, as desired.


\subsection*{4.6}
Let a matrix $A$ be an upper-triangular matrix. To find its eigenvalues, we find the characteristic polynomial:
\begin{align*}
p_A(z) &= \text{det}(zi - A)\\
&= \text{det}\left(
\begin{bmatrix}
    z & 0 & 0 & \dots & 0 \\
    0 & z & 0 & \dots & 0 \\
    0 & 0 & z & \dots & 0 \\
    \vdots & \vdots & \vdots & \ddots & \vdots \\
    0 & 0 & 0 & \dots & z \\
\end{bmatrix}
-
\begin{bmatrix}
    a_1 & a_2 & a_3 & \dots & a_n \\
    0 & b_1 & b_2 & \dots & b_{n-1} \\
    0 & 0 & c_1 & \dots & c_{n-1}\\
    \vdots & \vdots & \vdots & \ddots & \vdots \\
    0 & 0 & 0 & \dots & n_1 \\
\end{bmatrix}\right)\\
&= 
\text{det}\left(
\begin{bmatrix}
    z - a_1 & -a_2 & -a_3 & \dots &- a_n \\
    0 & z - b_1 & -b_2 & \dots & -b_{n-1} \\
    0 & 0 & z - c_1 & \dots &  -c_{n-1}\\
    \vdots & \vdots & \vdots & \ddots & \vdots \\
    0 & 0 & 0 & \dots & z - n_1 \\
\end{bmatrix}\right)\\
&= 
(z-a_1)(z-b_1)(z-c_1)\dots(z-n_1)
\end{align*}
which yields the desired result, that $a_1, b_1, c_1,\dots, n_1$ are eigenvalues of $A$ since the characteristic polynomial equals zero when they do.






\end{document}
