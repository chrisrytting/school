\documentclass[letterpaper,12pt]{article}

\usepackage{threeparttable}
\usepackage{geometry}
\geometry{letterpaper,tmargin=1in,bmargin=1in,lmargin=1.25in,rmargin=1.25in}
\usepackage[format=hang,font=normalsize,labelfont=bf]{caption}
\usepackage{amsmath}
\usepackage{multirow}
\usepackage{array}
\usepackage{delarray}
\usepackage{amssymb}
\usepackage{amsthm}
\usepackage{lscape}
\usepackage{natbib}
\usepackage{setspace}
\usepackage{float,color}
\usepackage[pdftex]{graphicx}
\usepackage{mathrsfs}  
\usepackage{pdfsync}
\usepackage{verbatim}
\usepackage{placeins} \usepackage{geometry}
\usepackage{pdflscape}
\synctex=1
\usepackage{hyperref}
\hypersetup{colorlinks,linkcolor=red,urlcolor=blue,citecolor=red}
\usepackage{bm}
\usepackage{amssymb}


\theoremstyle{definition}
\newtheorem{theorem}{Theorem}
\newtheorem{acknowledgement}[theorem]{Acknowledgement}
\newtheorem{algorithm}[theorem]{Algorithm}
\newtheorem{axiom}[theorem]{Axiom}
\newtheorem{case}[theorem]{Case}
\newtheorem{claim}[theorem]{Claim}
\newtheorem{conclusion}[theorem]{Conclusion}
\newtheorem{condition}[theorem]{Condition}
\newtheorem{conjecture}[theorem]{Conjecture}
\newtheorem{corollary}[theorem]{Corollary}
\newtheorem{criterion}[theorem]{Criterion}
\newtheorem{definition}{Definition} % Number definitions on their own
\newtheorem{derivation}{Derivation} % Number derivations on their own
\newtheorem{example}[theorem]{Example}
\newtheorem*{subsection*}{Exercise} % Number subsection*s on their own
\newtheorem{lemma}[theorem]{Lemma}
\newtheorem{notation}[theorem]{Notation}
\newtheorem{problem}[theorem]{Problem}
\newtheorem{proposition}{Proposition} % Number propositions on their own
\newtheorem{remark}[theorem]{Remark}
\newtheorem{solution}[theorem]{Solution}
\newtheorem{summary}[theorem]{Summary}
\bibliographystyle{aer}
\newcommand\ve{\varepsilon}
\renewcommand\theenumi{\roman{enumi}}

\title{Math 344 HW 3.6}
\author{Chris Rytting}


\begin{document}
\maketitle
\subsection*{3.31}
By Lemma 3.6.1, to find the minimum let $x = 1$. From the proof for Young's inequality, let $x = 1 = \frac{a^{p-1}}{b}$, \\
\begin{align*}
    ab & = \big( \frac{a^p}{p} + \frac{b^q}{q} \big)
\end{align*}\\
Which will hold iff $ b = a^{p-1}$. Raising both sides to the $q$, we have the following:\\
\begin{align*}
    b^q = a^{pq-q} = a^p
\end{align*}\\
Thus, equality holds iff $b^p = a ^p$.

\subsection*{3.32}
Let $\epsilon \geq 1$. By Young's inequality, we have the following:
\begin{align*}
    ab & \leq \frac{a^2}{2} + \frac{b^2}{2} \\
    & \leq \frac{\epsilon ^2}{\epsilon}\big( \frac{a^2}{2} + \frac{b^2}{2} \big)\\
    & \leq  \frac{a^2 + \epsilon^2 b^2}{2\epsilon} \\
    ab & \leq \frac{a^2}{2\epsilon} + \frac{\epsilon b^2}{2} \\
\end{align*}
Now, suppose $\epsilon < 1$. Thus, by Young's inequality, 
\begin{align*}
    ab & \leq \frac{a^2}{2} + \frac{b^2}{2} \\
    & \leq \frac{1}{\epsilon}\big( \frac{a^2 + b^2}{2} \big)\\
    & \leq \frac{a^2}{2\epsilon} + \frac{b^2}{2\epsilon} \\
    ab & \leq \frac{a^2}{2\epsilon} + \frac{\epsilon b^2}{2} \\
\end{align*}
In both of these cases, $ab \leq \frac{a^2}{2\epsilon} + \frac{\epsilon b^2}{2}$\\

\subsection*{3.33}
Note that if $a=b$, then
\begin{align*}
    a^\theta b^{1-\theta} =a^\theta a^{1-\theta} = a = \theta a + (1-\theta)a = \theta a + (1-\theta)b
\end{align*}
On the other hand, if $a \neq b$
\begin{align*}
a^\theta b^{1-\theta} &\leq \theta a + (1-\theta)b\\
\ln(a^\theta b^{1-\theta}) &\leq \ln(\theta a + (1-\theta)b)
\end{align*}
Due to the convexity of ln, $\theta \ln(a) < \ln(\theta a)$. Therefore,
\begin{align*}
    \ln(a^\theta b^{1-\theta}) &= \theta \ln(a) + (1-\theta)\ln(b)\\
    & < \ln(\theta a ) + \ln( (1-\theta) b) \\
    & < \ln( \theta a + (1- \theta)b) 
\end{align*}
Thus, equality holds if and only if $a = b$.

\subsection*{3.34}
If we let $\theta=\frac{1}{2}$, then 
\begin{align*}
&a^\frac{1}{2}b^\frac{1}{2} \leq \frac{1}{2} (a+b) \\
&(ab)^\frac{1}{2} \leq \frac{1}{2} (a+b) \\
&(\text{A})^\frac{1}{2} \leq \frac{1}{4} \text{P} \\
& P \geq 4 \sqrt{A}
\end{align*}
Where $A$ is the area and $P$ is the perimeter.
The minimum of the area is where $P=4 \sqrt{A}$ where
\[2(a+b)=4\sqrt{ab} \Rightarrow 4(a+b)^2 = 16(ab) \Rightarrow 4(a-b)^2=0 \Rightarrow a=b \]

\subsection*{3.35}
Using the Arithmetic Geometric Mean inequality,
\begin{align*}
    (x_1 \cdot ... \cdot x_n)^{\frac{1}{n}} \leq \frac{x_1+\dots+x_n}{n}
\end{align*}
Where the two quantities are equal iff each $x_i = x_j \quad \forall i,j$ .\\
The n-dimensional cube must have $n$ vertices, and $2^{n-1}$ edges by theorem, so the total length of all vertices is going to be $2^{n-1}(x_1+x_2+\dots+x_n)$, where each $x$ is a length of a vertex.
Volume is given by $2^n(x_1\cdot \dots \cdot x_n)^{\frac{1}{n}}$, Thus, the inequality yields
\begin{align*}
    2^{n}(x_1\cdot ... \cdot x_n)^{\frac{1}{n}} & \leq 2^{n-1} \sum_{i=1}^n x_i\\
    (x_1\cdot ... \cdot x_n)^{\frac{1}{n}} & \leq \frac{1}{n}\sum_{i=1}^n x_i\\
    \text{Area}^{\frac{1}{n}} & \leq \frac{1}{n}\sum_{i=1}^n x_i \\
\end{align*}
The minimum value is yielded when when these are equal.\\
Note, if $x_i = y$ for all i, then 
\begin{align*}
    y^n & = \frac{1}{n}\sum_{i=1}^n x_i = \frac{1}{n} (n(y^n)) = y^n
\end{align*}
Otherwise, we know that these will not be equal by our definition of the Arithmetic Geometric Mean inequality.\\
Thus, the $n$ dimensional rectangle with the least perimeter and a fixed area will be the square.

\subsection*{3.36}
Lemma: For $p > 1, a \geq 0$ we have $\|\mathbf{x}\|_p \geq \|\mathbf{x}\|_{p+a}$\\\\
Pf:\\
\begin{align*}
    \sum_{k=1}^{n} \|\mathbf{x}\|^p &= \sum_{k=1}^{n} \big( (\|\mathbf{x}\|^{p+a})^{\frac{p}{p+a}} \big) \\
    & \geq \big( \sum_{k=1}^{n}  (\|\mathbf{x}\|^{p+a}) \big)^{\frac{p}{p+a}} \\
\end{align*}
By Jensen's inequality, $\frac{p}{p+a} < 1 \implies $ convexity, implying 
\begin{align*}
    \big( \sum_{k=1}^{n} \|\mathbf{x}\|^p\big)^{\frac{1}{p}} &\geq \big( \sum_{k=1}^{n}  (\|\mathbf{x}\|^{p+a}) \big)^{\frac{1}{p+a}} \\
    \|\mathbf{x}_p \| & \geq \|\mathbf{x}\|_{p+a}
\end{align*}

Pf for 3.36:
Note, $\frac{1}{\frac{p}{r}}+\frac{1}{\frac{q}{r}} = 1$
\begin{align*}
    \big( \sum_{k=1}^n\|\mathbf{x}_k \mathbf{y}_k \|^r \big) ^{1/r} &\leq \big( \big(\sum_{k=1}^n\|\mathbf{x}_k \mathbf{y}_k \|\big)^r\big)^{\frac{1}{r}}& \\
    &  \text{ By Jensen's inequality, as $r>1$} \\
    & = \sum_{k=1}^n\|\mathbf{x}_k \mathbf{y}_k \| \quad \\\\
    &\text{         By Holder's inequality we have}\\\\
    & = \|\mathbf{x}\|_{\frac{p}{r}}\|\mathbf{y}\|_{\frac{q}{r}}
\end{align*}
Note that $\frac{p}{r} < p$ and $\frac{q}{r}<q$. Therefore, by Lemma:
\begin{align*}
    \|\mathbf{x}\|_{\frac{p}{r}}\|\mathbf{y}\|_{\frac{q}{r}} & \leq \|\mathbf{x}\|_p \|\mathbf{y}\|_q 
\end{align*}








\end{document}

