
\documentclass[letterpaper,12pt]{article}

\usepackage{threeparttable}
\usepackage{geometry}
\geometry{letterpaper,tmargin=1in,bmargin=1in,lmargin=1.25in,rmargin=1.25in}
\usepackage[format=hang,font=normalsize,labelfont=bf]{caption}
\usepackage{amsmath}
\usepackage{mathrsfs}
\usepackage{multirow}
\usepackage{array}
\usepackage{delarray}
\usepackage{listings}
\usepackage{amssymb}
\usepackage{amsthm}
\usepackage{lscape}
\usepackage{natbib}
\usepackage{setspace}
\usepackage{float,color}
\usepackage[pdftex]{graphicx}
\usepackage{pdfsync}
\usepackage{verbatim}
\usepackage{placeins}
\usepackage{geometry}
\usepackage{pdflscape}
\synctex=1
\usepackage{hyperref}
\hypersetup{colorlinks,linkcolor=red,urlcolor=blue,citecolor=red}
\usepackage{bm}


\theoremstyle{definition}
\newtheorem{theorem}{Theorem}
\newtheorem{acknowledgement}[theorem]{Acknowledgement}
\newtheorem{algorithm}[theorem]{Algorithm}
\newtheorem{axiom}[theorem]{Axiom}
\newtheorem{case}[theorem]{Case}
\newtheorem{claim}[theorem]{Claim}
\newtheorem{conclusion}[theorem]{Conclusion}
\newtheorem{condition}[theorem]{Condition}
\newtheorem{conjecture}[theorem]{Conjecture}
\newtheorem{corollary}[theorem]{Corollary}
\newtheorem{criterion}[theorem]{Criterion}
\newtheorem{definition}{Definition} % Number definitions on their own
\newtheorem{derivation}{Derivation} % Number derivations on their own
\newtheorem{example}[theorem]{Example}
\newtheorem{exercise}[theorem]{Exercise}
\newtheorem{lemma}[theorem]{Lemma}
\newtheorem{notation}[theorem]{Notation}
\newtheorem{problem}[theorem]{Problem}
\newtheorem{proposition}{Proposition} % Number propositions on their own
\newtheorem{remark}[theorem]{Remark}
\newtheorem{solution}[theorem]{Solution}
\newtheorem{summary}[theorem]{Summary}
\bibliographystyle{aer}
\newcommand\ve{\varepsilon}
\renewcommand\theenumi{\roman{enumi}}
\newcommand\norm[1]{\left\lVert#1\right\rVert}

\begin{document}

\title{Math 344 Homework 2.3}
\author{Chris Rytting}
\maketitle

\subsection*{2.14}
\[L(2e_1, -3e_2) = -4e_1 + 7e_2\]
\[L^2(2e_1, -3e_2) = 10e_1  -15e_2\]
\begin{align*}
    L =
    \begin{bmatrix}
        1 & 2 \\
        2 & -1 \\
    \end{bmatrix}
    L^2 =
    \begin{bmatrix}
        5 & 0 \\
        0 & 5 \\
    \end{bmatrix}
\end{align*}



\subsection*{2.15}
(ii)
\begin{align*}
\begin{bmatrix}
    0 & 0 & 0\\
    1 & 0 & 0\\
    0 & 1 & 0\\
    0 & 0 & 1\\
    0 & 0 & 0\\
\end{bmatrix}
\end{align*}

(iv)
\begin{align*}
\begin{bmatrix}
    0 & 0 & 0\\
    0 & -3 & 0\\
    0 & 4 & -6\\
    0 & 0 & 8\\
    0 & 0 & 0\\
\end{bmatrix}
\end{align*}
\subsection*{2.16}
For the transformation $p(x) \rightarrow p(x) + 4p'(x)$ we have the matrix representation:
\begin{align*}
    L\left( \begin{bmatrix} 1 \\ 0 \\ 1 \\ \end{bmatrix} \right) &= 
    \begin{bmatrix} 1 \\ 0 \\ 1 \\ \end{bmatrix} + 4\begin{bmatrix} 0 \\ 2 \\ 0 \\ \end{bmatrix} =
    \begin{bmatrix} 1 \\ 8 \\ 1 \\ \end{bmatrix} \\
    L\left( \begin{bmatrix} -1 \\ 1 \\ 0 \\ \end{bmatrix} \right) &= 
    \begin{bmatrix} -1 \\ 1 \\ 0 \\ \end{bmatrix} + 4\begin{bmatrix} 1 \\ 0 \\ 0 \\ \end{bmatrix} =
    \begin{bmatrix} 3 \\ 1 \\ 0 \\ \end{bmatrix} \\
    L\left( \begin{bmatrix} 1 \\ 0 \\ 0 \\ \end{bmatrix} \right) &= 
    \begin{bmatrix} 1 \\ 0 \\ 0 \\ \end{bmatrix} + 4\begin{bmatrix} 0 \\ 0 \\ 0 \\ \end{bmatrix} =
    \begin{bmatrix} 1 \\ 0 \\ 0 \\ \end{bmatrix} \\
\end{align*}
Ultimately giving:
\[
\begin{bmatrix}
    1 & 3 & 1 \\
    8 & 1 & 0 \\
    1 & 0 & 0 \\
\end{bmatrix}
\]
\subsection*{2.17}
Let the matrix representation of $S$ be given by
\begin{align*}
    S = 
    \begin{bmatrix}
        1 & 0 & 0 \\
        0 & 1 & 0 \\
        0 & 0 & 1 \\
    \end{bmatrix}
\end{align*}
\begin{align*}
    L(s_1) = L \Big(
    \begin{bmatrix}
        1\\
        0\\
        0
    \end{bmatrix}\Big)
    = \Big(
    \begin{bmatrix}
        \alpha\\
        0\\
        0
    \end{bmatrix}\Big)\\
\end{align*}

\begin{align*}
    L(s_2) = L \Big(
    \begin{bmatrix}
        0\\
        1\\
        0
    \end{bmatrix}\Big)
    = \Big(
    \begin{bmatrix}
        \alpha\\0\\0
    \end{bmatrix}\Big)\\
\end{align*}
\begin{align*}
    L(s_3) =L \Big(
    \begin{bmatrix}
        1\\ 
        \alpha\\
        0
    \end{bmatrix}\Big)
    = 
    \begin{bmatrix}
        0\\2\\ \alpha
    \end{bmatrix}\Big)\\
    \implies
    L = 
    \begin{bmatrix}
        \alpha & 1 & 0 \\
        0 & \alpha & 2 \\
        0 & 0 & \alpha \\
    \end{bmatrix}
\end{align*}

\subsection*{2.18}
\begin{align*}
    s_1 = e^{i \theta} = \text{cos} (\theta) + i \text{sin} (\alpha) = t_1 + t_2 
    \\s_2 = e^{-i \theta} = \text{cos} (\theta) - i \text{sin} (\alpha) = t_1 - t_2
    \\t_1 = \text{cos} (\theta) = \frac{1}{2}(\text{cos} (\theta) + i \text{sin} (\theta) ) + \frac{1}{2}(\text{cos} (\theta) - i \text{sin} (\theta) ) = \frac{1}{2}s_1 + \frac{1}{2}s_2 
    \\t_2 = i\text{sin} (\theta) = \frac{1}{2}(\text{cos} (\theta) + i \text{sin} (\theta) ) - \frac{1}{2}(\text{cos} (\theta) - i \text{sin} (\theta) ) = \frac{1}{2}s_1 - \frac{1}{2}s_2 
    \\
\end{align*}
    (i)\\
\begin{align*}
    \implies 
    C_{TS}=
    \begin{bmatrix}
        \frac{1}{2} & \frac{1}{2}\\
        \frac{1}{2} & -\frac{1}{2}\\
    \end{bmatrix}
\end{align*}
    \\(ii)\\
\begin{align*}
    \implies 
    C_{ST}=
    \begin{bmatrix}
        1 & 1\\
        1 & -1\\
    \end{bmatrix}
\end{align*}
    \\(iii)\\
\begin{align*}
    \\ C_{ST}^{-1} = -\frac{1}{2}
    \begin{bmatrix}
        -1 & -1 \\
        -1 & 1 \\
    \end{bmatrix}
    =
    \begin{bmatrix}
        \frac{1}{2} & \frac{1}{2}\\
        \frac{1}{2} & -\frac{1}{2}\\
    \end{bmatrix}
\end{align*}

\subsection*{2.19(i)}
For the derivative operator $D:V\rightarrow V$:\\
\\
We know that:
\begin{align*}
    &\frac{d}{d\theta}\left( e^{i\theta} \right) = ie^{i\theta} \rightarrow
    D\left( \begin{bmatrix} 1 \\ 0 \end{bmatrix} \right) = \begin{bmatrix} i \\ 0 \end{bmatrix} \\
    &\frac{d}{d\theta}\left( e^{-i\theta} \right) = -ie^{-i\theta} \rightarrow
    D\left( \begin{bmatrix} 0 \\ 1 \end{bmatrix} \right) = \begin{bmatrix} 0 \\ -i \end{bmatrix} \\
\end{align*}
So,
\[
D = \begin{bmatrix}
i & 0 \\
0 & -i \\
\end{bmatrix}
\]
\subsection*{2.19(ii)}
We know that:
\begin{align*}
    &\frac{d}{d\theta}\left( \text{cos}(\theta) \right) = -\text{sin}(\theta) \rightarrow
    D\left( \begin{bmatrix} 1 \\ 0 \end{bmatrix} \right) = \begin{bmatrix} 0 \\ i \end{bmatrix} \\
    &\frac{d}{d\theta}\left( i\text{sin}(\theta) \right) = i\text{cos}(\theta) \rightarrow
    D\left( \begin{bmatrix} 0 \\ 1 \end{bmatrix} \right) = \begin{bmatrix} i \\ 0 \end{bmatrix} \\
\end{align*}
So,
\[
D= \begin{bmatrix}
0 & i \\
i & 0 \\
\end{bmatrix}
\]
\subsection*{2.19(iii)}
We first want apply our linear transformation and then change our basis to $T$, this can be done using the transformation matrix for $D$ in terms of $S$, and then applying our transition matrix from $S$ to $T$. So if we multiply this transformation matrix and transition matrix, this will yield the desired result.
\[
\begin{bmatrix}
i & 0 \\
0 & -i \\
\end{bmatrix}
\begin{bmatrix}
0 & i \\
i & 0 \\
\end{bmatrix}
= 
\begin{bmatrix}
i & -i \\
i & i \\
\end{bmatrix}
\]

\subsection*{2.20}
We want to show that for the linear transformation $A = [a_1, a_2, \dots,a_n]$, it is true that 
$\mathscr{R}(A) = \text{span}\{a_1, a_2, \dots,a_n\}$. \\ \\
$\Rightarrow$ \\
Let us consider $b \in \mathscr{R}(A)$. Then the following is also true that 
$\exists ~ x \text{s.t.} ~ A(x) = b$. Since $x$ can be expressed $x = x_1, x_2, \dots, x_n$, 
we can also express $b$ as $b=x_1a_1+x_2a_2+\dots+x_na_n$. Thus we can see that b is represented 
as a linear combination of the columns of $A$ and therefore is also in $\mathscr{R}(A) = 
\text{span}\{a_1, a_2, \dots,a_n\}$. \\
\\
$\Leftarrow$ \\
Now let us consider $b \in \text{span}\{a_1, a_2, \dots,a_n\}$. Then it is true that 
$b$ is a linear combination of the elements of $\text{span}\{a_1, a_2, \dots,a_n\}$ 
and therefore can be represented as $b=x_1a_1+x_2a_2+\dots+x_na_n$, and we can say that $\exists ~ x 
~ \text{s.t.} ~ b=A(x)$. Showing that $b \in \mathscr{R}(A)$. \\
\\
Since every element of $\mathscr{R}(A)$ is in $\text{span}\{a_1, a_2, \dots,a_n\}$, and every element
of $\text{span}\{a_1, a_2, \dots,a_n\}$ is in $\mathscr{R}(A)$, it is true that  
$\mathscr{R}(A) = \text{span}\{a_1, a_2, \dots,a_n\}$.

















\end{document}

