\documentclass[letterpaper,12pt]{article}

\usepackage{threeparttable}
\usepackage{geometry}
\geometry{letterpaper,tmargin=1in,bmargin=1in,lmargin=1.25in,rmargin=1.25in}
\usepackage[format=hang,font=normalsize,labelfont=bf]{caption}
\usepackage{amsmath}
\usepackage{mathrsfs}
\usepackage{multirow}
\usepackage{array}
\usepackage{delarray}
\usepackage{listings}
\usepackage{amssymb}
\usepackage{amsthm}
\usepackage{lscape}
\usepackage{natbib}
\usepackage{setspace}
\usepackage{float,color}
\usepackage[pdftex]{graphicx}
\usepackage{pdfsync}
\usepackage{verbatim}
\usepackage{placeins}
\usepackage{geometry}
\usepackage{pdflscape}
\synctex=1
\usepackage{hyperref}
\hypersetup{colorlinks,linkcolor=red,urlcolor=blue,citecolor=red}
\usepackage{bm}


\theoremstyle{definition}
\newtheorem{theorem}{Theorem}
\newtheorem{acknowledgement}[theorem]{Acknowledgement}
\newtheorem{algorithm}[theorem]{Algorithm}
\newtheorem{axiom}[theorem]{Axiom}
\newtheorem{case}[theorem]{Case}
\newtheorem{claim}[theorem]{Claim}
\newtheorem{conclusion}[theorem]{Conclusion}
\newtheorem{condition}[theorem]{Condition}
\newtheorem{conjecture}[theorem]{Conjecture}
\newtheorem{corollary}[theorem]{Corollary}
\newtheorem{criterion}[theorem]{Criterion}
\newtheorem{definition}{Definition} % Number definitions on their own
\newtheorem{derivation}{Derivation} % Number derivations on their own
\newtheorem{example}[theorem]{Example}
\newtheorem{exercise}[theorem]{Exercise}
\newtheorem{lemma}[theorem]{Lemma}
\newtheorem{notation}[theorem]{Notation}
\newtheorem{problem}[theorem]{Problem}
\newtheorem{proposition}{Proposition} % Number propositions on their own
\newtheorem{remark}[theorem]{Remark}
\newtheorem{solution}[theorem]{Solution}
\newtheorem{summary}[theorem]{Summary}
\bibliographystyle{aer}
\newcommand\ve{\varepsilon}
\renewcommand\theenumi{\roman{enumi}}
\newcommand\norm[1]{\left\lVert#1\right\rVert}

\begin{document}

\title{Math 344 Homework 6.3}
\author{Chris Rytting}
\maketitle

\subsection*{6.12}
\textbf{i}
\begin{align*}
    Df(x) &= DAx \\
    &=
    D\begin{bmatrix}
        a_{11} & a_{12} & \cdots & a_{1n} \\
        a_{21} & a_{22} & \cdots & a_{2n} \\
        \vdots & \vdots & \vdots & \vdots \\
        a_{m1} & a_{m2} & \cdots & a_{mn} \\
    \end{bmatrix}
    \begin{bmatrix}
        x_1 \\x_2 \\ \vdots \\ x_n 
    \end{bmatrix}\\
    &=
    D\begin{bmatrix}
        a_{11}x_1 +  a_{12}x_2 +  \cdots + a_{1n}x_n \\
        a_{21}x_1 + a_{22}x_2 + \cdots + a_{2n}x_n \\
        \vdots \\
        a_{m1}x_1 + a_{m2}x_2 + \cdots + a_{mn}x_n \\
    \end{bmatrix}
\end{align*}
Since the Jacobian is the matrix representation of the linear map $Df(x)$, we have
   

\begin{align*}
    D\begin{bmatrix}
        a_{11}x_1 +  a_{12}x_2 +  \cdots + a_{1n}x_n \\
        a_{21}x_1 + a_{22}x_2 + \cdots + a_{2n}x_n \\
        \vdots \\
        a_{m1}x_1 + a_{m2}x_2 + \cdots + a_{mn}x_n \\
    \end{bmatrix} 
    &=
    \begin{bmatrix}
    D_1f_1(x) &D_2f_1(x) & \cdots &D_nf_1(x)  \\
    D_1f_2(x) &D_2f_2(x) & \cdots &D_nf_2(x)  \\
    \vdots & \vdots & \ddots & \vdots \\
    D_1f_m(x) &D_2f_m(x) & \cdots &D_nf_m(x)  \\
    \end{bmatrix}
    \\ &=
    \begin{bmatrix}
        a_{11} & a_{12} & \cdots & a_{1n} \\
        a_{21} & a_{22} & \cdots & a_{2n} \\
        \vdots & \vdots & \ddots & \vdots \\
        a_{m1} & a_{m2} & \cdots & a_{mn} \\
    \end{bmatrix}
    \\ &=A
\end{align*}

\textbf{ii}

\begin{align*}
    Df(x) &= Dx^TA \\
    &=D
    \begin{bmatrix}
        x_1 & x_2 & \cdots & x_n 
    \end{bmatrix}
    \begin{bmatrix}
        a_{11} & a_{12} & \cdots & a_{1n} \\
        a_{21} & a_{22} & \cdots & a_{2n} \\
        \vdots & \vdots & \ddots & \vdots \\
        a_{m1} & a_{m2} & \cdots & a_{mn} \\
    \end{bmatrix}\\
    &= 
    D\begin{bmatrix}
        a_{11}x_1 + \cdots + a_{m1}x_n &
        a_{12}x_1 + \cdots + a_{m2}x_n &
        \cdots &
        a_{1n}x_1 + \cdots + a_{mn}x_n 
    \end{bmatrix}\\
    &= 
    D\begin{bmatrix}
        f_1 &
        f_2 &
        \cdots &
        f_n &
    \end{bmatrix}
\end{align*}
by similar logic as \textbf{i}, we have that 
\begin{align*}
    D\begin{bmatrix}
        f_1 &
        f_2 &
        \cdots &
        f_n &
    \end{bmatrix}
    &= 
    \begin{bmatrix}
    D_1f_1(x) &D_2f_1(x) & \cdots &D_nf_1(x)  \\
    D_1f_2(x) &D_2f_2(x) & \cdots &D_nf_2(x)  \\
    \vdots & \vdots & \ddots & \vdots \\
    D_1f_m(x) &D_2f_m(x) & \cdots &D_nf_m(x)  \\
    \end{bmatrix}^T\\
    &= A^T
\end{align*}



\subsection*{6.13 (i)}


Let $f(x) = u(x)^Tv(x)$. It suffices to show that
\[ \lim_{h \to 0} \frac{\| u (x+h)^Tv(x+h) - u(x)v(x) - u(x)^TDv(x)h - v(x)^TDu(x)h\|}{\|h\|} = 0\]
By Proposition 6.2.17, $u,v$ are locally Lipschitz, therefore we have
\[\|u(x+h)-u(x)\| \leq L\|h\| \quad \|v(x+h)-v(x)\| \leq L\|h\|\]
implying

\[    \|u(x+h) - u(x)^T - Du(x)\| \leq \frac{\epsilon \|h\|}{3(\|v(x)\|+1)}\]
\[   \|v(x+h) - v(x)^T - Dv(x)\| \leq \frac{\epsilon \|h\|}{3(\|u(x)\|+1)}\]
$\|h\| <\delta_x$ implies that both of these derivatives exist.\\\\
Given $\epsilon > 0$, let $\delta = \text{min} \left\{ \delta_x, \frac{\epsilon}{3L\|Dv(x)\|}\right\}$\\
then $\|h\| < \delta$ implies
\begin{align*}
    & \| u (x+h)^Tv(x+h) - u(x)v(x) - u(x)^TDv(x)h - v(x)^TDu(x)h\| \\
    & \leq \|u(x+h)^T\|\|v(x+h)-v(x)-Dv(x)h\| + \|v(x)\|\|u(x+h)-u(x)^t - Du(x)h\|\\
    & \quad \quad \quad     +\|u(x+h)^T-u(x)^T\|\|Dv(x)h\|\|h\|\\
    & \leq \|u(x)^T + L\|\|v(x+h) - v(x) - Dv(x)h\| + \|v(x)\|\|u(x+h)^T-u(x)^T - Du(x)h\| \\
    & \quad \quad \quad     + \delta L\|Dv(x)\|\|h\|\\
    & < \epsilon \|h\|
\end{align*}


\subsection*{6.13 (ii)}
Given $f(x)= x^T \quad g(x) = Ax$, we have
\[D(fg) = x^TA + x^TA^T = x^T(A+A^T)\]
by (i).

\subsection*{6.13 (iii)}
Let $f(x) = Bw$. Then
\begin{align*}
    Df(x) &= B(x)Dw(x) + w^TDB(x)^T\\
    &= B(x)Dw(x) + \begin{bmatrix} w^TDb_1^T(x)\\
        \vdots \\
        w^T(x)Db_k^T(x)
    \end{bmatrix}
\end{align*}
by (i).




\subsection*{6.14}
We want to show that the inner product between the two of these is zero. Keep in mind that our field is the reals, so a hermitian is analogous to a transpose. Also, the derivative of a constant is 0. Then given the standard inner product, we have
\begin{align*}
    \int^{}_{}\overline {DF(\gamma(t))^T} \gamma'(t) dt &=  \int^{}_{}DF(\gamma(t)) \gamma'(t) dt \\
    &=  \int^{}_{}DF(\gamma(t))D\gamma(t) \gamma'(t) dt \\
    &=  \int^{}_{}DC\cdot D\gamma(t) \gamma'(t) dt \\
    &=  \int^{}_{}0\cdot D\gamma(t) \gamma'(t) dt \\
    &=  0
\end{align*}
showing orthogonality, the desired result.



\subsection*{6.15}

Note,
\begin{align*}
D(g \circ f)&=D(g(f(\textbf{x})))D(f(\textbf{x}))\\
\end{align*}
\[D(g(x)) = \begin{bmatrix} y & x\\ 2x& 2y \end{bmatrix}\]
\[D(f(x)) = \begin{bmatrix} -1 & cos(y) \\ e^x& -1 \end{bmatrix}\]
    We also know that
\[f(0,0) = (0,1)\]
yielding

\[Dg(f(0,0)) = \begin{bmatrix} 1&0\\0&2\end{bmatrix} \begin{bmatrix}-1 & 1 \\
                1 & -1 \\
\end{bmatrix}
= \begin{bmatrix}-1 & 1 \\
                2 & -2 \\
\end{bmatrix}\]










\subsection*{6.16}
We have that
\begin{align*}
    \langle f, g\rangle &= xy(\sin(y) - x) + (e^x - y)(x^2 + y^2) \\
    &= xy\sin(y) - x^2y + x^2e^x+y^2e^x-yx^2 - y^3 \\
\end{align*}
Differentiating, we have
\begin{align*}
    D(\langle f, g\rangle) & = 
    \begin{bmatrix}
        y\sin(y) - 2xy + 2xe^x + x^2e^x+ y^2e^x - 2xy \\
        x\sin(y) + xy\cos(y) - x^2 +2ye^x - x^2 - 3y^2
    \end{bmatrix}
\end{align*}
Evaluating at $(0,0)$ yields
\[\begin{bmatrix}
0\\
0
\end{bmatrix}\]

\subsection*{6.17}
By the chain rule, 
\begin{align*}
    D f( \textbf{x}_0) &= 2 \|A \textbf{x}_0 - \textbf{b} \|_2 \cdot \nabla \| A \textbf{x}_0 - \textbf{b} \|_2 \cdot A
\end{align*}



\end{document}
