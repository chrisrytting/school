\documentclass[letterpaper,12pt]{article}

\usepackage{threeparttable}
\usepackage{geometry}
\geometry{letterpaper,tmargin=1in,bmargin=1in,lmargin=1.25in,rmargin=1.25in}
\usepackage[format=hang,font=normalsize,labelfont=bf]{caption}
\usepackage{amsmath}
\usepackage{mathrsfs}
\usepackage{multirow}
\usepackage{array}
\usepackage{delarray}
\usepackage{listings}
\usepackage{amssymb}
\usepackage{amsthm}
\usepackage{lscape}
\usepackage{natbib}
\usepackage{setspace}
\usepackage{float,color}
\usepackage[pdftex]{graphicx}
\usepackage{pdfsync}
\usepackage{verbatim}
\usepackage{placeins}
\usepackage{geometry}
\usepackage{pdflscape}
\synctex=1
\usepackage{hyperref}
\hypersetup{colorlinks,linkcolor=red,urlcolor=blue,citecolor=red}
\usepackage{bm}


\theoremstyle{definition}
\newtheorem{theorem}{Theorem}
\newtheorem{acknowledgement}[theorem]{Acknowledgement}
\newtheorem{algorithm}[theorem]{Algorithm}
\newtheorem{axiom}[theorem]{Axiom}
\newtheorem{case}[theorem]{Case}
\newtheorem{claim}[theorem]{Claim}
\newtheorem{conclusion}[theorem]{Conclusion}
\newtheorem{condition}[theorem]{Condition}
\newtheorem{conjecture}[theorem]{Conjecture}
\newtheorem{corollary}[theorem]{Corollary}
\newtheorem{criterion}[theorem]{Criterion}
\newtheorem{definition}{Definition} % Number definitions on their own
\newtheorem{derivation}{Derivation} % Number derivations on their own
\newtheorem{example}[theorem]{Example}
\newtheorem{exercise}[theorem]{Exercise}
\newtheorem{lemma}[theorem]{Lemma}
\newtheorem{notation}[theorem]{Notation}
\newtheorem{problem}[theorem]{Problem}
\newtheorem{proposition}{Proposition} % Number propositions on their own
\newtheorem{remark}[theorem]{Remark}
\newtheorem{solution}[theorem]{Solution}
\newtheorem{summary}[theorem]{Summary}
\bibliographystyle{aer}
\newcommand\ve{\varepsilon}
\renewcommand\theenumi{\roman{enumi}}
\newcommand\norm[1]{\left\lVert#1\right\rVert}

\begin{document}

\title{Math 344 Homework }
\author{Chris Rytting}
\maketitle


\subsection*{6.6}
Given $f$, note that $|x| \leq (x^2 +y^2)^\frac{1}{2}$, and $|y|\leq (x^2 +y^2)^\frac{1}{2}$. Now, given $\epsilon > 0 \quad |x^2+y^2|<\delta$
\[\implies |f(x,y) - f(0,0)| = \frac{xy^2}{X^2+y^2}\leq \frac{(x^2+y^2)^{3/2}}{x^2+y^2} = (x^2+y^2)^{1/2} <\delta = \epsilon\]
Now, differentiating, we get
\[f_y = \frac{2xy(x^2+y^2) - 2xy^3}{(x^2+y^2)^2}\]
Now consider the sequence $\{x_n\}_{n=2}^\infty = (1/n, 1/n)$. Plugging this into $f_y$, we have 
\[f_y = \frac{2xy(x^2+y^2) - 2xy^3}{(x^2+y^2)^2} = \frac{\frac{4}{n^4}- \frac{2}{n^4}}{\frac{4}{n^4}} = \frac{1}{2}\]
Next, consider the sequence $\{y_n\}_{n=2}^\infty = (1/n, -1/n)$, and we have
\[= \frac{\frac{-4}{n^4}+ \frac{2}{n^4}}{\frac{4}{n^4}} = \frac{-1}{2}\]
implying that the limit is not the same coming from both sides, implying that $f$ is not differentiable at $(0,0)$, the desired result.

\subsection*{6.7}
Directional derivatives are as follows:
\[D_1f(0,0) = \lim_{h\to0} \frac{f(h,0) - 0}{h} = \frac{0}{h^2}= 0\]
\[D_2f(0,0) = \lim_{h\to0}\frac{f(0,h) - 0}{h} = \frac{0}{h}= 0\]
Therefore, the partial derivatives exist.
Converting $x$ and $y$ into polar coordinates, where 
\[x = r\cos(\theta) \quad y = r\sin(\theta)\] 
yielding 
\begin{align*}
\lim_{r\to0} \frac{r^2\cos(\theta)\sin(\theta)}{r^2\cos^2(\theta)+r\sin(\theta)}
&=\lim_{r\to0} \frac{r\cos(\theta)\sin(\theta)}{r\cos^2(\theta)+\sin(\theta)}
\\&=\lim_{r\to0} \frac{\cos(\theta)\sin(\theta)}{\cos^2(\theta)}
\\&=\lim_{r\to0}\frac{\sin(\theta)}{\cos(\theta)}\\
&\not = 0
\end{align*}
Consider the sequence $\{x_n\}_{i=n}^\infty = (1/n,1/n)$ which yields
\[\lim_{n \to 0} \frac{1}{n} = 0\]
and the derivatives do not converge to same value, so we have that $f$ is not differentiable.


\subsection*{6.8}
Directional derivatives are as follows:
\[ D_1f(x,y) = \lim_{h\rightarrow 0} \frac{0 \cdot h}{0^2 + h^2} = \lim_{h\rightarrow 0} \frac{0}{h^2} = 0 \]
\[ D_2f(x,y) = \lim_{h\rightarrow 0} \frac{h \cdot o}{h^2 + 0^2} = \lim_{h\rightarrow 0} \frac{0}{h^2} = 0 \]
Therefore, partial derivatives exist, and if $f$ is differentiable, total derivative is $0$.

Now, if $f$ is differentiable, we have the following:
\begin{align*}
    \lim_{h\rightarrow 0} \frac{\| f(0 + h) - f(0,0) - 0 \|}{\| {h} \|} &= 0\\
    &= \lim_{h\rightarrow 0} \frac{1}{\sqrt{h_x^2 + h_y^2}}\cdot \frac{\|h_x\|\|h_y\|}{\sqrt{h_x^2 + h_y^2}} \\
    &= \lim_{h\rightarrow 0} \frac{\|h_x\|\|h_y\|}{h_x^2 + h_y^2} \\
\end{align*}
Consider the sequence $\{(\frac{1}{n},\frac{1}{n})\}_{n=1}^{\infty}$. The limit $\frac{1/n^2}{2/n^2} = \frac{1}{2} \neq 0$ yielding a contradiction and we have the desired result.

\subsection*{6.9}
Given
\[ Df(0,0)({h}) = {0} \cdot {h} = {0} ~ \forall ~ {h} \]
Let ${h} = (x,y)$. Given $\varepsilon > 0$ let $\delta = \varepsilon$. Then $\| {h} - (0,0) \| = (x^2 + y^2)^\frac{1}{2} < \delta$. and we have the result.
\begin{align*}
    \lim_{h\rightarrow 0} \frac{\| f({h} + {0}) + f(0,0) + Df(0,0)({h} \| }{\| {h} \| } 
    &=  \lim_{h\rightarrow 0} \frac{|(x^2 + y^2)sin\left( \frac{1}{\sqrt{x^2+y^2}} \right)}{\sqrt{x^2 + y^2}}\\
    & \leq \frac{x^2 + y^2}{\sqrt{x^2 + y^2}} \\&< \varepsilon
\end{align*}
And we have that $f$ is differentiable at $(0,0)$.\\
\\
For $D_1f(x,y)$ and $\varepsilon_1, \varepsilon_2 > 0$,
\begin{align*}
    \frac{1}{h} &\cdot f(h + \varepsilon_1, \varepsilon_2) - f(\varepsilon_1, \varepsilon_2)\\
    &= \frac{1}{h} \cdot \left( ( (h + \varepsilon_1)^2 + \varepsilon_2^2)\cdot \sin\left(\frac{1}{\sqrt{(h + \varepsilon_1)^2 + \varepsilon_2^2}}\right) - (\varepsilon_1^2 + \varepsilon_2^2)\cdot \sin\left(\frac{1}{\sqrt{\varepsilon_1^2 + \varepsilon_2^2}}\right) \right) \\
    & \leq \frac{1}{h} \cdot \big( (h + \varepsilon_1)^2 + \varepsilon_2^2 - \varepsilon_1^2 - \varepsilon_2^2 \big) \\
    &= \frac{1}{h}(h^2 +2\varepsilon_1) = h + 2\varepsilon_1 < M_1
\end{align*}
For $D_2f(x,y)$ and $\varepsilon_1, \varepsilon_2 > 0$, 
\begin{align*}
    \frac{1}{h} &\cdot f(\varepsilon_1, h+\varepsilon_2) - f(\varepsilon_1, \varepsilon_2)\\
    &= \frac{1}{h} \cdot \left( ( \varepsilon_1^2 + (h+\varepsilon_2)^2)\cdot \sin\left(\frac{1}{\sqrt{(\varepsilon_1^2 + (h+\varepsilon_2)^2}}\right) - (\varepsilon_1^2 + \varepsilon_2^2)\cdot \sin\left(\frac{1}{\sqrt{\varepsilon_1^2 + \varepsilon_2^2}}\right) \right) \\
    & \leq \frac{1}{h} \cdot \big( \varepsilon_1^2 + (h+\varepsilon_2)^2 - \varepsilon_1^2 - \varepsilon_2^2 \big) \\
    &= \frac{1}{h}(h^2 +2\varepsilon_2) = h + 2\varepsilon_2 < M_2
\end{align*}
So the partial derivatives are bounded at $(0,0)$. Discontinuous as $f$ doesn't converge to zero for all $\varepsilon < 0$, while $f(0,0) = 0$.


\subsection*{6.10}
Given
\[f(x) = (f_1(x),...,f_n(x))\]
Since each derivative exists, given that $\epsilon / n >0$,$\exists \delta$ 
\[\implies  \lim_{n \to 0} \frac{\| f_i(x+h) - f_i(x) - Df_i(x)h\|_{y_i}}{\|h\|_x}<\epsilon_i\]
when $\|h\|_x < \delta_i$. Now we let $\delta = min\{\delta_i\}_{i=1}^n$. Note that
\[\epsilon > n\cdot \sup_i \frac{\| f_i(x+h) - f_i(x) - Df_i(x)h\|_{y_i}}{\|h\|_x}\geq \frac{\| \vec{f}(x+h) - \vec{f}(x) - D(f_1(x)h,...,Df_n(x)h)\|_{y}}{\|h\|_x} \]
Therefore the total derivative exists and is 
\[D(f_1(x)h,...,Df_n(x)h)\]
% \begin{align*}

% \langle T_0,T_0 \rangle &= \int^{1}_{-1} \frac{T_0^2}{\sqrt{1- x^2}} dx 
% \\&= \int^{1}_{-1}\frac{1}{\sqrt{1- x^2}}\frac{(2^{n-1})^2}{(2^{n-1})^2} \text{cos}(0 \cdot \text{cos}^{-1}(x))dx
% \\&= \int^{1}_{-1}\frac{1}{\sqrt{1- x^2}}\frac{(2^{n-1})^2}{(2^{n-1})^2} \text{cos}(0)dx
% \\&= \int^{1}_{-1}\frac{1}{\sqrt{1- x^2}} 1dx
% \\&= \int^{1}_{-1}\frac{1}{\sqrt{1- x^2}} dx
% \\&= \pi
% \end{align*}






\subsection*{6.11}

Note that
\begin{align*}
    \lim_{h \to \infty} \frac{\|E(t+h) - E(t) - Ae^{Ath}\|_x}{\|h\|} &= \lim_{h \to \infty} \frac{\|E(t+h) - E(t) - Ae^{Ath}\|_x}{\|h\|} \\
    &= \lim_{h \to \infty} \frac{\|e^{At}e^{Ah} - e^{At}- Ae^{Ath}\|_x}{\|h\|} \\
    &=\|e^{At}\| \lim_{h \to \infty} \frac{\|e^{Ah} - I - Ae^{Ah}\|_x}{\|h\|} \\
    &\text{Since $\|e^{At}\|$ is bounded by a constant $M_1$, we have} \\
    &\leq M_1 \lim_{h \to \infty} \frac{\|e^{Ah} - I - Ae^{Ah}\|_x}{\|h\|} \\
    &\text{and by example 5.1.19,}\\
    &\leq M_1 \lim_{h \to \infty} \frac{\sum^{\infty}_{k = 0} \frac{\|Ah\|^k}{k!}  + \|I\| + A \sum^{\infty}_{k = 0} \frac{\|h\|^k}{k!}}{\|h\|} \\
    &= M_1 \lim_{h \to \infty} \frac{\sum^{\infty}_{k = 0} \frac{\|h\|\|A\|^k}{k!}  + \|I\| + A \sum^{\infty}_{k = 0} \frac{\|h\|\|1\|^k}{k!}}{\|h\|} \\
    &= M_1 \lim_{h \to \infty} \sum^{\infty}_{k = 0} \frac{\|A\|^k}{k!}  + \|I\| + A \sum^{\infty}_{k = 0} \frac{\|1\|^k}{k!} \\
    &\text{and since each of these are bounded above,}\\&\text{say by a constant $M_2$, we have}\\
    &\leq M_1 M_2 
\end{align*}







\end{document}
