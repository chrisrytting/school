\documentclass[letterpaper,12pt]{article}

\usepackage{threeparttable}
\usepackage{geometry}
\geometry{letterpaper,tmargin=1in,bmargin=1in,lmargin=1.25in,rmargin=1.25in}
\usepackage[format=hang,font=normalsize,labelfont=bf]{caption}
\usepackage{amsmath}
\usepackage{mathrsfs}
\usepackage{multirow}
\usepackage{array}
\usepackage{delarray}
\usepackage{listings}
\usepackage{amssymb}
\usepackage{amsthm}
\usepackage{lscape}
\usepackage{natbib}
\usepackage{setspace}
\usepackage{float,color}
\usepackage[pdftex]{graphicx}
\usepackage{pdfsync}
\usepackage{verbatim}
\usepackage{placeins}
\usepackage{geometry}
\usepackage{pdflscape}
\synctex=1
\usepackage{hyperref}
\hypersetup{colorlinks,linkcolor=red,urlcolor=blue,citecolor=red}
\usepackage{bm}


\theoremstyle{definition}
\newtheorem{theorem}{Theorem}
\newtheorem{acknowledgement}[theorem]{Acknowledgement}
\newtheorem{algorithm}[theorem]{Algorithm}
\newtheorem{axiom}[theorem]{Axiom}
\newtheorem{case}[theorem]{Case}
\newtheorem{claim}[theorem]{Claim}
\newtheorem{conclusion}[theorem]{Conclusion}
\newtheorem{condition}[theorem]{Condition}
\newtheorem{conjecture}[theorem]{Conjecture}
\newtheorem{corollary}[theorem]{Corollary}
\newtheorem{criterion}[theorem]{Criterion}
\newtheorem{definition}{Definition} % Number definitions on their own
\newtheorem{derivation}{Derivation} % Number derivations on their own
\newtheorem{example}[theorem]{Example}
\newtheorem{exercise}[theorem]{Exercise}
\newtheorem{lemma}[theorem]{Lemma}
\newtheorem{notation}[theorem]{Notation}
\newtheorem{problem}[theorem]{Problem}
\newtheorem{proposition}{Proposition} % Number propositions on their own
\newtheorem{remark}[theorem]{Remark}
\newtheorem{solution}[theorem]{Solution}
\newtheorem{summary}[theorem]{Summary}
\bibliographystyle{aer}
\newcommand\ve{\varepsilon}
\renewcommand\theenumi{\roman{enumi}}
\newcommand\norm[1]{\left\lVert#1\right\rVert}

\begin{document}

\title{Math 320 Homework 6.1}
\author{Chris Rytting}
\maketitle

\subsection*{6.2}
Since $\textbf{x}^* $ is an interior point of $\mathscr{F}$, then there exists an epsilon ball $B(\textbf{x}^*, \varepsilon) $ such that $B(\textbf{x}^*, \varepsilon) \subset \mathscr{F}\subset \mathscr{F}' $.\\\\
Furthermore, since it is a local minimizer, there exists an open neighborhood $U = B(\textbf{x}^*, \delta) \in \mathbb{R}^n $ such that $f(\textbf{x}^* \leq f(\textbf{x}) \forall \textbf{x}^* \neq \textbf{x}$ where $\textbf{x} \in U \cap \mathscr{F} $.\\\\
 Then if we take the smaller of the two and let $\gamma = \text{min}(\delta,\varepsilon)$, then there exists an open neighborhood $U' = B(\textbf{x}^*,\gamma) $ such that $\mathscr{F}' \cap U' \subset \mathscr{F}\subset \mathscr{F}'$  for which $\textbf{x}^*  $ is a minimizer so we have that $\textbf{x}^* $ is a local minimizer for $\mathscr{F}'$.




\end{document}
