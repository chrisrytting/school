\documentclass[letterpaper,12pt]{article}

\usepackage{threeparttable}
\usepackage{geometry}
\geometry{letterpaper,tmargin=1in,bmargin=1in,lmargin=1.25in,rmargin=1.25in}
\usepackage[format=hang,font=normalsize,labelfont=bf]{caption}
\usepackage{amsmath}
\usepackage{mathrsfs}
\usepackage{multirow}
\usepackage{array}
\usepackage{delarray}
\usepackage{listings}
\usepackage{amssymb}
\usepackage{amsthm}
\usepackage{lscape}
\usepackage{natbib}
\usepackage{setspace}
\usepackage{float,color}
\usepackage[pdftex]{graphicx}
\usepackage{pdfsync}
\usepackage{verbatim}
\usepackage{placeins}
\usepackage{geometry}
\usepackage{pdflscape}
\synctex=1
\usepackage{hyperref}
\hypersetup{colorlinks,linkcolor=red,urlcolor=blue,citecolor=red}
\usepackage{bm}


\theoremstyle{definition}
\newtheorem{theorem}{Theorem}
\newtheorem{acknowledgement}[theorem]{Acknowledgement}
\newtheorem{algorithm}[theorem]{Algorithm}
\newtheorem{axiom}[theorem]{Axiom}
\newtheorem{case}[theorem]{Case}
\newtheorem{claim}[theorem]{Claim}
\newtheorem{conclusion}[theorem]{Conclusion}
\newtheorem{condition}[theorem]{Condition}
\newtheorem{conjecture}[theorem]{Conjecture}
\newtheorem{corollary}[theorem]{Corollary}
\newtheorem{criterion}[theorem]{Criterion}
\newtheorem{definition}{Definition} % Number definitions on their own
\newtheorem{derivation}{Derivation} % Number derivations on their own
\newtheorem{example}[theorem]{Example}
\newtheorem{exercise}[theorem]{Exercise}
\newtheorem{lemma}[theorem]{Lemma}
\newtheorem{notation}[theorem]{Notation}
\newtheorem{problem}[theorem]{Problem}
\newtheorem{proposition}{Proposition} % Number propositions on their own
\newtheorem{remark}[theorem]{Remark}
\newtheorem{solution}[theorem]{Solution}
\newtheorem{summary}[theorem]{Summary}
\bibliographystyle{aer}
\newcommand\ve{\varepsilon}
\renewcommand\theenumi{\roman{enumi}}
\newcommand\norm[1]{\left\lVert#1\right\rVert}

\begin{document}

\title{Math 344 Homework 5.6}
\author{Chris Rytting}
\maketitle

\subsection*{5.36}
Let $(x,d)$ be a metric space, suppose 
\[f: [0, \infty ) \to [0, \infty ) \quad  f(0) = 0, f'(x) > 0, f(a+b) \leq f(a) + f(b)\]
We know that this is positive semidefinite, and that $\delta(x,y) = \delta(y,x)$, therefore $f(\delta(x,y)) = f(\delta(y,x))$, and the triangle inequality holds by definition. \\\\
Now let $\varepsilon > 0$. Consider the open ball $B_\delta(x, \varepsilon)$. Note, the open ball $B_\rho(x, f(\varepsilon/2)) \subset B_\delta(x, \varepsilon) $. Consider that for $B_\rho$, $\exists \gamma \in [0,\infty)$ such that $f(\gamma) < \delta$, and the open ball $B_\delta(x,\gamma) \subset B_\rho(x, \varepsilon)$. \\\\
Thus, they are topologically equivalent.

\subsection*{5.37}
We know that this is positive as $\| f^{-1}(\textbf{y}) \| \geq 0$. Thus it fulfills positivity. \\
\\
Note that $\| a \textbf{y}  \|_f = |a|\| \textbf{y} \| _f$ and since $\exists x \in Y$ such that $f(x) = \textbf{y}$. Since isomorphisms are linear,
\[ \| a \textbf{y} \| _f = \| a f^{-1}(\textbf{y} \|_X = |a| \| f^{-1}(\textbf{y}) \|_X = |a| \| \textbf{y} \| _F    \]
and thus it fulfills scalar preservation.\\\\
Note that
\[  \| \textbf{x} + \textbf{y} \|_f = \| f^{-1}(\textbf{x}) + f^{-1}(\textbf{y}) \| \leq \| f^{-1}(\textbf{x}) \| + \| f^{-1}(\textbf{y}) \| = \| \textbf{x} \| _f + \| \textbf{y} \| _f    \]
And thus it fulfills the triangle inequality.
\subsection*{5.38}

$(\rightarrow)$ Suppose that two matrices, $\delta$ and $\rho$ are topologically equivalent.\\\\
The identity mapping $I : (x, \delta)\rightarrow (x, \rho)$ is clearly bijective.\\\\
As for continuity, since these are topologically equivalent, we have
\[B_\delta (x, \epsilon) \subset B_\rho(x, \gamma) \quad \epsilon, \gamma > 0\]

\[\implies I(B_\delta (x, \epsilon)) \subset I(B_\rho(x, \gamma))\]
This holds for $I ^{-1}$, therefore $I$ is a homeomorphism.\\
$(\leftarrow)$ Now suppose $I$ is a homeomorphism. Then it is bijective, continous, and its inverse is continuous. Thanks to continuity,
\[i(B_\delta(x, \epsilon)) = B+\rho(x, \gamma)\]
for some $\gamma$.\\
\[\implies B_\rho(x,\gamma /2) \subset B_\delta (x, \epsilon)\]
\[\implies   I(B_\rho (x, \epsilon)) = B_\delta(x, r)\]
\[\implies B_\delta(x, \gamma / 2 ) \subset B_\rho (x , \epsilon)\]
implying topological equivalence.

\subsection*{5.39}
Let 
\[f:S^1 \to \mathbb{R}\]
Now, $S^1$ can be uniquely expressed as $[0,2\pi]$, so let $f:[0,2\pi] \to \mathbb{R}$ where $f(0) = f(2\pi)$, where $f$ is continuous.\\\\
Splitting $f$ into two distinct functions by splitting the domain, letting 
\[f_1(x) = f(x)\quad x\in [0,\pi]\]
\[f_2(x) = f(x+\pi, \quad x \in [0,\pi]\] Note each is continuous and has the same beginning and end points. \\\\

Now let $g(x) =  f_1(x) - f_2(x)$. $g$ maps from $(f(0) - f(\pi), -(f(0) - f(\pi)))$. Therefore, by the IVT, 
\[\exists c \quad \text{ such that } g(c) = 0 \implies f_1(x) = f_2(x)\] 
and there exists two equivalent anti-podal points.


\subsection*{5.40}


Suppose $f: \mathbb{R} \to \mathbb{R}$ is a homeomorphism. Then there exists a homeomorphism
\[\mathbb{R} \setminus 0 \to \mathbb{R}^2 \setminus f(0)\] but this would suggest that $\mathbb{R}^2\ f(0)$ is connected. However, this is a contradiction of Theorem 5.6.25, and we have that it is not.

\subsection*{5.41}
No. Consider the sets of invertible matrices which are continuous and open such that
\[S_1 = \{A \in M_2(\mathbb{R})| \det(A) > 0\}\]
\[S_2 = \{A \in M_2(\mathbb{R})| \det(A) < 0\}\]
Note that $S_1 \bigcup S_2 = M_2(\mathbb{R})$ and they are disjoint.


\end{document}
