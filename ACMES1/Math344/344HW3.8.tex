\documentclass[letterpaper,12pt]{article}

\usepackage{threeparttable}
\usepackage{geometry}
\geometry{letterpaper,tmargin=1in,bmargin=1in,lmargin=1.25in,rmargin=1.25in}
\usepackage[format=hang,font=normalsize,labelfont=bf]{caption}
\usepackage{amsmath}
\usepackage{mathrsfs}
\usepackage{multirow}
\usepackage{array}
\usepackage{delarray}
\usepackage{listings}
\usepackage{amssymb}
\usepackage{amsthm}
\usepackage{lscape}
\usepackage{natbib}
\usepackage{setspace}
\usepackage{float,color}
\usepackage[pdftex]{graphicx}
\usepackage{pdfsync}
\usepackage{verbatim}
\usepackage{placeins}
\usepackage{geometry}
\usepackage{pdflscape}
\synctex=1
\usepackage{hyperref}
\hypersetup{colorlinks,linkcolor=red,urlcolor=blue,citecolor=red}
\usepackage{bm}


\theoremstyle{definition}
\newtheorem{theorem}{Theorem}
\newtheorem{acknowledgement}[theorem]{Acknowledgement}
\newtheorem{algorithm}[theorem]{Algorithm}
\newtheorem{axiom}[theorem]{Axiom}
\newtheorem{case}[theorem]{Case}
\newtheorem{claim}[theorem]{Claim}
\newtheorem{conclusion}[theorem]{Conclusion}
\newtheorem{condition}[theorem]{Condition}
\newtheorem{conjecture}[theorem]{Conjecture}
\newtheorem{corollary}[theorem]{Corollary}
\newtheorem{criterion}[theorem]{Criterion}
\newtheorem{definition}{Definition} % Number definitions on their own
\newtheorem{derivation}{Derivation} % Number derivations on their own
\newtheorem{example}[theorem]{Example}
\newtheorem{exercise}[theorem]{Exercise}
\newtheorem{lemma}[theorem]{Lemma}
\newtheorem{notation}[theorem]{Notation}
\newtheorem{problem}[theorem]{Problem}
\newtheorem{proposition}{Proposition} % Number propositions on their own
\newtheorem{remark}[theorem]{Remark}
\newtheorem{solution}[theorem]{Solution}
\newtheorem{summary}[theorem]{Summary}
\bibliographystyle{aer}
\newcommand\ve{\varepsilon}
\renewcommand\theenumi{\roman{enumi}}
\newcommand\norm[1]{\left\lVert#1\right\rVert}

\begin{document}

\title{Math 344 Homework 3.8}
\author{Chris Rytting}
\maketitle

\subsection*{3.44 (i)}


We know that $Ax \in \mathscr{R}(A)$ by definition. Moreover, we know that 
\[ A^HAx = 0 \implies Ax \in \mathscr{N}(A^H)\]

\subsection*{3.44 (ii)}
We want to show that $x \in \mathscr{N}(A^HA)$ iff $\mathscr{N}(A)$.\\\\
$(\rightarrow) \quad$ We know that $Ax \in \mathscr{R}(A)$ and $Ax \in \mathscr{N}(A^H)$. Since 
$\mathscr{R}(A) \cap \mathscr{R}(A) = 0$ by FST(Fundamental Subspaces Theorem), we know that $Ax=0$ since it is in both.
\\\\
$(\leftarrow)$\\\\
\[Ax = 0 \implies A^HAx = 0 \]
Which gives us the desired result.

\subsection*{3.44 (iii)}
First, $\mathscr{N}(A^HA) = \mathscr{N}(A)$, implying that their dimensions are equal to some constant $a$. 
By rank nullity Theorem, we know that 
\[\text{dim}(\mathscr{R}(A^HA)) + \text{dim}(\mathscr{N}(A^HA)) = n \]
and that
\[\text{dim}(\mathscr{R}(A)) + \text{dim}(\mathscr{N}(A)) = n \]
which yields
\[\text{dim}(\mathscr{R}(A^HA)) + a = n \]
\[\text{dim}(\mathscr{R}(A)) + a = n \]
\[\implies  \text{dim}(\mathscr{R}(A^HA)) = n - a = 
\text{dim}(\mathscr{R}(A))  \]
\[\implies  \text{dim}(\mathscr{R}(A^HA)) = 
\text{dim}(\mathscr{R}(A))  \]
Which is the desired result.

\subsection*{3.44 (iv)}
We know that $A$ being linearly independent implies that $\text{rank}(A) = n$. Since, by (iii), 
\[ n = \text{rank}(A) = \text{rank}(A^HA)\]
and $A^HA$ is $n\times n$, it must have rank $n$.

\subsection*{3.45 (i)}
\[P^2 = A(A^HA)^{-1}A^HA(A^HA)^{-1}A^H\]
\[ = A(A^HA)^{-1}IA^H = P\]

\subsection*{3.45 (ii)}
\[P^H = (A(A^HA)^{-1}A^H)^H = (A((A^HA)^{-1})^HA^H)^H = (A((A^HA)^{H})^{-1}A^H) = (A(A^HA)^{-1}A^H) = P\]

\subsection*{3.45 (iii)}
TODO \\\\
\\\\
\\\\
\\\\
\\\\
\\\\
\\\\
TODO   \\\\\\\\
\\\\
\\\\
\\\\
\\\\
\\\\
\\\\
\\\\



\subsection*{3.46}
Because $Q$ is orthonormal, we know that $Q^{-1} = Q^H$, and we have that
\begin{align*}
(QR)^HQR\hat x &= (QR)^Hb\\
R^HQ^HQR\hat x &= (QR)^Hb\\
R^HR\hat x &= R^HQ^Hb\\
\end{align*}
Which holds iff $R \hat x = Q^Hb$.

\subsection*{3.47}
\begin{align*}
    \begin{bmatrix}
        A & I \\
        0 & A^H
    \end{bmatrix}
    \begin{bmatrix}
        \hat x\\
        r
    \end{bmatrix}
    &=
    \begin{bmatrix}
        b \\ 0
    \end{bmatrix}\\
    \begin{bmatrix}
        A \hat x + r \\
        A^Hr
    \end{bmatrix}
    &=
    \begin{bmatrix}
        b\\0
    \end{bmatrix}\\
    \begin{bmatrix}
        \text{proj}{\mathscr{R}(A)}(b) + b - \text{proj}{\mathscr{R}(A)}(b) \\
        A^Hr
    \end{bmatrix}
    &=
    \begin{bmatrix}
        b\\0
    \end{bmatrix}\\
    \begin{bmatrix}
        b\\
        A^Hr
    \end{bmatrix}
    &=
    \begin{bmatrix}
        b\\0
    \end{bmatrix}
\end{align*}
Now, it is clear that $b=b$. As for the second equation, that $A^Hr = 0$, we know that $r \in \mathscr{R}(A)^\bot = \mathscr{N}(A^H)$
, meaning that $A^Hr = 0$, as desired.

\subsection*{3.48}
\[A^HAx = 
\begin{bmatrix}
    \overline x_1^2 & \overline x_2^2 & \cdots & \overline x_n^2 \\ 
    \overline y_1^2 & \overline y_2^2 & \cdots & \overline y_n^2 \\ 
\end{bmatrix}
\begin{bmatrix}
    x_1^2 & y_1^2 \\ 
    x_2^2 & y_2^2 \\ 
    \vdots &\vdots\\
    x_n^2 & y_n^2
\end{bmatrix}
\begin{bmatrix}
    r\\s
\end{bmatrix}
=
\begin{bmatrix}
    \overline x_1^2 & \overline x_2^2 & \cdots & \overline x_n^2 \\ 
    \overline y_1^2 & \overline y_2^2 & \cdots & \overline y_n^2 \\ 
\end{bmatrix}
\begin{bmatrix}
    1\\1\\\vdots\\1
\end{bmatrix}
 = A^Hx
\]
\subsection*{3.49 (i)}
Note that 
\[P(\alpha A  + \beta B) =\frac{\left( \alpha A  + \beta B \right) + \left( \alpha A  + \beta B \right)^2 }{2} = \frac{\alpha (A + (A)^T)}{2} + \frac{\beta (A + (A)^T)}{2} = \alpha P(A) + \beta P(B)  \]
Therefore, $P(A)$ is linear.

\subsection*{3.49 (ii)}
 \[P(A)^{2}=P(P(A))=\frac{\frac{A+A^{T}}{2}+\frac{(A+A^{T})^{T}}{2}}{2} 
=\frac{\frac{A+A^{T}}{2}+\frac{A+A^{T}}{2}}{2}=\frac{A+A^{T}}{2}=P(A) \] 

\subsection*{3.49 (iii)}
By 3.41, we know that $A^* = A^H$ under the Frobenius inner product, and since we are in the reals, $A^* = A^H = A^T$. It suffices to show, then, that $P^T = P$. Well,
\[P(A)^{T} = \frac{(A+A^{T})^{T} } {2 } = \frac{(A^{T}+(A^{T})^{T}) } {2 } = \frac{A+A^{T}}{2} = P(A) \] 

\subsection*{3.49 (iv)}
Assume $A \in \text{Skew}_n(\mathbb{R})$.
Then $A^T = -A$, and we have that
\[P(A) = \frac{A^T + A}{2} = 0\]
Then $A$ is in the null space of $A$.
\\\\
Now assume that $A$ is in the null space of $P$. Then
\[P(A) = \frac{A^T + A}{2} = 0 \implies A^T = -A\]
And we have containment both in both directions, implying the equality at hand.

\subsection*{3.49 (v)}
Assume $A \in \text{Sym}_n(\mathbb{R})$. Then $A^T = A$, and we have that 
\[P(A) = \frac{A^T + A}{2} = A\] which is in the range space of $P$ since $P(A) = A$. \\\\
Now assume that $A$ is in the range space of $P$. Then $A$ is given by
\[P(B) = \frac{B^T + B}{2} = A\]
Now, we know that $A_{ij} = \frac{B_{ij} + B_{ji}}{2} = A_{ji}$, implying that $A$ is symmetric, which is the desired result.

\subsection*{3.49 (vi)}
\begin{align*}
    ||A - P(A)||_F &= ||\frac{2A - A - A^T}{2}||_F\\ &= ||\frac{A - A^T}{2}||_F\\ 
    &= \sqrt{\text{tr} \left( \frac{(A^T-A)}{2}\frac{(A-A^T)}{2} \right)}\\
    &= \sqrt{ \left( \frac{(\text{tr}(A^TA) - \text{tr}(A^2) - \text{tr}((A^T)^2) + \text{tr}(AA^T))}{4} \right)}\\
    &= \sqrt{ \left( \frac{(\text{tr}(A^TA) - \text{tr}(A^2) - \text{tr}(A^2) + \text{tr}(A^TA))}{4} \right)}\\
    &= \sqrt{ \left( \frac{(2\text{tr}(A^TA) - 2\text{tr}(A^2)}{4} \right)}\\
    &= \sqrt{ \left( \frac{(\text{tr}(A^TA) - \text{tr}(A^2)}{2} \right)}\\
\end{align*}





\end{document}
