\documentclass[letterpaper,12pt]{article}

\usepackage{threeparttable}
\usepackage{geometry}
\geometry{letterpaper,tmargin=1in,bmargin=1in,lmargin=1.25in,rmargin=1.25in}
\usepackage[format=hang,font=normalsize,labelfont=bf]{caption}
\usepackage{amsmath}
\usepackage{mathrsfs}
\usepackage{multirow}
\usepackage{array}
\usepackage{delarray}
\usepackage{listings}
\usepackage{amssymb}
\usepackage{amsthm}
\usepackage{lscape}
\usepackage{natbib}
\usepackage{setspace}
\usepackage{float,color}
\usepackage[pdftex]{graphicx}
\usepackage{pdfsync}
\usepackage{verbatim}
\usepackage{placeins}
\usepackage{geometry}
\usepackage{pdflscape}
\synctex=1
\usepackage{hyperref}
\hypersetup{colorlinks,linkcolor=red,urlcolor=blue,citecolor=red}
\usepackage{bm}


\theoremstyle{definition}
\newtheorem{theorem}{Theorem}
\newtheorem{acknowledgement}[theorem]{Acknowledgement}
\newtheorem{algorithm}[theorem]{Algorithm}
\newtheorem{axiom}[theorem]{Axiom}
\newtheorem{case}[theorem]{Case}
\newtheorem{claim}[theorem]{Claim}
\newtheorem{conclusion}[theorem]{Conclusion}
\newtheorem{condition}[theorem]{Condition}
\newtheorem{conjecture}[theorem]{Conjecture}
\newtheorem{corollary}[theorem]{Corollary}
\newtheorem{criterion}[theorem]{Criterion}
\newtheorem{definition}{Definition} % Number definitions on their own
\newtheorem{derivation}{Derivation} % Number derivations on their own
\newtheorem{example}[theorem]{Example}
\newtheorem{exercise}[theorem]{Exercise}
\newtheorem{lemma}[theorem]{Lemma}
\newtheorem{notation}[theorem]{Notation}
\newtheorem{problem}[theorem]{Problem}
\newtheorem{proposition}{Proposition} % Number propositions on their own
\newtheorem{remark}[theorem]{Remark}
\newtheorem{solution}[theorem]{Solution}
\newtheorem{summary}[theorem]{Summary}
\bibliographystyle{aer}
\newcommand\ve{\varepsilon}
\renewcommand\theenumi{\roman{enumi}}
\newcommand\norm[1]{\left\lVert#1\right\rVert}

\begin{document}

\title{Math 344 Homework 6.4}
\author{Chris Rytting}
\maketitle

\subsection*{6.19}
We have that 
\begin{align*}
    \|f(\textbf{y} ) - f(\textbf{x} )\|_Y &= \|\int^{1}_{0}Df(t \textbf{y}  + (1-t)\textbf{x} ))(\textbf{y} -\textbf{x} ) dt\|_Y\\
    &\leq \| \int_0^1 Df\big(t \textbf{y} + (1- t) \textbf{x} \big) ( \textbf{y} - \textbf{x}) dt\|_Y \\
    &\leq \int_0^1 \|  Df\big(t \textbf{y} + (1- t) \textbf{x} \big) ( \textbf{y} - \textbf{x}) dt\|_X \\
    &\leq \int_0^1 \|  Df\big(t \textbf{y} + (1- t) \textbf{x} \big) \|_X,Y \| ( \textbf{y} - \textbf{x})dt \|_X \\
    &\leq \int_0^1 \sup_{ \textbf{c} \in \ell( \textbf{x}, \textbf{y})} \|  Df\big( \textbf{c} \big) \|_{X,Y} \| ( \textbf{y} - \textbf{x}) dt\|_X \\
    &\leq \sup_{ \textbf{c} \in \ell( \textbf{x}, \textbf{y})} \|  Df\big( \textbf{c} \big) \|_{X,Y} \| ( \textbf{y} - \textbf{x}) \|_X \\
\end{align*}


\subsection*{6.20}
Consider
\[F(t) =  \int_{g(c)}^{t} f(\tau) d\tau\] By FTC, $F'(t) = f(t)$. 
By this and by chain rule, 
\[\int_c^d f(g(s))g'(s)ds = \int_c^d F'(g(s))g'(s)ds= \int_{g(c)}^{g(d)} DF(g(s))ds\]
Now by FTC
\[= F(g(d)) - F(g(c)) = \int_{g(c)}^{g(d)} f(\tau)d\tau - \int_{g(c)}^{g(c)} f(\tau)d\tau = \int_{g(c)}^{g(d)} f(\tau)d \tau\]
which is the desired result.

\subsection*{6.21}
If we know that a sequence $(f_n)_{n=0}^\infty \in C(U;Y)$ is Cauchy in $C(U;Y)$, then the restriction $(f_n|_K)_{n=0}^\infty \in (C(K;Y), \|\cdot \|_{L^\infty})$ is Cauchy for every compact subset $K \subset U$. We also know that $(f_n)_{n=0}^\infty \in C(U;Y)$ is uniformly convergent, meaning that $(f_n|_K)_{n=0}^\infty$ 
converges to $f|_K \text{ in }(C(K;Y), \|\cdot \|_{L^\infty})$ for every compact subset $K \subset U$.
Now, if this is true for an arbitrary sequence $f_n \in C(U;Y)$, then it will be true for all sequences in an open set contained in the closed set, which we know it is because the closure of a set contains the interior and exterior points of the set.



\subsection*{6.22 (i)}
As derivative exists and is given by
        \[ f'(x) = \frac{x}{\sqrt{\frac{1}{n^2} + x^2} }\]
        for any $x \in [-1,1]$.

\subsection*{6.22 (ii)}
We have that \[\sup_{(0,1)} f_n(x) = \sqrt{\frac{n^2 + 1}{n^2}}\] Furthermore, any compact set lies in the interval 
        $[a,b]$ where $0<a<b<\sqrt{\frac{n^2+1}{n^2}}$\\
        Therefore, we have that \[\|f_n(x)|_{[a,b]}\|_{L^\infty} = \sqrt{\frac{n^2+1}{n^2}} \to |x| \text{ as }n \to \infty\]
        So $f_n(x)$ converges uniformly to $|x|$ on $[-1,1]$.

\subsection*{6.22 (iii)}
Note $f(x) = |x|$. Then 
        \[f'(x) = \begin{cases} -1 & \text{if}~x < 0 \\ 1 & \text{if}~x>0\\\end{cases}\]
        Which is not discontinuous at $x=0$, implying that it is not differentiable there.

\subsection*{6.22 (iv)}
The criterion that $f_n(\mathbf{x}_*))_{n=0}^\infty \subset C^1(U;Y)$ does not 
        converge in $Y$ is not fulfilled, and we have that the theorem holds.

\subsection*{6.23}
Let 
\[S_k = \sum^{k}_{n=0}  Df_n = D \sum^{k}_{n=0} f_n \]
so $\{s_k\}^\infty_{k=0}$ converges ass. on $U$. Now
\[t_k = \sum^{k}_{n=0} f_n(x_0)\]
so $\{t_k\}_{k=0}^\infty$ converges on $Y$.





\end{document}
