\documentclass{article}
\usepackage{threeparttable}
\usepackage{geometry}
\geometry{letterpaper,tmargin=0.25in,bmargin=0.25in,lmargin=0.25in,rmargin=0.25in}
\usepackage[format=hang,font=normalsize,labelfont=bf]{caption}
\usepackage{amsmath}
\usepackage{anyfontsize}
\usepackage{mathrsfs}
\usepackage{multirow}
\usepackage{array}
\usepackage{delarray}
\usepackage{listings}
\usepackage{amssymb}
\usepackage{amsthm}
\usepackage{lscape}
\usepackage{natbib}
\usepackage{setspace}
\usepackage{float,color}
\usepackage[pdftex]{graphicx}
\usepackage{pdfsync}
\usepackage{verbatim}
\usepackage{placeins}
\usepackage{geometry}
\usepackage{pdflscape}
\synctex=1
\usepackage{hyperref}
\hypersetup{colorlinks,linkcolor=red,urlcolor=blue,citecolor=red}
\usepackage{bm}


\theoremstyle{definition}
\newtheorem{theorem}{Theorem}
\newtheorem{acknowledgement}[theorem]{Acknowledgement}
\newtheorem{algorithm}[theorem]{Algorithm}
\newtheorem{axiom}[theorem]{Axiom}
\newtheorem{case}[theorem]{Case}
\newtheorem{claim}[theorem]{Claim}
\newtheorem{conclusion}[theorem]{Conclusion}
\newtheorem{condition}[theorem]{Condition}
\newtheorem{conjecture}[theorem]{Conjecture}
\newtheorem{corollary}[theorem]{Corollary}
\newtheorem{criterion}[theorem]{Criterion}
\newtheorem{definition}{Definition} % Number definitions on their own
\newtheorem{derivation}{Derivation} % Number derivations on their own
\newtheorem{example}[theorem]{Example}
\newtheorem{exercise}[theorem]{Exercise}
\newtheorem{lemma}[theorem]{Lemma}
\newtheorem{notation}[theorem]{Notation}
\newtheorem{problem}[theorem]{Problem}
\newtheorem{proposition}{Proposition} % Number propositions on their own
\newtheorem{remark}[theorem]{Remark}
\newtheorem{solution}[theorem]{Solution}
\newtheorem{summary}[theorem]{Summary}
\bibliographystyle{aer}
\newcommand\ve{\varepsilon}
\renewcommand\theenumi{\roman{enumi}}
\newcommand\norm[1]{\left\lVert#1\right\rVert}

\begin{document}




{\fontsize{4}{5} \selectfont
$\mathbf{DEFVecSp}:1.x+y=y+x2.(x+y)+z = x+(y+z)3.Add.Id.0\in V | 0+x = x4. \exists Add.Inv. (-x) | x+(-x) = 0(5.) F.Dis.Law a(x+y) = ax + ay(6.) S.Dis.Law (a+b)(x) = ax + bx(7.)Mul.Id.1x=x(8.) (ab)x = a(bx)$
$\mathbf{THM1.1.13}$ If $W$ is a subset of a vector space $V$ s.t. $\mathbf{x,y} \in @$ and for any $a,b \in \mathbb{F} $ the vector $a \mathbf{x} + b \mathbf{y}  \in W$, then $W$ is a subspace of $V$.
$\mathbf{DEFLinHull}$ of $S \langle S \rangle$,  smallest subspace of $V$ that contains $S$,i.e. intersection of all subspaces of $V$ that contain $S$.
$\mathbf{THM1.2.6Span}(S)$ = $\langle S \rangle$.
$\mathbf{DEF}\bigoplus$ Where $W_1, W_2$ are subspaces of $V$, then $W_1 + W_2 = W_1 \bigoplus W_2 $ if $W_1 \cap W_2 = {0}.$
$\mathbf{DEF Complementary subspaces} W_1$ and $W_2$ if $V = W_1 \bigoplus W_2$
$\mathbf{THM Replacement}$: $V$ is a vector space spanned by $S = {s_1,\cdots,s_m}$. If $T = {t_1,\cdots,t_n}$ is a L.I. subset of $V$, then $\leq m$ and $\exists S' \subset S$ having $m-n$ elements such that $T\cup S'$ spans $V$.
$\mathbf{THM Extension}$: $W$ is a subspace of $V$If $T = {t_1,\cdots,t_n}$ and $S = {s_1,\cdots,s_m}$ span $W$ and $V$, respectivley, then $\exists S' \subset S$ having $m-n$ elements such that $T \cup S'$ is a basis for $V$.
$\mathbf{DEF Quotient Spaces}$: $W$ subspace of $V$. The set ${x+W | x \in V}(or equivalently [[ x ]] | x \in V)$ of all cosets of $W$ in $V$ is denoted $V/W$ and is called the quotient of $V$ modulo $W$.
$\mathbf{DEF} \boxplus \boxdot$:Let $W$ be a subspace of $V$. Define operations $\boxplus: V/W \times V/W \rightarrow V/W$ and $\boxdot : \mathbb{F} \times V/W \rightarrow V/W$ given by (i) $(x+W) \boxplus (y+W) = (x+y) + W$ and $a \boxdot(x+W) = (ax) + W$. These are the operations of vector addition and scalar multiplication on $V/W$.
$\mathbf{   CHAP2   } $
$\mathbf{DEF Linear transformation}$ Let $V$ and $V$ be vector spaces over $\mathbb{F} $. A map $L: V \rightarrow W$ is a linear transformation from $V$ into $W$ if $L(ax_1 + bx_2) = aL(x_1) + bL(x_2)$ for $x_1,x_2 \in V$ and $a,b \in \mathbb{F}$
$\mathbf{COR 2.1.17}$ A linear transforamtion is invertible if and only if it is bijective.
$\mathbf{Prop. 2.1.24: }$ If $V  \cong W$ are isomorphic vector spaces, with isoorphism $L:V \rightarrow W$, then: 
(i) A linear equation holds in V iff it also holds in W: that is $\sum_{i=1} ^\mathscr{l} a_i \mathbf{x_i} = \mathbf{0}$ holds in V iff $\sum_{i=1} ^\mathscr{l} a_i L_i \mathbf{x_i} = \mathbf{0}$ holds in W.
(ii) A set B = $\{\mathbf{v_i} ,\dots, \mathbf{ v_n}\}$ is a basis of V iff LB = $\{L\mathbf{v_i} ,\dots,L \mathbf{ v_n}\}$ is a basis for W. Moreover, the dimension of V is equal to the dimension of V.
(iii) The subspaces of V are in vijective correspondence with the subspaces of W.
(iv) If K: W $\rightarrow$ U is any linear transformation, then the composition KL:V$\rightarrow$ U is also a linear transformation and we have $\mathscr{N}(KL) = L^{-1} \mathscr{N} (K) = \{v | L(\mathbf{v}) \in \mathscr{N} (K)\}$ and $\mathscr{R} (KL) = \mathscr{R} (K)$
$\mathbf{THM F.Iso.} $ If $V$ and $X$ are vector spaces and $L: V\rightarrow X$ is a linear transformation, then $V/\mathscr{N} (L) \cong \mathscr{R} (L)$. in particular, if $L$ is surjective, then $V/N(L) \cong X$.
$\mathbf{THM2.2.7}$ If $V$ is a finite-dimensional vector space and $W$ is a subspace of $V$, then $\text{dim} (V) = \text{dim} (W) + \text{dim} (V/W) $
$\mathbf{THM Rank-Nullity}$ Let $V$ and $W$ be finite-dimensional vector spaces. If $L:V \rightarrow $ is a linear transformation then $\text{dim} (V) = \text{dim} \mathscr{R} (L) + \text{dim} \mathscr{N} (L) = \text{rank} (L) + \text{nullity} (L)$.
$\mathbf{COR Sec. Iso. Thm.}$ Assume $V_1$ and $V_2$ are subspaces of $V$. Then $V_1/(V_1 \cap V_2) \cong (V_1 + V_2)/V_2$.
$\mathbf{COR Dim. Formula}$ If $V_1$ and $V_2$ are finite-dimensional subspaces of a vector space $V$, then $\text{dim} (V_1) + \text{dim} (V_2) = \text{dim} (V_1 \cap V_2) + \text{dim} (V_1 + V_2)$
$\mathbf{DEF Similar Matrices} $ Two square matrices $A,B \in M_n(\mathbb{F} )$ are said to be similar if there exists a nonsingular $P \in M_n(\mathbb{F} )$ such that $B = P^-1AP$.
$\mathbf{DEF Bernstein Polynomials} $ Given $n \in \mathbb{N}_{\geq 0}J$, the Bernstein polynomials ${B_j^n(x)}_{j=0}^n$ of degree $n$ are defined as $B_j^n(x) = \binom nj x^j(1-x)^{n-j}$, where $\binom nj = \frac{n!}{j!(n-j)!}$
$\mathbf{LEM 2.5.3}$For $j = 0,1,\dots,n ~ ~ B_j^n(x) = \sum^{n}_{i=j}(-1)^{i-j} \binom ni \binom ij x^i $
$\mathbf{THM2.5.4}$ For any $n \in \mathbb{N}$, the set $T_n$ of degree $n$ Bernstein polynomials $T_n = {B_j^n (x) }_{j=0}^n$ forms a basis for $\mathbb{F}[x]^n $
$\mathbf{DEF Trace}$ The trace is the sum of the elements along the main diagonal
$\mathbf{PROP2.6.2}$All of the elementary matrices are invertible.
$\mathbf{DEF Row Equivalence}$The $B$ is said to be row equivalent to the matrix $A$ if there exists a finite collection of elementary matrices $E_1,E_2,\dots, E_n$ such that $B = E_1E_2\dots E_n$
$\mathbf{DEF REF}$ $A$ is REF if (i) leading coefficient of each row is strictly to the right of the previous row's leading coefficient (ii) All nonzero rows are above any zero rows and $ \mathbf{RREF} $ if (iii) the leading coefficient of every row is 1 (iv) The leading coefficient of every row is the only nonzero entry in its column.
$\mathbf{DEF Permutation}$ Different arrangements of a set. Even if it has an even number of inversions, odd if an odd number of inversions. Sign is 1 if even, -1 if odd.
$\mathbf{DEF Inversion}$ A pair $(\sigma(i), \sigma(j))$ such that $i<j$ and $\sigma(i) > \sigma(j)$.
$\mathbf{THM2.8.7}$ If $A, B \in M_n(\mathbb{F})$, then $\text{det}(AB) = \text{det}(A)\text{det}(B)$
$\mathbf{COR2.8.8} \text{det}(A^{-1} = (\text{det}(A))^{-1})$
$\mathbf{Cramer's Rule}$ If $A \in M_n(\mathbb{F})$ is nonsingular, then the unique solution to $Ax=b$ is $x = A^{-1}b = \frac{\text{adj}(A)}{\text{det}(A)}b$. Moreover, if $A_i(b) \in M_n(\mathbb{F})$ is the matrix $A$ with the i-th column replaced by b, then the i-th coordinate of $x$ is $x_i = \frac{\text{det}(A_i(b))}{\text{det}(A)}$
$\mathbf{Exam 2}$ $DEF \mathbf{innerproduct}$ on $V$ for $\mathbf{x,y,z} \in V$, $a,b \in \mathbb{F}: (i) \langle x,x \rangle \geq 0, eq. iff \mathbf{x} =0 $(ii) $\langle x,a \mathbf{y} + b \mathbf{z}  \rangle = a \langle x,y \rangle + b \langle x,z \rangle$(iii)$ \langle x,y \rangle = \overline { \langle y,x \rangle }$PROP:(i)$\langle \mathbf{x + y}, \mathbf{z} \rangle = \langle \mathbf{x}, \mathbf{z} \rangle + \langle \mathbf{y}, \mathbf{z} \rangle $(ii)$ \langle a\mathbf{x}, \mathbf{y} \rangle = \bar a \langle \mathbf{x}, \mathbf{y} \rangle $FunFact: $0 \leq \langle x - \lambda y, x - \lambda y \rangle = \langle x,x\rangle - \langle \lambda y, x \rangle - \langle x, \lambda y \rangle + \langle \lambda y, \lambda y \rangle = \langle x,x \rangle - \bar \lambda \langle y,x\rangle - \lambda \langle x,y\rangle + \lambda \bar \lambda \langle y,y\rangle$Frobenius inner product: $\langle A, B \rangle = \text{tr}(A^HB)$Orthogonal if $\langle \mathbf{y}, \mathbf{x} \rangle = 0$.Cauchy-Schwarz: $|\langle \mathbf{x}, \mathbf{y} \rangle | \leq \|\mathbf{x}\| \|\mathbf{y}\|$PROOF:suppose $\mathbf{u}, \mathbf{v} \neq 0$, choose$\lambda = \frac{|\langle \mathbf{v}, \mathbf{u} \rangle |}{\langle \mathbf{v}, \mathbf{u} \rangle }$, and $|\lambda| = 1$,Thus, $0 \leq \| \frac{\lambda \mathbf{u}}{\|u\|} - \frac{\mathbf{v}}{\|\mathbf{v}\|}\|^2 = $$|\lambda|^2 -2 \Re(\langle \mathbf{\frac{\lambda \mathbf{v}}{\mathbf{v}}},\frac{\mathbf{u}}{\|\mathbf{u}\|} \rangle ) + 1 = 2 - 2 \frac{\lambda \langle \mathbf{v}, \mathbf{u} \rangle }{\|\mathbf{v}\mathbf{u}\|} \implies |\langle \mathbf{u}, \mathbf{v} \rangle = \lambda \| \langle \mathbf{v}, \mathbf{u} \rangle \| \leq \|\mathbf{u}\| \| \mathbf{v}\|$ \textbf{PR. 3.1.23}: If $\textbf{u}$ is unit, $\text{proj}_{\textbf{u}}: V\rightarrow V$ is a linear operator and: (i) $\text{proj}_{\textbf{u}} \circ \text{proj}_{\textbf{u}} =\text{proj}_{\textbf{u}} $ (ii)$\textbf{r}=\textbf{v}-\text{proj}_{\textbf{u}}(\textbf{v})$ is orthogonal to $\text{span}\{\textbf{u}\}$ (iii) $\text{proj}_{\textbf{u}}(\textbf{v})$ unique vector in $\text{span}\{\textbf{u}\}$ nearest $\textbf{v}$. Angle: $ \text{cos}(\theta) = \frac{\langle \mathbf{x}, \mathbf{y} \rangle }{\| \mathbf{x}\| \|\mathbf{y}\|}$ Pyth: if x,y are orthogonal $\|\mathbf{x}+ \mathbf{y}\|^2 = \|\mathbf{x}\|^2 + \|\mathbf{y}\|^2$\textbf{TH. 3.2.3}: If $(V, \langle .,. \rangle)$ and $\{ \textbf{x}_i \}_{i=1}^m$ is a finite orthonormal set: (i) If $x = \sum_{i=1}^m a_i\textbf{x}_i$, then $\langle \textbf{x}_i, \textbf{x} \rangle = a_i$ for all $i=1,2,\dots,m$ (ii) If $x = \sum_{i=1}^m a_i\textbf{x}_i$ and $y = \sum_{i=1}^m b_i\textbf{y}_i$, then $\langle \textbf{x}, \textbf{y} \rangle = \sum_{i=1}^m \overline{a}_ib_i$ (iii) If $x = \sum_{i=1}^m a_i\textbf{x}_i$, then $\|x\|^2 = \sum_{i=1}^m |a_i|^2$. 
linear map is orthonormal if for every x,y $\langle \mathbf{x}, \mathbf{y} \rangle _V = \langle \mathbf{Lx}, \mathbf{Ly} \rangle _W$
Proj onto unit: $\text{proj}_\mathbf{u}(\mathbf{x}) = \langle \mathbf{u}, \mathbf{x} \rangle \mathbf{u}$
THM: $Q$, $Q_1$ orthonormal: (i) $\|Q \mathbf{x}\| = \|\mathbf{x}\| 
(ii) QQ_1$ is orthonormal$
(iii) Q^{-1} = Q^H $is orthonormal$
(iv) Q^HA = QQ^H = I
(v) $columns are orthonormal$
(vi) |det(Q)| = 1$
\textbf{DEF}: Bessel's Inequality: If $(V, \langle .,. \rangle)$ and $\{ \textbf{x}_i \}_{i=1}^m$ is a finite orthonormal set: $\|\textbf{v}\| \geq \sum_{i=1}^m | \langle \textbf{x}_i, \textbf{v} \rangle |^2 = \| \text{proj}_X\textbf{v} \|^2$. 
\textbf{DEF}: Pythagorean Theorem: If $(V, \langle .,. \rangle)$ and $\{ \textbf{x}_i \}_{i=1}^m \subset V$ is a finite orthonormal subset with span $X$, for every $\textbf{v} \in V$: $\|\textbf{v} \|^2 = \|\text{proj}_X(\textbf{v}) \|^2 + \| \textbf{v} - \text{proj}_X(\textbf{v}) \|^2 = \sum_{i=1}^m |\langle \textbf{x}_i, \textbf{v} \rangle |^2 +  \| \textbf{v} - \sum_{i=1}^m \langle \textbf{x}_i, \textbf{v} \rangle \textbf{x}_i \|^2$
Gram-Schmidt: Given $(x_1\dots x_n)$ to get orthonormal basis $(q_1 \dots q_n)$ 1) $q_1 = \frac{x}{\|x\|}$ 
2) $p_1 = \text{proj}_{q_1}(x_2) = \langle \mathbf{q_1}, \mathbf{x_2} \rangle q_1$,
and $q_2 = \frac{x_2 - p_1}{\|x_2 - p_1\|}$
Repeat, $p_{n-1} = \langle \mathbf{q_1}, \mathbf{x_n} \rangle q_1 \dots + \langle \mathbf{q_{n-1}}, \mathbf{x_n} \rangle q_{n-1}$, and $ q_n = \frac{x_n - p_{n-1}}{\|x_n - p_{n-1}}$
QR: $A=QR$ where $Q$ is orthonormal columns of $A$, calculated through Gram-Schmidt.
Note, $A=QR \implies Q^HA=R$. So calculate $R$.
HyperPlane: Defined as perpendicular to a particular $v$, thus $proj_Y(x) = x - \langle \mathbf{v}, \mathbf{x} \rangle \mathbf{v}$ and the transformation is given by $I - 2 \frac{vv^H}{v^Hv}$
NORMS: (i) Positivity $\|x\| \geq 0$ with equality if and only if $x=0$
(ii) Scale preservation $\|ax\| = |a| \|x\|$
(iii) Triangle inequality (Follows from Cauchy Schwarz) $\|x+y\| \leq \|x\|+\|y\|$ Every innerproduct has norm $\|x\| = sqrt{\langle \mathbf{x}, \mathbf{x} \rangle }$
Norms: $\|x\|_p = (\sum |x|^p)^{1/p}$  $\|A\|_F = \sqrt{\text{tr}(A^HA)}$
Induced Norm on Linear Transformations: $\|T\|_{V,W} = sup_{\|x\|_V=1} \|Tx\|_W$
THM:If $T \in \mathscr{B}(X,Y), S\in \mathscr{B}(Y,Z)$ then $ ST \in \mathscr{B}(X,Z)$ and $ \|ST\|_{X,Z} \leq \|S\|_{Y,Z}\|T\|_{X,Y}$
Remark: for $n \geq 1$ we have $\|T^n\| \leq \|T\|^n$. If $\|T\| \leq 1$ then $\|T\|^n$ approaches 0.
EX: $\|A\|_p = sup_{x \neq 0} \frac{\|Ax\|_p}{\|x\|_p}, 1 \leq p \leq \infty$
Young's: $ab \leq \frac{a^p}{p} + \frac{b^q}{q}$ if $\frac{1}{p}+ \frac{1}{q} = 1$
Arithmatic Geotmetric mean: $a^\theta b^{1-\theta} \leq \theta a +(1-\theta)b$ for $0\leq \theta \leq 1$.
Holder's: if $\frac{1}{p} + \frac{1}{q} = 1$ where $1\leq p \leq \infty$ then, $\sum |xy| \leq (\sum|x|^p)^{1/p}(\sum|y|^q)^{1/q} = \|x\|_p\|y\|_q$
(Note, $p=q=2$ implies Cauchy Swarz)
Minkowski: $\|x+y\|_p \leq \|x\|_p + \|y\|_p$
Finite Dimensional Riesz Thm: Let $L:V \rightarrow \mathbb{F}$, $\exists! y \in V s.t. L(x) = \langle \mathbf{y}, \mathbf{x} \rangle \forall x \in V$, and $\|L\| = \|y\| = \sqrt{\langle \mathbf{y}, \mathbf{y} \rangle}$
Adjoint: Adjoint of L is a linear transformation s.t. $\langle \mathbf{w}$,$ \mathbf{Lv} \rangle_W = \langle \mathbf{L^*w}, \mathbf{v} \rangle _v$ $ \forall v \in V, w \in W$. 
THM: Let $L:V\rightarrow W$ be finite, adjoint $L^*$ exists and is unique.
Proof: Let $L_w:V\rightarrow \mathbb{F}$ be defined by $L_w(v) = \langle \mathbf{w}, \mathbf{L(v)} \rangle _W$. By Riesz, $\exists! u \in V s.t. L_w(v) = \langle \mathbf{u}, \mathbf{v} \rangle_V \forall V$. Let $L^*:W \rightarrow V$ be $L^*(w) = u$. Thus, $\langle \mathbf{w}, \mathbf{L(v)} \rangle_W = \langle \mathbf{L^*(w)}, \mathbf{v} \rangle_V \forall v \in V, w \in W$. Show linearity and uniqueness.
Prop 3.7.12 $(S+T)^* = S^* + T^*$ and $(\alpha T)^* = \bar \alpha T^*$ 
OrthComplement of $ S \subset V$ is the set $S^\bot = \{y \in V|\langle \mathbf{x}, \mathbf{y} \rangle = 0, \forall x \in S\}$, 
Note $S^\bot$ is a subset of V. If W is finite dim. subspace of V, then $V = W \oplus W^\bot$. 
Fund. Subspaces: $\mathscr{R}(L)^\bot = \mathscr{N}(L^*)$ and $\mathscr{N}(L)^\bot = \mathscr{R}(L^*)$
COR: Let V,W be finite, $L:V \rightarrow W$ $V = \mathscr{N}(L) \oplus \mathscr{R}(L^*)$ and $W = \mathscr{R}(L) \oplus \mathscr{N}(L^*)$.
Least squares: $\hat{ \mathbf{x}} = (A^HA)^{-1}A^H \mathbf{b}$ is unique minimizer.
Semi-Spectral mapping: If $\lambda_i$ are the eigenvalues of a semisimple matrix $A \in M_n(\mathbb{F})$, and if f(x) is any polynomial, then $\{f(\lambda_i\}$ are the eigenvalues of $f(A)$.
ExcerCh3: For reals, $\langle \mathbf{x}, \mathbf{y} \rangle = \frac{1}{4}(\|x+y\|^2 - \|x-y\|^2)$ and $\|x\|^2 +\|y\|^2 = \frac{1}{2}(\|x+y\|^2 + \|x-y\|^2)$
CH4: Eigenvalues and eigenvectors depend only on the linear transformation and not the choice of basis of it. Let $C_{TS}$ be the transition matrix, $A \in S, B \in T$ s.t. $[x]_T = C[x]_S$ and $B = CAC^{-1}$. Thus, 
$B[x]_T = C A C^{-1} C[x]_S = C\lambda[x]_S = \lambda C[X]_S = \lambda[X]_T$
THM: Following are equivalent, (i) $\lambda$ is an eigenvalue (ii) There is a nonzero x such that $(\lambda I -A)x = 0$ (iii) $\Sigma_\lambda(A) \neq \{\mathbf{0}\}$ (iv) $\lambda I -A$ is is singular, thus det = 0.
Prop: If A, B are similar, that is $A = PBP^{-1}$. They are the same operator, and 
(i) have the same charcatistic poly, eigenvalues, the eigenbases are isomorphic.
Invariant: A subspace is invariant if for $L:V \rightarrow V$, $L(W) \subset W$. 
A is simple if eigenvalues are distinct, semi-simple if eigenbasis spans A. Diagonalizable iff semisimple. $A=PDP^{-1}$. P is eigenvectors, D is eigenvalues. 
Fibonnaci Numbers, Make the matrix, calculate eigens, note 300th number is
$v_{301} = Av_{300} = A^{300}v_0 = P^{-1}D^{300}Pv_0$ where $v_0$ is starting vector of sequence, of course, round to nearest integer.
Power Method: pick vector, muliply by A, normalize, iterate. Implies dominant eigenvector and value if semi-simple.
Rayleigh quotient does the same thing, but faster. $\frac{\langle \mathbf{x}, \mathbf{Ax} \rangle }{\|x\|^2}$ Implies $\lambda$ and eigenvector, which converge to $^*$some$^*$ eigenvalue and vector.
Orthonormally similar if $B=U^HAU$, U is an orthonormal iff $\langle \mathbf{Ux}, \mathbf{Uy} \rangle = \langle \mathbf{x}, \mathbf{y} \rangle = x^Hy$.
Hermitian matricies are linear operators that preserve length and angle on different bases.
Lem: If A is Hermitian, ortho similar to B, then B is Hermitian.
Schur's Lemma: Every $nxn$ matrix, A, is orthonormally similar to an upper triangular matrix. Proved by induction.
Spct Thm I: Every Hermitian matrix A is orthonormally similar to a real diagonal matrix. 
Proof: By Schur's, A is ortho to an upper triangular, T. Since A is hermitian, so is T, and $T^H = T$ thus all eigenvalues(diagonals) are real.
Normal Matrix: Spct Thm holds for all Normals. Normal when $A^H = AA^H$.
Spct Thm II: A matrix A is normal iff it is orthonormally similar to a diagnoal matrix, equivalently, if it has an orthonormal eigenbasis. Proof: by Shur's T is uppertriangular
Other direction, just multiply out. 
Positive Definite: Has positive eigen values,there exists a unique lower triangular matrix L, with real and strictly positive diagonal elements such that $M = LL^*$, that's cholesky decomp.
It is invertible, and it's inverse is positive definite. Sum and product of semi definites are semi definite. $Q^TMQ$ is positive semi definite.
Determinant is bounded by product of diagonal elements.
$A = S^HS$, and if A is definite, S is nonsingular.
SVD: if A is of rank r, $\exists \text{orthonormal} U, V$ and real valued $\Sigma = diag(\sigma_1,\dots,\sigma_r, \dots,0,0,)$ s.t. $A = U\Sigma V^H$ where $\sigma_i$ is positive real valued. $\Sigma$ is unique. To calculate, $\Sigma = \sqrt{\lambda_i} $ of $A^HA$. V is construction of eigenvectors of $A^HA$, with unfilled spots filled by gramschmidt (if eigenbasis doesn't exist). U is determined by $u_i = \frac{1}{\sigma}Av_i$ where $u_i, v_i$ are columns of $U,V$ respectively. Compact form is where we drop all zeros of sigma, and fit U, V accordingly.
Moore-Penrose: $A^\dagger = V_1 \Sigma^{-1}_1 U_1^H$ of compact, or $V\Sigma^\dagger U^*$ where $\Sigma^\dagger$ is recipricol of non-zeros on diagonal, transpose. 
Schmidt-Eckart-Young: for A of rank r, and each s<r, $\sigma_{s+1} = inf_{rank(B) =s}\|A-B\|_2$ 
with minimizer $ B^\diamond = \sum_{i=1}^s \sigma_iu_iv_i^H$ where each $\sigma$ is the singular value of A, with corresponding u,v columns of U,V in the compact form of SVD.
Scmidt-Eckart-Young-Mirsky: $(\sum_{j=s+1}^r)^{1/2} = inf_{rank(B)=s)}\|A-B\|_F $
Cor: $A = U\Sigma V^H$ is the SVD where A has rank r >s then for any $mxn \Delta$ such that $A+\Delta$ has rank s we have: $\|\Delta\|_2 \geq \sigma_{s+1}$ and $\|\Delta\|_F \geq (\sum^r_{k=s+1}\sigma_k)^{1/2}$ and equality holds if $\Delta = -\sum^r_{i=s+1} \sigma_i u_iv_i^H$.
The infimum of $\|\Delta\|$ rank$(I-A\Delta) < m $is $i/\sigma_1$ with minimizer $\Delta^* = \frac{1}{\sigma_1}v_1u_1^H$
Small gains: if $A \in M_n$, then $I-A\Delta$ is nonsingular, provided that $\|A\|_2\|\Delta\|_2 <1$
For $A \in M_{mxn}(\mathbb{F})$: (i) $\|A\|_2 = \sigma_1$(largest signular value) (ii) if A is invertible, then $\|A^{-1}\|_2 = \frac{1}{\sigma_n}$ (iii) $\|A^H\|^2_2 = \|A^T\|^2_2 = \|A^HA\|_2 = \|A\|^2_2 $ (iv) if U, V are orthonormal then $\|UAV\|$ (v) $\|UAV\|_2 = \|A\|_F$ (vi) $\|A\|_F = (\sigma_1^2+\dots+\sigma_r^2)^{1/2}$. (vii) $|det(A)| = \Pi \sigma_i$. 
The SVD gives an orthonormal basis of four fundamental subspaces. first r columns of V are basis of $\mathscr{R}(A^H)$, the last n-r columns of V span $\mathscr{N}(A)$, first r columns of U span $\mathscr{R}(A)$ last m-r span $\mathscr{N}(A^H)$. 
Remark: Let V,W be normed linear spaces, if $L \in \mathscr{B}$ then the induced norm on L satisfies $\|Lx\|_W \leq \|L\|_{V,W}\|x\|_V$.
Fun Fact: $0 \leq \langle \mathbf{x - \lambda y}, \mathbf{x - \lambda y} \rangle = \langle \mathbf{x}, \mathbf{x} \rangle - \langle \mathbf{\lambda y}, \mathbf{x} \rangle - \langle \mathbf{x}, \mathbf{\lambda y } \rangle + \langle \mathbf{\lambda y }, \mathbf{\lambda y} \rangle = \langle \mathbf{x}, \mathbf{x} \rangle -\hat \lambda \langle \mathbf{y}, \mathbf{x} \rangle - \lambda \langle \mathbf{x}, \mathbf{y} \rangle + \hat \lambda \lambda \langle \mathbf{y}, \mathbf{y} \rangle $. 
Polar Decomp: If $A \in M_{mxn}(\mathbb{F})$ and $m \geq n$. Then there exists a Q with orthonormal columns and positive semidefinite P such that $ A=PQ$. Because $A = U\Sigma V^H = U V^H V \Sigma V^H$. Let $Q = UV^H$ and $ P = V\Sigma V^H$ Norms: 1-Norm max sum of columns. $\infty$-Norm max sum of rows. 2-Norm is largest singular value. 
Semi-Simple: Diagonalizable, Eigenbasis exists, distinct eigenvalues iff simple. Semi Positive Definite: $ \langle \mathbf{x}, \mathbf{Ax} \rangle \geq 0$, non-negative eigenvalues, chebesky decomp, sub-matrix positive, sort of kind of hermitian. 
Normal: Orthonormally similar to Diagonal matrix, $A^HA = AA^H$, Hermetians, skews, orthonormals... $\mathbf{EXAM 3}$ Def. Metric (i) Positive definite, (ii) symmetry, (iii) Triangle inequality, $d(x,y) \leq d(x,z) + d(z,y)$. 
THM. The union of any collection of open sets is open, intersection of finite open sets is open.
Def. Continuous I Let $(X,d), (Y,\rho)$ be metric spaces. A function $f: X \to Y$ is continuous at a point $x_0 \in X$ if for all $\epsilon > 0$ $\exists \delta > 0$ s.t. $\rho(f(x), f(x_0)) < \epsilon$ whenever $d(x,x_0) < \delta$.
Let $(X,d), (Y,\rho)$. A function $f: X \to Y$ is continuous on X iff $f^{-1}(U)$ of every open set $U \subset Y$ is open in X.
Continuous II: Let (X,d) and $(Y,\rho)$. A function $f: X \to Y$ is continous at $x_0 \in X$ iff $\forall \epsilon >0,~\exists \delta > 0$ s.t. $f(B(x_0,\delta))\subset B(f(x_0),\epsilon)$
Def. Limit Point, p, if every neighborhood of p intersects E\ {p}
Thm. Let (X,d) be a metric space. A set $U \subset X$ is open iff complement $U^c$ is closed.
THM. Intersection of closed sets is closed, union of finite collection of closed sets is closed.
Ex. $f(x,y) = x^2y^x/(x^2+y^2)^2$ is continuous at zero. Note, $\|x\|\leq (x^2 + y^2)^2$, $\|y\|\leq (x^2 + y^2)^2$, and thus $\|x^2y^3\|\leq(x^2+y^2)^{\frac{5}{2}}$
Cluster Point if for all $\epsilon > 0$, and $N>0$, $\exists n \geq N$ s.t. $d(x,x_n) < \epsilon$
$f(x,y) = xy/(x^2+y^2)$ is not continuous at zero, because subsequnces $x_n = (1/n, 1/n)$ converge to 1/2 for all n, but limit$x_n = 0$.
Thm. For any y in metric space and sequence $(x_n)_{n=0}^\infty$ in X that converges to x, we have $\lim_{n\to \infty} d(x_n,y) = d(\lim_{n\to \infty} x_n, y) = d(x,y)$, or if the metric is defined by a norm d(x,y) = |x-y|, then limit commutes.
Def. Cauchy. A sequence is Cauchy if $\forall \epsilon > 0$, $\exists N>0$ s.t. $d(x_m,d_n) < \epsilon$ whenver m,n>N.
Convergent and bounded.
Convergence is preserved under continiuity, cauchy sequences are preserved under uniform continuity. A metric space is complete if every Cauchy sequence converges. 
Cartesian products of complete metrics are complete.
Def. Uniform Continous if $\forall \epsilon > 0$ $\exists \delta > 0$ s.t. distince is $>\epsilon$ whenever $d(x,y) > \delta$
Function uniform on closed intervals, but ont on R. 
Lipshittz (locally Lipschitz) is bounded, which is great. If $f:X\to Y$ is a bounded linear transformation of normed linear spaces, then f is uniformly continuous.
A set is compact if every open cover has a finite subcover. (Heine) Closed and bounded is compact. Continuous image of compact set is compact.
Chacterizations of Compactness(i): If every finite subcollection has a non-empty intersection. (ii) If every infinite sequence has at least one cluster point. (iii) Bolzano Weistrass property. (iv) Sequentially compact, (v) X is totally bounded and every open voever has a positive lebesgue number.
A metric space is compact is it is complete and totally bounded.
Pointwise looks at each point, Uniform convergence looks at the infinity norm of the whole set, and that must be zero.
A complete, normed, linear space is Banach.
$exp(A) = \sum_{k=0}^\infty \frac{A^k}{k!}\leq e^{\|A\|} < \infty$ for any bounded operator A.
$\sum_{k=0}^\infty \|A^k\| \leq \frac{1}{1-\|A\|} < \infty$ for bounded cuntions and $(1-A)^{-1} = \sum_0^\infty A^k$.
If $\|A^{-1}<M$, then for $\|E\| < 1/ \|A{-1}\|$, $\|(A +E)^{-1}\| \leq \|A^{-1}\| \ 1-\|E\| \|A^{-1}\|$. 
Def Homeomorphism is a bijective, continuous map $f:(X,d) \to (Y,p)$ whose inverse is continuous. 
Topological properties open sets, compactness, convergence, Connectedness
Not topological: Completeness, Cauchy (unless using topologically equivalent norms)
Def Connected: A set is disconnected if there are disjoint nonempty open subsets $G_1, G_2$ such that $X = G_1 \bigcup G_2$. X is connected if not disconnected. If X is disconnected, $G_1, G_2$ are both open and closed. Hence a space X is connected iff the only sets that are open and closed are X and empty set.
Thm If$ f :X\to Y$ and X is connected, so is Y. 
Thm Path connected space is connected, but connected space is not necessarily path connected.
IVT: Assume X is connected and $f: X \to \mathbb{R}$ is continuous. if $f(x) < f(y)$ and $c \in (f(x), f(y))$, $\exists z \in X$ s.t. $f(z) = c$. 
A continuous real valued map f on the unit circle has anti-podal points that are equal. 
Regulated Integral: Bounded functions are a Banach space.
Continuous LInear Ext: Let $(Z, |\cdot|_z)$ be a normed linear space, X a banach space and $S \subset Z$ be desne. If $T: S \to X$ is a bounded linear transformation, then  T has a unique linear extention to $\bar T$ satisfying $\| \bar T\| = \|T\|$.
%%TODO MAke this definition of reg. integral better. I don't want to RN
Single Variable Banach Reg. Integral: Integral is basically just a bounded linear thing 
Properties of the integral: %%TODO add these. I don't have the will
$\mathbf{Ch. 6}$ The directional derivative of $f$ at $x$ with respect to $v$ is the limit $\lim _{t\to 0}\frac{f(x+tv) - f(x)}{t}$ this limit is often denoted $D_vf(x)$. \textbf{partial derivatives} the ith partial derivative of f at the point $x$ is given by the limit $D_if(x) = \frac{f(x+he_i) - f(x)}{h}$. $\textbf{Frechet derivative}^*$: Let $(X, ||.||)_X$ and $(Y, ||.||)_Y$ be banach spaces and let $U \subset X$ be an open set. A map $f$ is differentiable at $x \in U$ if there exists a bouned linear transformation $Df(x):X\to Y$ s. t. $\lim_{h\to 0} \frac{||f(x+h) - f(x) - Df(x)h||_Y}{||h||_X}=0$.
 X and Y are banach spaces, $U \subset X$ be an open set, and f be differentiable on U. if Df given by $x \to Df(x)$ is also continuous, we say that f is continously differentiable on U.  if f is differentiable on U, then f is locally \textbf{lipshitz}, that is $\forall~x_0 \in U~ \exists B(x_0, \delta) \subset U$ and $L >0$ s.t. $||f(x) - f(x_0)||_Y \leq L||x -x_0||_X$ whenever $||x-x_0||_X < \delta$.

\textbf{Mean Value Theorem:} Let $(X, \| \cdot \| _X)$ be a Banach space, $U \subset X$ be an open set, and $f:U \rightarrow \mathbb{R}$ be differentiable on $U$. If for $\textbf{x}, \textbf{x} \in U$, the entire line segment $\ell(\textbf{x}, \textbf{x}) := \{(1-t)\textbf{x} + t \textbf{x} \in [0,1]\}$ is also in $U$, then there exists $\textbf{c}\in \ell(\textbf{x}, \textbf{x})$ such that: $f(\textbf{y}) - f(\textbf{x}) = Df(\textbf{c})(\textbf{y} \textbf{x})$. 
\textbf{Fundamental Theorem of Calculus:} Let $(X, \| \cdot \| _X)$ be a Banach space: (i) If $f\in C([a,b];X)$, then for all $t \in (a,b)$ we have that: $\frac{d}{dt} \int_a^t f(s)ds = f(t)$ (ii) If $F:[a,b] \rightarrow X$ is continuously differentiable on $(a,b)$ and $DF(t)$ extends to a continuous function on $[a,b]$, then: $\int_a^b DF(s)ds = F(b) - F(a)$. 
\textbf{Integral Mean Value Theorem:} Let $(X, \| \cdot \| _X)$ and $(Y, \| \cdot \| _Y)$ be a Banach spaces, $U \subset X$ be an open set, and $f:U \rightarrow Y$ be continuously differentiable on $U$. If the line segment $\ell(\textbf{x}, \textbf{x}) = \{t \textbf{x} + (1-t) \textbf{x} | t \in [0,1] \}$ is contained in $U$, then: $f(\textbf{x}) - f(\textbf{x} = \int_0^1 Df(t \textbf{y} + (1-t)\textbf{x}) (\textbf{x} - \textbf{x})dt$, alternatively if we let $\textbf{y} = \textbf{x} + \textbf{h}$, then $f(\textbf{x} + \textbf{h}) - f(\textbf{x}) = \int^1_0 Df(\textbf{x} + t \textbf{h}) \textbf{h} dt$ or $\| f(\textbf{x}) - f(\textbf{x}) \|_Y \leq \sup_{\textbf{c}\in \ell(\textbf{x}, \textbf{x})} \| Df(\textbf{c}) \|_{X,Y} \| \textbf{x} - \textbf{x} \|_X $ 
\textbf{Change of Variable Formula:} Let $(X, \| \cdot \| _X)$ be a Banach space and $f\in C([a,b];X)$. If $g:[c,d] \rightarrow [a,b]$ is continuous and $g'$ is continuous on $(c,d)$ and can be continuously extended to $[c,d]$, then $\int_c^d f(g(s))g'(s)ds = \int_{g(c)}^{g(d)} f(\tau) d\tau$

$\textbf{Lemma}^*$: Given $f:X \to Y$ and linear $L: X\to Y$, to prove that $f$ is differentiable at $x$ with derivative $L$ is equivalent to proving that for every $\epsilon >0$ there is $||\xi||<\delta$ we have $||f(x + \xi) - f(x) - L \xi||_Y \leq \epsilon ||\xi||_X$. $\mathbf{Linearity}$: assume X and Y are Banach spaces, that U is an open neighborhood in X, and $f: U \to Y$ and $g: U \to Y$. If f and g are differentiable on U and $a,b \in \mathbb{F}$, then $af +bg$ is also differentiable on U, and $D(af(x) + bg(x)) = aDf(x) +bDg(x)$. \textbf{Product Rule}: Let X be banach space, that $U$ is an open neighborhood of $X$, and that $f:U\to \mathbb{F}$ and $g:U \to \mathbb{F}$ if $f$ and $g$ are differentiable on U, then the product map $fg$ is also differentiable on $U$ and $D(f(x)g(x)) = g(x)Df(x)+ f(x)Dg(x)$ for each $x \in U$. Rules of differentiation: (i) $u(x), v(x)$ are differentaible from $R^n$ to $R^m$ and $f: \mathbb{R}^n \to\mathbb{R}$ then $Df(x) = u(x)^TDv(x) +v(x)^T Du(x)$. (ii) if $g:\mathbb{R}^n \to \mathbb{R}$ s.t. $g(x) = x^TAx$ then $Dg(x) = x^T(A+A^T)$.\textbf{Chain Rule}: Assume X, Y, and Z are Banach, U and V are open neighborhoods of X and Y respectively, and $f: U\to V$ and $g:V\to Z$ with $f(U) \subset V$ if f is differentiable on U and g is differentiable on V, then the composite map $h = f \circ g$ is also differentiable on U and $Dh(x) = Dg(f(x))Df(x)$, for each $x\in U$



 $\mathbf{Sec 6.5}$ 
 Let $U \subset R^n$ be open. If $f: U \to \mathbb{R}^m$ is twice continuously differentiable on U, then $D^2f(x)(x_1,x_2) = D^2f(x)(x_2,x_1)$. Or $\frac{\partial^2 f_k}{\partial x_i \partial x_j} = \frac{\partial^2 f_k}{\partial x_j \partial x_i}$
 Taylor's Theorem: For X,Y be Banach. $U \subset X$ an open set, and $f: U \to Y$ be k times differentiable. If $x \in U$ and $h \in X$ are such that the line $l(x, x+h)$ is contained in U, then $f(x+h) = f(x) + Df(x)h + \frac{D^2f(x)h^2}{2!} + R_k$. Example: for $f(x,y) = e^{x+y}$ at $(0,0)$ in direction of $(h_1,h_2)$. Note, $f(0,0) = 1$. $D_hf(x) = \nabla f(0,0) \cdot h = h_1D_1(0,0) + h_2D_2(0,0) = h_1 + h_2$. 



}
\end{document}
