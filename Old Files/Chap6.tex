\documentclass[letterpaper,12pt]{article}

\usepackage{threeparttable}
\usepackage{geometry}
\geometry{letterpaper,tmargin=1in,bmargin=1in,lmargin=1.25in,rmargin=1.25in}
\usepackage[format=hang,font=normalsize,labelfont=bf]{caption}
\usepackage{amsmath}
\usepackage{multirow}
\usepackage{array}
\usepackage{delarray}
\usepackage{amssymb}
\usepackage{amsthm}
\usepackage{lscape}
\usepackage{natbib}
\usepackage{setspace}
\usepackage{float,color}
\usepackage[pdftex]{graphicx}
\usepackage{pdfsync}
\usepackage{verbatim}
\usepackage{placeins}
\usepackage{geometry}
\usepackage{pdflscape}
\synctex=1
\usepackage{hyperref}
\hypersetup{colorlinks,linkcolor=red,urlcolor=blue,citecolor=red}
\usepackage{bm}


\theoremstyle{definition}
\newtheorem{theorem}{Theorem}
\newtheorem{acknowledgement}[theorem]{Acknowledgement}
\newtheorem{algorithm}[theorem]{Algorithm}
\newtheorem{axiom}[theorem]{Axiom}
\newtheorem{case}[theorem]{Case}
\newtheorem{claim}[theorem]{Claim}
\newtheorem{conclusion}[theorem]{Conclusion}
\newtheorem{condition}[theorem]{Condition}
\newtheorem{conjecture}[theorem]{Conjecture}
\newtheorem{corollary}[theorem]{Corollary}
\newtheorem{criterion}[theorem]{Criterion}
\newtheorem{definition}{Definition} % Number definitions on their own
\newtheorem{derivation}{Derivation} % Number derivations on their own
\newtheorem{example}[theorem]{Example}
\newtheorem{exercise}[theorem]{Exercise}
\newtheorem{lemma}[theorem]{Lemma}
\newtheorem{notation}[theorem]{Notation}
\newtheorem{problem}[theorem]{Problem}
\newtheorem{proposition}{Proposition} % Number propositions on their own
\newtheorem{remark}[theorem]{Remark}
\newtheorem{solution}[theorem]{Solution}
\newtheorem{summary}[theorem]{Summary}
\bibliographystyle{aer}
\newcommand\ve{\varepsilon}
\renewcommand\theenumi{\roman{enumi}}
\newcommand\norm[1]{\left\lVert#1\right\rVert}

\begin{document}

\iffalse
\section{First Section}\label{SecFirst}

  Put first section text here. You can even reference Section \ref{SecFirst} with hyperlinks.

  Here is an equation.
  \begin{equation}\label{EqPopDef}
    N_t\equiv\sum_{s=1}^{E+S} \omega_{s,t} \quad\forall t
  \end{equation}

  \begin{equation}\label{EqPopDef}
    B_t\equiv\sum_{s=1}^{E+S} \omega_{s,t} \quad\forall t
  \end{equation}


  Below is some commented out text for a figure. I have commented it out because I didn't include the figure file with this template. Learn everything else on your own.
\fi


  % \begin{figure}[htb]\centering \captionsetup{width=4.0in}
  %   \caption{\label{FigLogAbility}\textbf{Exogenous life cycle income ability paths $\log(e_{j,s})$ with $S=80$ and $J=7$}}
  %   \fbox{\resizebox{4.0in}{2.0in}{\includegraphics{images/ability_log_2D_no_vline.png}}}
  % \end{figure}


\end{document}
